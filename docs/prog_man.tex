% 
% prog_man.tex - The IRIT programmer's manual - main module.
% 
% Author:	Gershon Elber
% 		Computer Science Dept.
% 		Technion, IIT
%

%\documentstyle[10pt]{book}
\documentstyle{book}

\setlength{\oddsidemargin}{0.0in}
\setlength{\evensidemargin}{0.0in}
\setlength{\topmargin}{-0.2in}
\setlength{\textwidth}{6.5in}
\setlength{\textheight}{9.0in}

\makeindex

\begin{document}

\small
\setlength{\baselineskip}{0.85\baselineskip}

\tableofcontents

%%%%%%%%%%%%%%%%%%%%%%%%%%%%%%%%%%%%%%%%%%%%%%%%%%%%%%%%%%%%%%%%%%%%%%%%%%%%%
\chapter{Introduction}

This manual describes the different libraries of the IRIT solid
modeling environment. Quite a few libraries can be found to manipulate
geometry in general, freeform curves and surfaces, symbolic
computation, trimmed surfaces, triangular patches, freeform trivariate
functions, Boolean operations, input output data file parsing, and
miscelleneous.

All interface to the libraries should be made via the appropriate header
files that can be found in the include subdirectory. Most libraries
have a single header file that is named the same as the library.
Functions and constants that are visible to the users of the libraries
are prefixed with a unique prefix, usually derived from the library
name itself. External definitions that start with an underscore should
not be used, even if found in header files.

The header file {\bf include/irit\_sm.h} must be sourced by every
source file in the solid modeller. In most cases, this file is sourced
indirectly via local header files.

The following libraries are avaliable in IRIT:
\begin{center}
\begin{tabular}{|l|l|} \hline
    Name of Library & Tasks \\ \hline
    bool & Boolean operations on polygonal models. \\
    cagd & Low level freeform curves and surfaces. \\
    geom & General geometry functions. \\
    grap & General graphics/display functions. \\
    mdl  & Model's processing functions. \\
    misc & Memory allocation, configuration files, attributes, etc. \\
    mvar & Multi variate functions. \\
    prsr & Input and output for file/sockets of objects of IRIT. \\
    rndr & Scan conversion rendering functions. \\
    symb & Symbolic manipulation of curves and surfaces. \\
    trim & Trimmed surfaces support. \\
    triv & Freeform trivariate functions. \\
    trng & Triangular patches support. \\
    user & General high level user inteface functions. \\
    xtra & Public domain code that is not part of IRIT. \\ \hline
\end{tabular}
\end{center}

%%%%%%%%%%%%%%%%%%%%%%%%%%%%%%%%%%%%%%%%%%%%%%%%%%%%%%%%%%%%%%%%%%%%%%%%%%%%%

\subsection{Tools and Programs}

The IRIT package includes several complete programs such as poly3d-h
(hidden line removal), irender (scan conversion tool), and irit2ps (a
filter to Postscript).  Somewhat different than most other programs is
the kernel interpreter, also called irit.  The irit program is nothing
more than an interpreter (written in C) that invokes numerous
functions in the several libraries provided.  In order to add a new
function to the irit interpreter, the following sequence of operations
must be followed:

\begin{enumerate}
\item Write a C function that accepts only one or more of the following
     type of parameters.  All parameters, including the IrtRType, {\em must}
     be transferred by address:
    \begin{itemize}
	\item IrtRType         (see irit\_sm.h).
	\item IrtVecType       (see irit\_sm.h).
	\item IrtPtType        (see irit\_sm.h).
	\item CagdCtlPtStruct  (see cagd\_lib.h).
	\item IrtPlnType       (see irit\_sm.h).
	\item StringType       (char *).
	\item IPObjectStruct   (see iritprsr.h).  This includes all object
		types in IPObjectStruct other than the above.
    \end{itemize}
\item The written C function can return one of:
    \begin{itemize}
	\item IrtRType by value.
	\item IPObjectStruct by address.
	\item Nothing (a procedure).
    \end{itemize}
\item According to the returned type by the new C function, go to file
	{\bf inptprsl.h}, and add a new enum NEW\_FUNC for the new function in
	enum RealValueFuncType (for IrtRType returned value), in
	enum ObjValueFuncType (for IPObjectStruct * returned value), or in
	enum GenValueFuncType (for no returned value).
\item In {\bf inptevl1.c}, add one new line in one of NumFuncTable,
	ObjFuncTable, or GenFuncTable (depends upon the return value).
	The line will have the form of:
	\begin{verbatim}
            { ``FuncName'', NEW_FUNC, CFunctionName,  N, { Param1Type,
                                              Param2Type, ... ParamNType } }
	\end{verbatim}
	for procedures with no return values, and of the form of:
	\begin{verbatim}
            { ``FuncName'', NEW_FUNC, CFunctionName,  N, { Param1Type,
                                    Param2Type, ... ParamNType }, RetValType }
	\end{verbatim}
	otherwise. \verb-N- is the number of parameters,
	\verb-`FuncName''- is the (unique) name that will be used in
	the interpreter, CFunctionName is the name of the C function you
	have written. This C function must be declared in one of the header
	files that inptevl1.c includes. ParamIType, for I between 1 and N,
	and RetValType are of type IritExprType (see inptprsl.h).
\item Thats it!
\end{enumerate}

For example, to add the C function declared as
\begin{verbatim}
    IPObjectStruct *PolyhedraMoments(IPObjectStruct *IPObjectStruct,
                                     IrtRType *m);
\end{verbatim}
one has to add the following line to ObjFuncTable in iritevl1.c
\begin{verbatim}
    { ``PMOMENT'', PMOMENT, { POLY_EXPR, REAL_EXPR }, VECTOR_EXPR },
\end{verbatim}
where PMOMENT needs to be added to ObjValueFuncType in iritprsl.h.

While all objects in the interpreted space are of type IPObjectStruct,
the functions' interface unfolds many types to simplify matter.
IrtRType, IrtVecType, Strings, etc. are all extracted from the given
(IPObjectStruct) parameters and passed directly by address.  For
returned values, only numeric real data is allowed where everything
else must be returned wrapped in an IPObjectStruct.

%%%%%%%%%%%%%%%%%%%%%%%%%%%%%%%%%%%%%%%%%%%%%%%%%%%%%%%%%%%%%%%%%%%%%%%%%%%%%

The following chapters reference the different functions of the different
libraries. Chapter~\ref{chap-examples} provides several examples of
writing C code using the IRIT libraries.

%%%%%%%%%%%%%%%%%%%%%%%%%%%%%%%%%%%%%%%%%%%%%%%%%%%%%%%%%%%%%%%%%%%%%%%%%%%%%
\chapter{Boolean Library, bool\_lib}

\section{General Information}

On of the crucial operation in any solid modeling environment is the
ability to perform Boolean operations among different geometric
objects.  The interface of the library is defined in {\em
include/bool\_lib.h}. This library supports only Boolean operations of
polygonal objects. The Boolean operations of OR, AND, SUBtract,
NEGate, CUT, and MERGE are supported via the {\bf BoolOperType}
typedef:
\begin{center}
\begin{tabular}{||c|c||} \hline 
    BoolOperType & Meaning \\ \hline
    BOOL\_OPER\_OR & Union of two geometrical objects \\
    BOOL\_OPER\_AND & Intersection of two geometrical objects \\
    BOOL\_OPER\_SUB & The difference of two geometrical objects \\
    BOOL\_OPER\_NEG & Unary Inside-out of one geometrical object \\
    BOOL\_OPER\_CUT & Boundary of one object outside the other \\
    BOOL\_OPER\_MERGE & Simple merge without any computation \\ \hline
\end{tabular}
\end{center}

The {\bf BoolOperType} typedef is used in two dimensional Boolean operations
invoked via {\bf Boolean2D}. Three dimensional Boolean operations are
invoked via {\bf BooleanXXX} where XXX is one of OR, AND, SUB, NEG, CUT
or MERGE, denoting the same as in the table above.

In addition several state functions are available to control the way
the Boolean operations are conducted. Functions to enable the dump of
(only) the intersection curves, to handle coplanar polygons, and to
set the axis along which the sweep is performed are provided.

All globals in this library have a profix of {\bf Bool}.

%%%%%%%%%%%%%%%%%%%%%%%%%%%%%%%%%

\section{Algorithmic Hints Behind the Boolean Operations}

Let ${\cal M}_i,~i = 1,2$ be two polygonal models to compute a Boolean
operation between and let $P_m$ denote the $m$'th polygon of ${\cal M}_i$.
Denote by $InterList( P_m ),~P_m \in {\cal M}_i$, the list of line
segments representing the intersections of polygon $P_m$ with polygons
in ${\cal M}_j$.  A naive first, intersection, step in a Boolean
operation between two polyhedra models, ${\cal M}_1$ and ${\cal M}_2$,
will consist of the following stages:
\begin{bf}
\begin{tt}
\begin{tabbing}
\hspace{2em}\=\hspace{2em}\=\hspace{2em}\=\hspace{2em}\=\hspace{2em}\=\hspace{2em}\=\kill
\> For each polygon $P_m$ in ${\cal M}_1$ \\
\> begin \\
\> \> $InterList( P_m )$ $\Leftarrow$ $\emptyset$ \\
\> end \\
\> For each polygon $P_n$ in ${\cal M}_2$ \\
\> begin \\
\> \> $InterList( P_n )$ $\Leftarrow$ $\emptyset$ \\
\> end \\
\\
\> For each polygon $P_m$ in ${\cal M}_1$ \\
\> begin \\
\> \> For each polygon $P_n$ in ${\cal M}_2$ \\
\> \> begin \\
\> \> \> If ( $P_m \cap P_n$ ) \\
\> \> \> begin \\
\> \> \> \> ${\cal L}$ $\Leftarrow$ $P_m \cap P_n$; \\
\> \> \> \> $InterList( P_m )$ $\Leftarrow$ $InterList( P_m )$ $\cup$ ${\cal L}$; \\
\> \> \> \> $InterList( P_n )$ $\Leftarrow$ $InterList( P_n )$ $\cup$ ${\cal L}$; \\
\> \> \> end \\
\> \> end \\
\> end
\end{tabbing}
\end{tt}
\end{bf}

At the end of the first, intersection stage, each polygon, $P_m$ in
either ${\cal M}_1$ (or $P_n \in {\cal M}_2$) contains in its
$InterList( P_m )$ the set of line segments of the intersection(s) of
$P_m$ with all polygons in the other model.  In order to figure out
orientation, each intersection line segment also contains an
orientation hint as to the inside direction of the two polygons that
intersect at that line segment.  This list of intersection line
segments can now be processed:

\begin{bf}
\begin{tt}
\begin{tabbing}
\hspace{2em}\=\hspace{2em}\=\hspace{2em}\=\hspace{2em}\=\hspace{2em}\=\kill
\> For each polygon $P_m$ in ${\cal M}_1$ \\
\> begin \\
\> \> $Polys$ $\Leftarrow$ $InterList( P_m )$ sorted into polylines; \\
\> \> $OpenList( P_m )$ $\Leftarrow$ Open polylines of $Polys$; \\
\> \> $ClosedList( P_m )$ $\Leftarrow$ Closed polylines of $Polys$; \\
\> end \\
\> For each polygon $P_n$ in ${\cal M}_2$ \\
\> begin \\
\> \> $Polys$ $\Leftarrow$ $InterList( P_n )$ sorted into polylines; \\
\> \> $OpenList( P_n )$ $\Leftarrow$ Open polylines of $Polys$; \\
\> \> $ClosedList( P_n )$ $\Leftarrow$ Closed polylines of $Polys$; \\
\> end \\
\end{tabbing}
\end{tt}
\end{bf}

Open polylines, $OpenList( P_m )$, are intersection polylines that are
open.  These polylines starts on the boundary of polygon $P_m$ and
also ends on the boundary, at a different boundary location though.
In contrast, the set of closed polylines, $ClosedList( P_n )$, contains
closed loops that are completely contained in $P_n$.

The polygons are now ready for {\em trimming}.  Depending upon the Boolean
operation that is requested, the inside or the outside of ${\cal M}_1$ or
${\cal M}_2$ is selected for the output result.  Denote by
${\cal M}_i \ Out \ {\cal M}_j$ (${\cal M}_i \ In \ {\cal M}_j$) the region of
${\cal M}_i$ that is outside (inside) ${\cal M}_j$. Then,

\bigskip 
\begin{center}
\begin{tabular}{||c|c||} \hline
Boolean Operation & implementation as \\ \hline \hline
${\cal M}_1 \cup {\cal M}_1$ & $\{ {\cal M}_1\ Out\ {\cal M}_2 \} \cup
			     \{ {\cal M}_2\ Out\ {\cal M}_1 \}$ \\
${\cal M}_1 \cap {\cal M}_1$ & $\{ {\cal M}_1\ In\ {\cal M}_2 \} \cup
			     \{ {\cal M}_2\ In\ {\cal M}_1 \}$ \\
${\cal M}_1 - {\cal M}_1$ & $\{ {\cal M}_1\ Out\ {\cal M}_2 \} \cup
			     Inverse\{ {\cal M}_2\ In\ {\cal M}_1 \}$ \\ \hline
\end{tabular}
\end{center}
\bigskip 

Note the $Inverse$ operation that reverses the orientation inside
out.

At this point, all Boolean operations are translated into two complementary
type of operations, $In$ and $Out$. The computation of
${\cal M}_i \ Out \ {\cal M}_j$ is conducted by tracing the outside regions
of the polygons of ${\cal M}_i$ that {\em intersect} ${\cal M}_j$ and
{\em propagating} this inside--outside information to the non intersecting
polygons:
\begin{bf}
\begin{tt}
\begin{tabbing}
\hspace{2em}\=\hspace{2em}\=\hspace{2em}\=\hspace{2em}\=\hspace{2em}\=\kill
\> For each polygon $P_m$ in ${\cal M}_1$ with non empty $OpenList( P_m )$ \\
\> \mbox{\hspace{2.425in}} or non empty $ClosedList( P_m )$ \\
\> begin \\
\> \> $P_m^{in}$ $\Leftarrow$ Inside region of $P_m$; \\
\> \> $P_m^{out}$ $\Leftarrow$ Outside region of $P_m$; \\
\\
\> \> For each {\em original} edge, $e \in P_m^{in}$ \\
\> \> begin \\
\> \> \> Push $e$ onto $InsideStack$; \\
\> \> end \\
\> \> For each {\em original} edge, $e \in P_m^{out}$ \\
\> \> begin \\
\> \> \> Push $e$ onto $OutSideStack$; \\
\> \> end \\
\> end \\
\\
\> While $InsideStack$ not empty\\
\> begin \\
\> \> $e$ $\Leftarrow$ edge poped from top of $InsideStack$; \\
\> \> $P_m$, $P_l$ $\Leftarrow$ two polygons sharing $e$ in ${\cal M}_1$; \\
\\
\> \> If ( $P_m$ has empty $OpenList( P_m )$ and empty $ClosedList( P_m )$, \\
\> \> \mbox{\hspace{3in}} and $P_m$ is not classified yet ) \\
\> \> begin \\
\> \> \> Classify $P_m$ as inside; \\
\> \> \> push all edges of $P_m$, but $e$, onto $InsideStack$; \\
\> \> end \\
\> \> If ( $P_l$ has empty $OpenList( P_l )$ and empty $ClosedList( P_l )$, \\
\> \> \mbox{\hspace{3in}} and $P_l$ is not classified yet ) \\
\> \> begin \\
\> \> \> Classify $P_l$ as inside; \\
\> \> \> push all edges of $P_l$, but $e$, onto $InsideStack$; \\
\> \> end \\
\> end \\
\\
\> While $OutsideStack$ not empty\\
\> begin \\
\> \> $e$ $\Leftarrow$ edge poped from top of $outsideStack$; \\
\> \> $P_m$, $P_l$ $\Leftarrow$ two polygons sharing $e$ in ${\cal M}_1$; \\
\\
\> \> If ( $P_m$ has empty $OpenList( P_m )$ and empty $ClosedList( P_m )$, \\
\> \> \mbox{\hspace{3in}} and $P_m$ is not classified yet ) \\
\> \> begin \\
\> \> \> Classify $P_m$ as outside; \\
\> \> \> push all edges of $P_m$, but $e$, onto $OutsideStack$; \\
\> \> end \\
\> \> If ( $P_l$ has empty $OpenList( P_l )$ and empty $ClosedList( P_l )$, \\
\> \> \mbox{\hspace{3in}} and $P_l$ is not classified yet ) \\
\> \> begin \\
\> \> \> Classify $P_l$ as outside; \\
\> \> \> push all edges of $P_l$, but $e$, onto $OutsideStack$; \\
\> \> end \\
\> end \\
\end{tabbing}
\end{tt}
\end{bf}

At the end of this process, $M_1$ is completely split into two sets of
polygons, classified as either inside $M_2$ or outside $M_2$.  An
identical spliting process can be applied to ${\cal M}_2$.  The proper
inside/outside lists of ${\cal M}_1$ can now be combined with the
proper list of classified polygons of ${\cal M}_2$ to yield the proper
result of the requested Boolean operation.

While this completes the Boolean operation, vast room for improvements
can be found, with some improvements implemented in IRIT.  For
example, the first step of the intersection computation necessitates
an $O(n^2)$ polygon--polygon intersection tests, where $n$ is in the
order of the number of polygons in ${\cal M}_i$.  Clearly, one can
hope for a better time complexity.  Bounding boxes, hierarchical
bounding boxes, or three dimensional space sweep techniques can all be
applied to reduce the overhead invested in these $O(n^2)$ tests.
Efficient sorting of line segments within a polygon, when forming the
open and closed loops is another optimization consideration.

\section{Library Functions}
\input{prog_man/bool_lib.tex}

%%%%%%%%%%%%%%%%%%%%%%%%%%%%%%%%%%%%%%%%%%%%%%%%%%%%%%%%%%%%%%%%%%%%%%%%%%%%%
\chapter{CAGD Library, cagd\_lib}

\section{General Information}

This library provides a rich set of function to create, convert,
display and process freeform Bezier and NURBs curves and surfaces. The
interface of the library is defined in {\em include/cagd\_lib.h}. This
library mainly supports low level freeform curve and surface
operations.  Supported are curves and surfaces from scalars to five
dimensions as E1/P1 to E5/P5 using the {\bf CagdPointType}. Pi is a
rational (projective) version of Ei, with an additional W coefficient.
Polynomial in the power basis have some very limited support as well.
Different data structures to hold UV parameter values, control points,
vectors, planes, bounding boxes, polylines and polygons are defined as
well as the data strcutures to hold the curves and surfaces
themselves,
\begin{verbatim}
typedef struct CagdCrvStruct {
    struct CagdCrvStruct *Pnext;
    struct IPAttributeStruct *Attr;
    CagdGeomType GType;
    CagdPointType PType;
    int Length;            /* Number of control points (== order in Bezier). */
    int Order;	    /* Order of curve (only for Bspline, ignored in Bezier). */
    CagdBType Periodic;			   /* Valid only for Bspline curves. */
    CagdRType *Points[CAGD_MAX_PT_SIZE];     /* Pointer on each axis vector. */
    CagdRType *KnotVector;
} CagdCrvStruct;

typedef struct CagdSrfStruct {
    struct CagdSrfStruct *Pnext;
    struct IPAttributeStruct *Attr;
    CagdGeomType GType;
    CagdPointType PType;
    int ULength, VLength;	 /* Mesh size in the tensor product surface. */
    int UOrder, VOrder;   /* Order in tensor product surface (Bspline only). */
    CagdBType UPeriodic, VPeriodic;      /* Valid only for Bspline surfaces. */
    CagdRType *Points[CAGD_MAX_PT_SIZE];     /* Pointer on each axis vector. */
    CagdRType *UKnotVector, *VKnotVector;
} CagdSrfStruct;
\end{verbatim}

Curves and surfaces have a geometric type {\bf GType} to prescribe the
type of entity (such as CAGD\_SBEZIER\_TYPE for Bezier surface) and a
point type {\bf PType} to prescribe the point type of the entity (such
as CAGD\_PT\_E3\_TYPE for three dimensional Euclidean control points).
{\bf Length} and {\bf Order} slots are used to hold the number of
control points in the mesh and or control polygon and the order(s) of
the basis functions. {\bf Periodic} flag(s) are used to denote
periodic end conditions. In addition, {\bf KnotVector} slot(s) are
used if the entity exploits Bspline basis functions, or NULL
otherwise.

The control polygon and/or mesh itself is organized in the {\bf
Points} slot as a vector of size {\bf CAGD\_MAX\_PT\_SIZE} of vectors
of {\bf CagdRType}s. For surfaces, the mesh is ordered U first and the
macros of {\bf CAGD\_NEXT\_U} {\bf CAGD\_NEXT\_V}, and {\bf
CAGD\_MESH\_UV} can be used to determine indices in the mesh.

All structures in the cagd library can be allocated using New
constrcutures (i.e. {\bf CagdUVNew} or {\bf CagdCrfNew}, freed using
Free destructores (i.e. {\bf CagdSrfFree} or {CagdBBoxFree}, linked
list free using FreeList destructores (i.e. {\bf
CagdPolylineFreeList}), and copied using copy constructores {i.e. {\bf
CagdPtCopy} or {\bf CagdCtlPtCopyList}).

This library has its own error handler, which by default prints an
error message and exit the program called {\bf CagdFatalError}.

Most globals in this library have a prefix of {\bf Cagd} for general
cagd routines. Prefix of {\bf Bzr} is used for Bezier routines, prefix
of {\bf Bsp} for Bspline specific routines, prefix of {\bf Cnvrt} for
conversion routines, and {\bf Afd} for adaptive forward differencing
routines.

\section{Library Functions}
\input{prog_man/cagd_lib.tex}

%%%%%%%%%%%%%%%%%%%%%%%%%%%%%%%%%%%%%%%%%%%%%%%%%%%%%%%%%%%%%%%%%%%%%%%%%%%%%
\chapter{Geometry Library, geom\_lib}

\section{General Information}

This library handles general computational geometry algorithms and
geometric queries such as a distance between two lines, bounding
boxes, convexity and convex hull of polygons, polygonal constructors
of primitives (cylinders, spheres, etc.), basic scan conversion
routines, etc.

\section{Library Functions}
\input{prog_man/geom_lib.tex}

%%%%%%%%%%%%%%%%%%%%%%%%%%%%%%%%%%%%%%%%%%%%%%%%%%%%%%%%%%%%%%%%%%%%%%%%%%%%%
\chapter{Graphics Library, grap\_lib}

\section{General Information}

This library handles general drawing and display algorithms, including
tesselation of all geometric objects, such as curves and surfaces, into
displayable primitives, i.e. polygons and polylines.

\section{Library Functions}
\input{prog_man/grap_lib.tex}

%%%%%%%%%%%%%%%%%%%%%%%%%%%%%%%%%%%%%%%%%%%%%%%%%%%%%%%%%%%%%%%%%%%%%%%%%%%%%
\chapter{Model Library, mdl\_lib}

\section{General Information}

This library provides the necessary tools to represent and process
models.  Models are sets of trimmed surfaces forming a closed
2-manifold shell.  Models are typically the result of Boolean operations
over freeform (trimmed) NURBs geometry but can also be constructed
directly or via other schemes.

\section{Library Functions}
\input{prog_man/mdl_lib.tex}

%%%%%%%%%%%%%%%%%%%%%%%%%%%%%%%%%%%%%%%%%%%%%%%%%%%%%%%%%%%%%%%%%%%%%%%%%%%%%
\chapter{Miscelleneous Library, misc\_lib}

\section{General Information}

This library holds general miscelleneous functions such as reading
configuration files, low level homogeneous matrices computation, low
level attributes and general machine specific low level routines.
Several header files can be found for this library:
\begin{center}
\begin{tabular}{||c|c||} \hline
    Header (include/*.h) & Functionality \\ \hline
    config.h   & Manipulation of configuration files (*.cfg files) \\
    dist\_pts.h & Enegry based distribution of points. \\
    gen\_mat.h  & Homogeneous matrices manipulation \\
    getarg.h   & Command line parsing for application tools \\
    imalloc.h  & Low level dynamic memory functions for IRIT \\
    miscattr.h & Low level attribute related functions \\
    priorque.h & An implementation of a priority queue \\
    xgeneral.h & Low level, machine specific, routines \\ \hline
\end{tabular}
\end{center}

\section{Library Functions}
\input{prog_man/misc_lib.tex}

%%%%%%%%%%%%%%%%%%%%%%%%%%%%%%%%%%%%%%%%%%%%%%%%%%%%%%%%%%%%%%%%%%%%%%%%%%%%%
\chapter{Multi variate functions, mvar\_lib}

\section{General Information}

This library holds functions to handle functions of arbitrary number
of variables. In this context curves (univariate), surfaces
(bivariate) and trivariates are special cases. This library provides a
rich set of functions to manipulate freeform Bezier and/or NURBs
multivariates. This library heavily depends on the cagd an symb
libraries. Functions are provided to create, copy, and destruct
multivariates, to extract isoparametric lower degree multivariates, to
evaluate, refine and subdivide, to read and write multivariates, to
differentiate, degree raise, make compatible and convert back and
forth to/from curves, surfaces, and trivariates.

A multivariate has m orders, m Length prescriptions and,
possibly, m knot vectors (if Bspline).  In addition it contains
an m dimensional volume of control points,
\begin{verbatim}
typedef struct MvarMVStruct {
    struct MvarMVStruct *Pnext;
    struct IPAttributeStruct *Attr;
    MvarGeomType GType;
    CagdPointType PType;
    int Dim;		      /* Number of dimensions in this multi variate. */
    int *Lengths;               /* Dimensions of mesh size in multi-variate. */
    int *SubSpaces;	   /* SubSpaces[i] = Prod(i = 0, i-1) of Lengths[i]. */
    int *Orders;                /* Orders of multi varariate (Bspline only). */
    CagdBType *Periodic;            /* Periodicity - valid only for Bspline. */
    CagdRType *Points[CAGD_MAX_PT_SIZE];     /* Pointer on each axis vector. */
    CagdRType **KnotVectors;
} MvarMVStruct;
\end{verbatim}

The interface of the library is defined in {\em include/mvar\_lib.h}. 

This library has its own error handler, which by default prints an
error message and exit the program called {\bf MvarFatalError}.

All globals in this library have a prefix of {\bf Mvar}.

\section{Library Functions}
\input{prog_man/mvar_lib.tex}

%%%%%%%%%%%%%%%%%%%%%%%%%%%%%%%%%%%%%%%%%%%%%%%%%%%%%%%%%%%%%%%%%%%%%%%%%%%%%
\chapter{Prsr Library, prsr\_lib}

\section{General Information}

This library provides the data file interface for IRIT. Functions are
provided to read and write data files, both compressed (on unix only,
using {\em compress}), and uncompressed, in binary and/or ascii text
modes. This library is also used to exchange data between the IRIT
server and the display devices' clients.  Several header files can be
found for this library:
\begin{center}
\begin{tabular}{||c|c||} \hline
    Header (include/*.h) & Functionality \\ \hline
    allocate.h & High level dynamic allocation of objects \\
    attribut.h & High level attributes for objects \\
    ip\_cnvrt.h & Freeform to polygon and polyline high level conversion \\
    iritprsr.h & Main interface to reading and writing data \\
    irit\_soc.h & Socket communication for data exchange \\ \hline
\end{tabular}
\end{center}

\section{Library Functions}
\input{prog_man/prsr_lib.tex}

%%%%%%%%%%%%%%%%%%%%%%%%%%%%%%%%%%%%%%%%%%%%%%%%%%%%%%%%%%%%%%%%%%%%%%%%%%%%%
\chapter{Rendering Library, rndr\_lib}

\section{General Information}

This library provides a powerful full screen scan conversion Z-buffer
tool to process IRIT geometry and convert it into images.  This
library allows one to scan convert any IRIT geometry including
polylines and curves that are converted to skinny polygons on the fly.
The library offers regular scan conversion with flat, Gouraud, and
Phong shading and several antialiasing filters along with advanced
features such as transparency and animation support, and width depth
cueing on polyline/curves rendering.  The library also provide direct
access to the depth Z-buffer as well as a stencil buffer.

\section{Library Functions}
\input{prog_man/rndr_lib.tex}

%%%%%%%%%%%%%%%%%%%%%%%%%%%%%%%%%%%%%%%%%%%%%%%%%%%%%%%%%%%%%%%%%%%%%%%%%%%%%
\chapter{Symbolic Library, symb\_lib}

\section{General Information}

This library provides a rich set of functions to symbolically manipulate
freeform curves and surfaces. This library heavily depends on the cagd
library. Functions are provided to low level add, subtract, and multiply
freeform curves and surfaces, to compute fields such as curvature,
and to extract singular points such as extremums, zeros, and inflections.
High level tools to metamorph curves and surfaces, to compute layout (prisa)
of freeform surfaces, to compute offset approximations of curves and
surfaces, and to compose curves and surfaces are also provided.

The interface of the library is defined in {\em include/symb\_lib.h}. 

This library has its own error handler, which by default prints an
error message and exit the program called {\bf SymbFatalError}.

Globals in this library have a prefix of {\bf Symb} for general
symbolic routines. Prefix of {\bf Bzr} is used for Bezier routines,
and prefix of {\bf Bsp} for Bspline specific routines.

\section{Library Functions}
\input{prog_man/symb_lib.tex}

%%%%%%%%%%%%%%%%%%%%%%%%%%%%%%%%%%%%%%%%%%%%%%%%%%%%%%%%%%%%%%%%%%%%%%%%%%%%%
\chapter{Trimmed surfaces Library, trim\_lib}

\section{General Information}

This library provides a set of functions to manipulate freeform
trimmed Bezier and/or NURBs surfaces. This library heavily depends on
the cagd library. Functions are provided to create, copy, and destruct
trimmed surfaces to extract isoparametric curves, to evaluate, refine
and subdivide, to read and write trimmed surfaces, degree raise, and
approximate using polygonal representations.  A trimming surface is
defined out of a tensor product surface and a set of trimming loops
that trims out portions of the parametric space of the surface,
\begin{verbatim}
typedef struct TrimSrfStruct {
    struct TrimSrfStruct *Pnext;
    IPAttributeStruct *Attr;
    int Tags;
    CagdSrfStruct *Srf;			  /* Surface trimmed by TrimCrvList. */
    TrimCrvStruct *TrimCrvList;		         /* List of trimming curves. */
} TrimSrfStruct;
\end{verbatim}

Each trimming loop consists of a set of trimming curve segments:
\begin{verbatim}
typedef struct TrimCrvStruct {
    struct TrimCrvStruct *Pnext;
    IPAttributeStruct *Attr;
    TrimCrvSegStruct *TrimCrvSegList;    /* List of trimming curve segments. */
} TrimCrvStruct;
\end{verbatim}

Each trimming curve segment contains a representation for the curve in the
UV space of the surface as well as a representation in the Euclidean space,
\begin{verbatim}
typedef struct TrimCrvSegStruct {
    struct TrimCrvSegStruct *Pnext;
    IPAttributeStruct *Attr;
    CagdCrvStruct *UVCrv;    /* Trimming crv segment in srf's param. domain. */
    CagdCrvStruct *EucCrv;       /* Trimming curve as an E3 Euclidean curve. */
} TrimCrvSegStruct;
\end{verbatim}


The interface of the library is defined in {\em include/trim\_lib.h}. 

This library has its own error handler, which by default prints an
error message and exit the program called {\bf TrimFatalError}.

All globals in this library have a prefix of {\bf Trim}.

\section{Library Functions}
\input{prog_man/trim_lib.tex}

%%%%%%%%%%%%%%%%%%%%%%%%%%%%%%%%%%%%%%%%%%%%%%%%%%%%%%%%%%%%%%%%%%%%%%%%%%%%%
\chapter{Trivariate Library, triv\_lib}

\section{General Information}

This library provides a rich set of functions to manipulate freeform
Bezier and/or NURBs trivariate. This library heavily depends on the cagd
library. Functions are provided to create, copy, and destruct trivariates,
to extract isoparametric surfaces, to evaluate, refine and subdivide, to
read and write trivariates, to differentiate, degree raise, make compatible
and approximate iso-surface at iso values using polygonal representations.

A trivariate has three orders, three Length prescriptions and,
possibly, three knot vectors (if Bspline).  In addition it contains
a three dimensional volume of control points,
\begin{verbatim}
typedef struct TrivTVStruct {
    struct TrivTVStruct *Pnext;
    struct IPAttributeStruct *Attr;
    TrivGeomType GType;
    CagdPointType PType;
    int ULength, VLength, WLength;/* Mesh size in tri-variate tensor product.*/
    int UVPlane;	  /* Should equal ULength * VLength for fast access. */
    int UOrder, VOrder, WOrder;       /* Order in trivariate (Bspline only). */
    CagdBType UPeriodic, VPeriodic, WPeriodic;    /* Valid only for Bspline. */
    CagdRType *Points[CAGD_MAX_PT_SIZE];     /* Pointer on each axis vector. */
    CagdRType *UKnotVector, *VKnotVector, *WKnotVector;
} TrivTVStruct;
\end{verbatim}

The interface of the library is defined in {\em include/triv\_lib.h}. 

This library has its own error handler, which by default prints an
error message and exit the program called {\bf TrivFatalError}.

All globals in this library have a prefix of {\bf Triv}.

\section{Library Functions}
\input{prog_man/triv_lib.tex}

%%%%%%%%%%%%%%%%%%%%%%%%%%%%%%%%%%%%%%%%%%%%%%%%%%%%%%%%%%%%%%%%%%%%%%%%%%%%%
\chapter{Triangular Library, trng\_lib}

\section{General Information}

This library provides a subset of functions to manipulate freeform
triangular Bezier and Bspline patches. This library heavily depends on
the cagd library. Functions are provided to create, copy, and destruct
triangular patches, to extract isoparametric curves, to evaluate,
refine and subdivide, to read and write triangular patches, to
differentiate, and approximate using polygonal representations.

A triangular patch has one prescription of Length and one prescription
of Order, the total Order and the length of an edge of the triangle.
The control mesh mesh has Length * (Length + 1) / 2 control points,
\begin{verbatim}
typedef struct TrngTriangSrfStruct {
    struct TrngTriangSrfStruct *Pnext;
    struct IPAttributeStruct *Attr;
    TrngGeomType GType;
    CagdPointType PType;
    int Length;		    /* Mesh size (length of edge of triangular mesh. */
    int Order;		      /* Order of triangular surface (Bspline only). */
    CagdRType *Points[CAGD_MAX_PT_SIZE];     /* Pointer on each axis vector. */
    CagdRType *KnotVector;
} TrngTriangSrfStruct;
\end{verbatim}

The interface of the library is defined in {\em include/trng\_lib.h}. 

This library has its own error handler, which by default prints an
error message and exit the program called {\bf TrngFatalError}.

All globals in this library have a prefix of {\bf Trng}.

\section{Library Functions}
\input{prog_man/trng_lib.tex}

%%%%%%%%%%%%%%%%%%%%%%%%%%%%%%%%%%%%%%%%%%%%%%%%%%%%%%%%%%%%%%%%%%%%%%%%%%%%%
\chapter{User Library, user\_lib}

\section{General Information}

This library includes user interface related geometrical functions such
as ray surface intersection (for mouse click/select operations), etc.

The interface of the library is defined in {\em include/user\_lib.h}. 

\section{Library Functions}
\input{prog_man/user_lib.tex}

%%%%%%%%%%%%%%%%%%%%%%%%%%%%%%%%%%%%%%%%%%%%%%%%%%%%%%%%%%%%%%%%%%%%%%%%%%%%%
\chapter{Extra Library, xtra\_lib}

\section{General Information}

This library is not an official part of IRIT and contains public
domain code that is used by routines in IRIT.

The interface of the library is defined in {\em include/extra\_fn.h}. 

\section{Library Functions}
\input{prog_man/xtra_lib.tex}

%%%%%%%%%%%%%%%%%%%%%%%%%%%%%%%%%%%%%%%%%%%%%%%%%%%%%%%%%%%%%%%%%%%%%%%%%%%%%
\chapter{Programming Examples}
\label{chap-examples}

This chapter describes several simple examples of C programs that
exploits the libraries of IRIT. All external function are defined in
the {\em include} subdirectory of IRIT and one can 'grep' there for the
exact include file that contains a certain function name. All C programs
in all C files should include 'irit\_sm.h' as their first include file
of IRIT, before any other include file of IRIT. Header files are set so
C++ code can compile and link to it without any special treatment.

\section{Setting up the Compilation Environment}

In order to compile programs that uses the libraries of IRIT, a makefile
has to be constructed. Assuming IRIT is installed in {\em /usr/local/irit},
here is a simple makefile that can be used (for a unix environment):

\begin{verbatim}
IRIT_DIR = /usr/local/irit

include $(IRIT_DIR)/makeflag.unx

OBJS	= program.o

program: $(OBJS)
	$(CC) $(CFLAGS) -o program $(OBJS) $(LIBS) -lm $(MORELIBS)
\end{verbatim}

The simplicity of this makefile is drawn from the complexity of
makeflag.unx. The file makeflag.unx sets the CC, CFLAGS, LIBS, and
MORELIBS for the machined using among other things. Furthermore,
makeflag.unx also sets the default compilation rules from C sources to
object files.  The file makeflag.unx had to be modified once, when
IRIT was installed on this system. If the used system is not a unix
environment, then the file makefile.unx will have to be replaced with
the proper makeflag file.  In an OS2 environment, using the emx gcc
compiler, the makefile is even simpler since the linking rule is also
defined in makeflag.os2:

\begin{verbatim}
IRIT_DIR = \usr\local\irit

include $(IRIT_DIR)\makeflag.os2

OBJS	= program.o

program.exe: $(OBJS)
\end{verbatim}

Finally, here is a proper makefile for Windows NT:

\begin{verbatim}
IRIT_DIR = \usr\local\irit

include $(IRIT_DIR)\makeflag.wnt

OBJS	= program.obj

program.exe: $(OBJS)
        $(IRITCONLINK) -out:program.exe $(OBJS) $(LIBS) $(W32CONLIBS)
\end{verbatim}

\section{Simple C Programs using IRIT}

Now that we have an idea how to compile C code using IRIT, here are several
examples to read, manipulate and write IRIT data files. You will be able to
find all these examples in the doc/cexample directory.

\subsection{Boolean Operations over Polyhedra (boolean.c)}

This example demonstates the use of the Boolean operations package
(bool\_lib), creates few polygonal primitives and apply Boolean
operations between them.  The resulting model is dumped to stdout.

\input{cexample/boolean.tex}

\subsection{Distance Maps to Planar Curves (dist\_map.c)}

This example demonstates the use of symbolic distance maps
computations to freeform curves, in the plane, and the sampling
of the map into an image that is dumped out.

\input{cexample/dist_map.tex}

\subsection{Importance Edges Evaluations in Polygonal Meshes (imprtnc.c)}

This example demonstates the tesselation of arbitrary input model(s)
into polygons and the probabalistic estimation of importance edges in the
polygonal geometry.  Dumped out is the polygonal geometry with vertices
having UV coordinates if available, normals if available, and importance.

\begin{verbatim}
/*****************************************************************************
*   Computes the importance of vertices and dump as vertex list + polygons   *
******************************************************************************
* (C) Gershon Elber, Technion, Israel Institute of Technology                *
******************************************************************************
* Written by:  Gershon Elber                                Ver 1.0, June 1995   *
*****************************************************************************/

#include "irit_sm.h"
#include "allocate.h"
#include "iritprsr.h"
#include "geom_lib.h"
#include "misc_lib.h"
#include "ip_cnvrt.h"

IRIT_STATIC_DATA char
    *CtrlStr = "Imprtnc c%- h%- DFiles!*s";
IRIT_STATIC_DATA int GlblVrtxImportanceCount,
    GlblCrvtrInfoFlag = FALSE;
IRIT_STATIC_DATA IrtRType GlblVrtxImportanceVal;
IRIT_STATIC_DATA IPVertexStruct *GlblVrtxImportance;

static void DumpOneTraversedObject(IPObjectStruct *PObj, IrtHmgnMatType Mat);
static IPObjectStruct *SetCurvatureEstimates(IPObjectStruct *PObj);
static void DumpOneObjData(IPObjectStruct *PObj);
static void ProcessVertexImportance(IPVertexStruct *V1,
                                    IPVertexStruct *V2,
                                    IPPolygonStruct *Pl1,
                                    IPPolygonStruct *Pl2);
static void GenPolyImportance(IPObjectStruct *PObj);

void main(int argc, char **argv)
{
    int NumFiles, Error,
        HelpFlag = FALSE;
    char **FileNames;
    IPObjectStruct *PObjects;
    IrtHmgnMatType CrntViewMat;

    if ((Error = GAGetArgs(argc, argv, CtrlStr, &GlblCrvtrInfoFlag,
                           &HelpFlag, &NumFiles, &FileNames)) != 0) {
        GAPrintErrMsg(Error);
        GAPrintHowTo(CtrlStr);
        exit(1);
    }

    if (HelpFlag) {
        fprintf(stderr, "This is Importance testing...\n");
        GAPrintHowTo(CtrlStr);
        exit(0);
    }

    /* Get the data files: */
    IPSetFlattenObjects(FALSE);
    if ((PObjects = IPGetDataFiles(FileNames, NumFiles, TRUE, FALSE)) == NULL)
        exit(1);
    PObjects = IPResolveInstances(PObjects);

    if (IPWasPrspMat)
        MatMultTwo4by4(CrntViewMat, IPViewMat, IPPrspMat);
    else
        IRIT_GEN_COPY(CrntViewMat, IPViewMat, sizeof(IrtHmgnMatType));

    /* Here some useful parameters to play with in tesselating freeforms: */
    IPFFCState.FineNess = 15; /* Resolution of tesselation, larger is finer. */
    IPFFCState.ComputeUV = TRUE;       /* Wants UV coordinates for textures. */
    IPFFCState.FourPerFlat = TRUE;/* 4 polygons per ~flat patch, 2 otherwise.*/
    IPFFCState.LinearOnePolyFlag = TRUE;   /* Linear srf generates one poly. */

    IPTraverseObjListHierarchy(PObjects, CrntViewMat, DumpOneTraversedObject);

    IPFreeObjectList(PObjects);
}

/*****************************************************************************
* DESCRIPTION:                                                               *
*   Update curvature property attribute to vertices of the given poly model. *
*                                                                            *
* PARAMETERS:                                                                *
*   PObj:   Poly model to udpate curvatrue info into.                             *
*                                                                            *
* RETURN VALUE:                                                              *
*   IPObjectStruct *:   A poly model, similar to PObj, but with curvature    *
*                        information attached as attributes.                  *
*****************************************************************************/
static IPObjectStruct *SetCurvatureEstimates(IPObjectStruct *PObj)
{
    int OldCirc = IPSetPolyListCirc(TRUE);
    IPPolygonStruct *Pl;
    IPObjectStruct *PTmp1, *PTmp2;

    /* Convert to a regular polygonal model with triangles only. */
    for (Pl = PObj -> U.Pl; Pl != NULL; Pl = Pl -> Pnext) {
        if (IPVrtxListLen(Pl -> PVertex) != 3)
            break;
    }
    PTmp2 = IPCopyObject(NULL, PObj, FALSE);
    GMVrtxListToCircOrLin(PTmp2 -> U.Pl, TRUE);
    if (Pl != NULL) {
        PTmp1 = GMConvertPolysToTriangles(PTmp2);
        IPFreeObject(PTmp2);
        PTmp2 = GMRegularizePolyModel(PTmp1);
        IPFreeObject(PTmp1);
    }

    GMVrtxListToCircOrLin(PTmp2 -> U.Pl, FALSE);
    IPSetPolyListCirc(OldCirc);

    GMPlCrvtrSetCurvatureAttr(PTmp2 -> U.Pl, 1, TRUE);

    return PTmp2;
}

/*****************************************************************************
* DESCRIPTION:                                                               *
*   Call back function of IPTraverseObjListHierarchy. Called on every non    *
* list object found in hierarchy.                                            *
*                                                                            *
* PARAMETERS:                                                                *
*   PObj:       Non list object to handle.                                   *
*   Mat:        Transformation matrix to apply to this object.               *
*                                                                            *
* RETURN VALUE:                                                              *
*   void                                                                     *
*****************************************************************************/
static void DumpOneTraversedObject(IPObjectStruct *PObj, IrtHmgnMatType Mat)
{
    IPObjectStruct *PObjs;

    if (IP_IS_FFGEOM_OBJ(PObj))
        PObjs = IPConvertFreeForm(PObj, &IPFFCState);
    else
        PObjs = PObj;

    for (PObj = PObjs; PObj != NULL; PObj = PObj -> Pnext) {
        if (IP_IS_POLY_OBJ(PObj)) {
            if (GlblCrvtrInfoFlag) {
                IPObjectStruct
                    *PTri = SetCurvatureEstimates(PObj);

                DumpOneObjData(PTri);
                IPFreeObject(PTri);
            }
            else
                DumpOneObjData(PObj);
        }
    }
}

/*****************************************************************************
* DESCRIPTION:                                                               *
*   Convert the polygonal mesh into a vertex list with importance and the    *
* polygons as indices into this list.                                        *
*                                                                            *
* PARAMETERS:                                                                *
*   PObj:   A polygonal mesh to dump.                                        *
*                                                                            *
* RETURN VALUE:                                                              *
*   void                                                                     *
*****************************************************************************/
static void DumpOneObjData(IPObjectStruct *PObj)
{
    IPPolyVrtxIdxStruct
        *PVIdx = IPCnvPolyToPolyVrtxIdxStruct(PObj, TRUE, 0);
    int        i,
        **Polygons = PVIdx -> Polygons;
    IPVertexStruct
        **Vertices = PVIdx -> Vertices;

    /* Compute importance for the vertices as "Imprt" attributes.  Note the */
    /* PVIdx data structure points on the original vertices so we are fine. */
    GenPolyImportance(PObj);

    /* Dump the vertices: */
    fprintf(stderr, "OBJECT \"%s\" - VERTICES:\n", PObj -> ObjName);
    for (i = 0; Vertices[i] != NULL; i++) {
        IrtRType R;
        float *uv;

        printf("%3d:  %6.3f %6.3f %6.3f", i,
               Vertices[i] -> Coord[0], 
               Vertices[i] -> Coord[1], 
               Vertices[i] -> Coord[2]);
        if (IP_HAS_NORMAL_VRTX(Vertices[i]))
            printf(" [%6.3f %6.3f %6.3f]",
                   Vertices[i] -> Normal[0], 
                   Vertices[i] -> Normal[1], 
                   Vertices[i] -> Normal[2]);
        if ((uv = AttrGetUVAttrib(Vertices[i] -> Attr, "uvvals")) != NULL)
            printf(" {%6.3f %6.3f}", uv[0], uv[1]);

        R = AttrGetRealAttrib(Vertices[i] -> Attr, "Imprt");
        if (!IP_ATTR_IS_BAD_REAL(R))
            printf(" (%6.3f)", R);

        if (GlblCrvtrInfoFlag) {
            printf("\n\t<K1=%6.3f D1 = %s>\n\t<K2=%6.3f D2 = %s>",
                   AttrGetRealAttrib(Vertices[i] -> Attr, "K1Curv"),
                   AttrGetStrAttrib(Vertices[i] -> Attr, "D1"),
                   AttrGetRealAttrib(Vertices[i] -> Attr, "K2Curv"),
                   AttrGetStrAttrib(Vertices[i] -> Attr, "D2"));
        }

        printf("\n");
    }

    fprintf(stderr, "\nOBJECT \"%s\" - Vertices in polygons:\n",
            PObj -> ObjName);
    for (i = 0; PVIdx -> PPolys[i] != NULL; i++) {
        IPPolyPtrStruct
            *PPoly = PVIdx -> PPolys[i];

        printf("%3d: ", i);
        for ( ; PPoly != NULL; PPoly = PPoly -> Pnext) {
            printf("%3d ", AttrGetIntAttrib(PPoly -> Poly -> Attr, "_PIdx"));
        }
        printf("\n");
    }

    /* Dump the polygons: */
    fprintf(stderr, "\nOBJECT \"%s\" - Polygons from vertices:\n",
            PObj -> ObjName);
    for (i = 0; Polygons[i] != NULL; i++) {
        int *Pl = Polygons[i];

        printf("%3d: ", i);
        while (*Pl >= 0)
            fprintf(stderr, " %5d", *Pl++);

        fprintf(stderr, "    -1\n");
    }

    IPPolyVrtxIdxFree(PVIdx);
}

/*****************************************************************************
* DESCRIPTION:                                                               *
*   Call back function for GMPolyAdjacncyVertex to process every edge of a   *
* given vertex.  The edge is provided as (V, V -> Pnext)                     *
*                                                                            *
* PARAMETERS:                                                                *
*   V1, V2:    Two vertices defining this edge.  Note the vertices are NOT   *
*               necessarily chained together into a list.                     *
*   Pl1, Pl2:  The two polygons that share this edge.  The edge (V1, V2) is  *
*               in both Pl1 and Pl2, with not necessarily the exact pointers  *
*               IPVertexStruct of V1 and V2.                                     *
*                                                                            *
* RETURN VALUE:                                                              *
*   void                                                                     *
*****************************************************************************/
static void ProcessVertexImportance(IPVertexStruct *V1,
                                    IPVertexStruct *V2,
                                    IPPolygonStruct *Pl1,
                                    IPPolygonStruct *Pl2)
{
    if (!IRIT_PT_APX_EQ_EPS(V1 -> Coord, GlblVrtxImportance -> Coord,
                            IRIT_EPS) &&
        !IRIT_PT_APX_EQ_EPS(V2 -> Coord, GlblVrtxImportance -> Coord,
                            IRIT_EPS))
        fprintf(stderr, "Edge does not match the given vertex - adj error!\n");

    if (Pl1 != NULL && Pl2 != NULL) {
        GlblVrtxImportanceCount++;
        GlblVrtxImportanceVal += acos(IRIT_DOT_PROD(Pl1 -> Plane,
                                                    Pl2 -> Plane));
    }
}

/*****************************************************************************
* DESCRIPTION:                                                               *
*   Computes importance for each vertex in the given polygonal mesh.         *
*                                                                            *
* PARAMETERS:                                                                *
*   PObj:         Polygonal model to process for importance.                     *
*                                                                            *
* RETURN VALUE:                                                              *
*   void                                                                     *
*****************************************************************************/
static void GenPolyImportance(IPObjectStruct *PObj)
{
    VoidPtr
        PAdj = GMPolyAdjacncyGen(PObj, IRIT_EPS);
    IPPolygonStruct *Pl;

    for (Pl = PObj -> U.Pl; Pl != NULL; Pl = Pl -> Pnext) {
        int PlImportanceCount = 0;
        IrtRType
            PlImportanceVal = 0.0;
        IPVertexStruct *V;

        for (V = Pl -> PVertex; V != NULL; V = V -> Pnext) {
            GlblVrtxImportanceVal = 0.0;
            GlblVrtxImportanceCount = 0;
            GlblVrtxImportance = V;

            GMPolyAdjacncyVertex(V, PAdj, ProcessVertexImportance);

            if (GlblVrtxImportanceCount > 0) {
                GlblVrtxImportanceVal /= (GlblVrtxImportanceCount * M_PI);

                AttrSetRealAttrib(&V -> Attr, "Imprt", GlblVrtxImportanceVal);
            }
            else
                fprintf(stderr, "Failed to compute Importance for vertex\n");
        }
    }

    GMPolyAdjacncyFree(PAdj);
}
\end{verbatim}


\subsection{Least sqaures curve fitting and process communication (lst\_sqrs.c)}

This example creates a Bspline curve that least squares approximates
a given set of data points. In the default values, a quadratic Bspline
curve of 10 control points least sqaures approximates a set of 100
three dimensional points. Also shown is a polyline of the original
(100) points. Arbitrary number of curves will be displayed every, one
for every <return> keystroke until Q<return> is typed.

\begin{verbatim}
/*****************************************************************************
* Least squares fit a curve to random points.                                     *
* We also see how to fork out a display device and communicate with it.      *
******************************************************************************
* (C) Gershon Elber, Technion, Israel Institute of Technology                *
******************************************************************************
* Written by:  Gershon Elber                                Ver 1.0, June 1995   *
*****************************************************************************/

#include "irit_sm.h"
#include "iritprsr.h"
#include "allocate.h"
#include "attribut.h"
#include "cagd_lib.h"
#include "geom_lib.h"
#include "grap_lib.h"
#include "misc_lib.h"

static char *CtrlStr =
    "Lst_Sqrs n%-#Pts!d d%-Degree!d f%-DOF!d p%-PrgmName!s h%-";

void main(int argc, char **argv)
{
    int i, Error, PrgmIO,
        NumOfPoints = 100,
        NumOfPtsFlag = FALSE,
        Degree = 3,
        DegreeFlag = FALSE,
        NumOfDOF = 10,
        NumOfDOFFlag = FALSE,
        PrgmFlag = FALSE,
        HelpFlag = FALSE;
    char *Err,
        *Program = getenv("IRIT_DISPLAY");

#ifdef __WINNT__
    if (Program == NULL)
        Program = "wntgdrvs -s-";
#endif /* __WINNT__ */
#ifdef __UNIX__
    if (Program == NULL)
        Program = "x11drvs -s-";
#endif /* __UNIX__ */

    if ((Error = GAGetArgs(argc, argv, CtrlStr,
                           &NumOfPtsFlag, &NumOfPoints,
                           &DegreeFlag, &Degree,
                           &NumOfDOFFlag, &NumOfDOF,
                           &PrgmFlag, &Program,
                           &HelpFlag)) != 0) {
        GAPrintErrMsg(Error);
        GAPrintHowTo(CtrlStr);
        exit(1);
    }

    if (HelpFlag) {
        GAPrintHowTo(CtrlStr);
        exit(0);
    }

    IPSocSrvrInit();            /* Initialize the listen socket for clients. */

    if ((PrgmIO = IPSocExecAndConnect(Program,
                                      getenv("IRIT_BIN_IPC") != NULL)) >= 0) {
        char Line[IRIT_LINE_LEN];
        IPObjectStruct
            *PClrObj = IPGenStrObject("command_", "clear", NULL);

        do {
            CagdPtStruct
                *PtList = NULL;
            IPPolygonStruct
                *PPoly = IPAllocPolygon(0, NULL, NULL);
            CagdCrvStruct *Crv;
            IPObjectStruct *PCrvObj, *PPolyObj;

            for (i = 0; i < NumOfPoints; i++) {
                int j;
                IPVertexStruct *V;
                CagdPtStruct
                    *Pt = CagdPtNew();

                if (i == 0) {
                    for (j = 0; j < 3; j++)
                        Pt -> Pt[j] = IritRandom(-1.0, 1.0);
                }
                else {
                    for (j = 0; j < 3; j++)
                        Pt -> Pt[j] = PtList -> Pt[j] + IritRandom(-0.1, 0.1);
                }

                V = IPAllocVertex(0, NULL, PPoly -> PVertex);
                for (j = 0; j < 3; j++)
                    V -> Coord[j] = Pt -> Pt[j];
                PPoly -> PVertex = V;

                IRIT_LIST_PUSH(Pt, PtList);
            }

            Crv = BspCrvInterpPts(PtList, Degree + 1,
                                  NumOfDOF, CAGD_UNIFORM_PARAM, FALSE);
            CagdPtFreeList(PtList);

            CagdCrvWriteToFile3(Crv, stdout, 0, "This is from LstSqrs", &Err);

            /* Generate objects out of the geometry and set proper attrs. */
            PCrvObj = IPGenCRVObject(Crv);
            AttrSetObjectColor(PCrvObj, IG_IRIT_CYAN);

            PPolyObj = IPGenPOLYObject(PPoly);
            IP_SET_POLYLINE_OBJ(PPolyObj);
            AttrSetObjectColor(PPolyObj, IG_IRIT_YELLOW);

            /* Clear old data and display our curve and data. */
            IPSocWriteOneObject(PrgmIO, PClrObj);
            IPSocWriteOneObject(PrgmIO, PCrvObj);
            IPSocWriteOneObject(PrgmIO, PPolyObj);

            IPFreeObject(PCrvObj);
            IPFreeObject(PPolyObj);

            gets(Line);
        }
        while (Line[0] != 'q' && Line[0] != 'Q');

        IPSocDisConnectAndKill(TRUE, PrgmIO);
    }

    exit(0);
}
\end{verbatim}


\subsection{Multivariate Solver (msolve.c)}

This example demonstates the use of the multivariate solver from the
mvar\_lib library.  Input is a set of n polynomial equations in m
variables and dumped out are the solutions to these equations.

\input{cexample/msolve.tex}

\subsection{Compute Area of a Polygonal Model (polyarea.c)}

Here is a simple program to compute the total area of all polygons in
the given data file. The program expects one argument on the command
line which is the name of the file to read, and it prints out one
line with the total computed area.

\input{cexample/polyarea.tex}

\subsection{Converts a Freeform Surface into Polygons (polygons.c)}

This true filter reads a single surface from stdin and dumps out a
polygonal approximation of it to stdout. They are several parameters
that controls the way a surface is approximated into polygons and in
this simple filter they are being held fixed in a set of integer
variables.

\begin{verbatim}
/*****************************************************************************
* Read a surface and dump out a tesselated version of it.                     *
******************************************************************************
* (C) Gershon Elber, Technion, Israel Institute of Technology                *
******************************************************************************
* Written by:  Gershon Elber                                Ver 1.0, June 1995   *
*****************************************************************************/

#include "irit_sm.h"
#include "cagd_lib.h"
#include "iritprsr.h"
#include "ip_cnvrt.h"
#include "allocate.h"

void main(int argc, char **argv)
{
    int Handler,
        FourPerFlat = TRUE, /* Settable parameters of IritSurface2Polygons. */
        FineNess = 20,
        ComputeUV = FALSE,
        ComputeNrml = FALSE,
        Optimal = FALSE;

    if ((Handler = IPOpenDataFile("-", TRUE, TRUE)) >= 0) {
        IPObjectStruct
            *PObj = IPGetObjects(Handler);

        /* Done with file - close it. */
        IPCloseStream(Handler, TRUE);

        /* Process the surface into polygons. */
        if (IP_IS_SRF_OBJ(PObj)) {
            IPPolygonStruct
                *PPoly = IPSurface2Polygons(PObj -> U.Srfs, FourPerFlat,
                                            FineNess, ComputeUV,
                                            ComputeNrml, Optimal);
            IPObjectStruct
                *PObjPoly = IPGenPOLYObject(PPoly);

            IPStdoutObject(PObjPoly, FALSE);

            IPFreeObject(PObjPoly);
        }
        else
            fprintf(stderr, "Read object is not a surface.\n");

        IPFreeObject(PObj);
    }
    else {
        fprintf(stderr, "Failed to read from stdin\n");
        exit(1);
    }

    exit(0);
}
\end{verbatim}


\subsection{Linear Transformations' Filter (transfrm.c)}

This little more complex program transfroms all the geometry in the
read data which can be any number of files according to the specified
transformations on the command line. The command line is parsed via
{\bf GAGetArgs} and its associated functions. The transformation
matrix is then computed with the aid of the matrix package and applied
to the read geometry at once.

\input{cexample/transfrm.tex}

Here is the result of running 'transfrm -h':
\begin{verbatim}
This is Transform...
Usage: Transfrm [-x Degs] [-y Degs] [-z Degs] [-t X Y Z] [-s Scale] [-h] DFiles
\end{verbatim}

When you are considering the usefulness of this tool remember that the
transformations are applied to the geometry in an internal order which is
different from the command line order. That is,
\begin{verbatim}
     transfrm -x 30 -y 30 geometry.dat > tgeometry.dat
\end{verbatim}
will compute the exact same transfrom as,
\begin{verbatim}
     transfrm -y 30 -x 30 geometry.dat > tgeometry.dat
\end{verbatim}
This, while rotation is not a commutative operation.
Nonetheless, you may split the operations. That is:
\begin{verbatim}
     transfrm -y 30 geometry.dat | transfrm -x 30 - > tgeometry.dat
\end{verbatim}
or
\begin{verbatim}
     transfrm -x 30 geometry.dat | transfrm -y 30 - > tgeometry.dat
\end{verbatim}
will do exactly what one expects.

%%%%%%%%%%%%%%%%%%%%%%%%%%%%%%%%%%%%%%%%%%%%%%%%%%%%%%%%%%%%%%%%%%%%%%%%%%%%%
\twocolumn
% 
% prog_man.tex - The IRIT programmer's manual - main module.
% 
% Author:	Gershon Elber
% 		Computer Science Dept.
% 		Technion, IIT
%

%\documentstyle[10pt]{book}
\documentstyle{book}

\setlength{\oddsidemargin}{0.0in}
\setlength{\evensidemargin}{0.0in}
\setlength{\topmargin}{-0.2in}
\setlength{\textwidth}{6.5in}
\setlength{\textheight}{9.0in}

\makeindex

\begin{document}

\small
\setlength{\baselineskip}{0.85\baselineskip}

\tableofcontents

%%%%%%%%%%%%%%%%%%%%%%%%%%%%%%%%%%%%%%%%%%%%%%%%%%%%%%%%%%%%%%%%%%%%%%%%%%%%%
\chapter{Introduction}

This manual describes the different libraries of the IRIT solid
modeling environment. Quite a few libraries can be found to manipulate
geometry in general, freeform curves and surfaces, symbolic
computation, trimmed surfaces, triangular patches, freeform trivariate
functions, Boolean operations, input output data file parsing, and
miscelleneous.

All interface to the libraries should be made via the appropriate header
files that can be found in the include subdirectory. Most libraries
have a single header file that is named the same as the library.
Functions and constants that are visible to the users of the libraries
are prefixed with a unique prefix, usually derived from the library
name itself. External definitions that start with an underscore should
not be used, even if found in header files.

The header file {\bf include/irit\_sm.h} must be sourced by every
source file in the solid modeller. In most cases, this file is sourced
indirectly via local header files.

The following libraries are avaliable in IRIT:
\begin{center}
\begin{tabular}{|l|l|} \hline
    Name of Library & Tasks \\ \hline
    bool & Boolean operations on polygonal models. \\
    cagd & Low level freeform curves and surfaces. \\
    geom & General geometry functions. \\
    grap & General graphics/display functions. \\
    mdl  & Model's processing functions. \\
    misc & Memory allocation, configuration files, attributes, etc. \\
    mvar & Multi variate functions. \\
    prsr & Input and output for file/sockets of objects of IRIT. \\
    rndr & Scan conversion rendering functions. \\
    symb & Symbolic manipulation of curves and surfaces. \\
    trim & Trimmed surfaces support. \\
    triv & Freeform trivariate functions. \\
    trng & Triangular patches support. \\
    user & General high level user inteface functions. \\
    xtra & Public domain code that is not part of IRIT. \\ \hline
\end{tabular}
\end{center}

%%%%%%%%%%%%%%%%%%%%%%%%%%%%%%%%%%%%%%%%%%%%%%%%%%%%%%%%%%%%%%%%%%%%%%%%%%%%%

\subsection{Tools and Programs}

The IRIT package includes several complete programs such as poly3d-h
(hidden line removal), irender (scan conversion tool), and irit2ps (a
filter to Postscript).  Somewhat different than most other programs is
the kernel interpreter, also called irit.  The irit program is nothing
more than an interpreter (written in C) that invokes numerous
functions in the several libraries provided.  In order to add a new
function to the irit interpreter, the following sequence of operations
must be followed:

\begin{enumerate}
\item Write a C function that accepts only one or more of the following
     type of parameters.  All parameters, including the IrtRType, {\em must}
     be transferred by address:
    \begin{itemize}
	\item IrtRType         (see irit\_sm.h).
	\item IrtVecType       (see irit\_sm.h).
	\item IrtPtType        (see irit\_sm.h).
	\item CagdCtlPtStruct  (see cagd\_lib.h).
	\item IrtPlnType       (see irit\_sm.h).
	\item StringType       (char *).
	\item IPObjectStruct   (see iritprsr.h).  This includes all object
		types in IPObjectStruct other than the above.
    \end{itemize}
\item The written C function can return one of:
    \begin{itemize}
	\item IrtRType by value.
	\item IPObjectStruct by address.
	\item Nothing (a procedure).
    \end{itemize}
\item According to the returned type by the new C function, go to file
	{\bf inptprsl.h}, and add a new enum NEW\_FUNC for the new function in
	enum RealValueFuncType (for IrtRType returned value), in
	enum ObjValueFuncType (for IPObjectStruct * returned value), or in
	enum GenValueFuncType (for no returned value).
\item In {\bf inptevl1.c}, add one new line in one of NumFuncTable,
	ObjFuncTable, or GenFuncTable (depends upon the return value).
	The line will have the form of:
	\begin{verbatim}
            { ``FuncName'', NEW_FUNC, CFunctionName,  N, { Param1Type,
                                              Param2Type, ... ParamNType } }
	\end{verbatim}
	for procedures with no return values, and of the form of:
	\begin{verbatim}
            { ``FuncName'', NEW_FUNC, CFunctionName,  N, { Param1Type,
                                    Param2Type, ... ParamNType }, RetValType }
	\end{verbatim}
	otherwise. \verb-N- is the number of parameters,
	\verb-`FuncName''- is the (unique) name that will be used in
	the interpreter, CFunctionName is the name of the C function you
	have written. This C function must be declared in one of the header
	files that inptevl1.c includes. ParamIType, for I between 1 and N,
	and RetValType are of type IritExprType (see inptprsl.h).
\item Thats it!
\end{enumerate}

For example, to add the C function declared as
\begin{verbatim}
    IPObjectStruct *PolyhedraMoments(IPObjectStruct *IPObjectStruct,
                                     IrtRType *m);
\end{verbatim}
one has to add the following line to ObjFuncTable in iritevl1.c
\begin{verbatim}
    { ``PMOMENT'', PMOMENT, { POLY_EXPR, REAL_EXPR }, VECTOR_EXPR },
\end{verbatim}
where PMOMENT needs to be added to ObjValueFuncType in iritprsl.h.

While all objects in the interpreted space are of type IPObjectStruct,
the functions' interface unfolds many types to simplify matter.
IrtRType, IrtVecType, Strings, etc. are all extracted from the given
(IPObjectStruct) parameters and passed directly by address.  For
returned values, only numeric real data is allowed where everything
else must be returned wrapped in an IPObjectStruct.

%%%%%%%%%%%%%%%%%%%%%%%%%%%%%%%%%%%%%%%%%%%%%%%%%%%%%%%%%%%%%%%%%%%%%%%%%%%%%

The following chapters reference the different functions of the different
libraries. Chapter~\ref{chap-examples} provides several examples of
writing C code using the IRIT libraries.

%%%%%%%%%%%%%%%%%%%%%%%%%%%%%%%%%%%%%%%%%%%%%%%%%%%%%%%%%%%%%%%%%%%%%%%%%%%%%
\chapter{Boolean Library, bool\_lib}

\section{General Information}

On of the crucial operation in any solid modeling environment is the
ability to perform Boolean operations among different geometric
objects.  The interface of the library is defined in {\em
include/bool\_lib.h}. This library supports only Boolean operations of
polygonal objects. The Boolean operations of OR, AND, SUBtract,
NEGate, CUT, and MERGE are supported via the {\bf BoolOperType}
typedef:
\begin{center}
\begin{tabular}{||c|c||} \hline 
    BoolOperType & Meaning \\ \hline
    BOOL\_OPER\_OR & Union of two geometrical objects \\
    BOOL\_OPER\_AND & Intersection of two geometrical objects \\
    BOOL\_OPER\_SUB & The difference of two geometrical objects \\
    BOOL\_OPER\_NEG & Unary Inside-out of one geometrical object \\
    BOOL\_OPER\_CUT & Boundary of one object outside the other \\
    BOOL\_OPER\_MERGE & Simple merge without any computation \\ \hline
\end{tabular}
\end{center}

The {\bf BoolOperType} typedef is used in two dimensional Boolean operations
invoked via {\bf Boolean2D}. Three dimensional Boolean operations are
invoked via {\bf BooleanXXX} where XXX is one of OR, AND, SUB, NEG, CUT
or MERGE, denoting the same as in the table above.

In addition several state functions are available to control the way
the Boolean operations are conducted. Functions to enable the dump of
(only) the intersection curves, to handle coplanar polygons, and to
set the axis along which the sweep is performed are provided.

All globals in this library have a profix of {\bf Bool}.

%%%%%%%%%%%%%%%%%%%%%%%%%%%%%%%%%

\section{Algorithmic Hints Behind the Boolean Operations}

Let ${\cal M}_i,~i = 1,2$ be two polygonal models to compute a Boolean
operation between and let $P_m$ denote the $m$'th polygon of ${\cal M}_i$.
Denote by $InterList( P_m ),~P_m \in {\cal M}_i$, the list of line
segments representing the intersections of polygon $P_m$ with polygons
in ${\cal M}_j$.  A naive first, intersection, step in a Boolean
operation between two polyhedra models, ${\cal M}_1$ and ${\cal M}_2$,
will consist of the following stages:
\begin{bf}
\begin{tt}
\begin{tabbing}
\hspace{2em}\=\hspace{2em}\=\hspace{2em}\=\hspace{2em}\=\hspace{2em}\=\hspace{2em}\=\kill
\> For each polygon $P_m$ in ${\cal M}_1$ \\
\> begin \\
\> \> $InterList( P_m )$ $\Leftarrow$ $\emptyset$ \\
\> end \\
\> For each polygon $P_n$ in ${\cal M}_2$ \\
\> begin \\
\> \> $InterList( P_n )$ $\Leftarrow$ $\emptyset$ \\
\> end \\
\\
\> For each polygon $P_m$ in ${\cal M}_1$ \\
\> begin \\
\> \> For each polygon $P_n$ in ${\cal M}_2$ \\
\> \> begin \\
\> \> \> If ( $P_m \cap P_n$ ) \\
\> \> \> begin \\
\> \> \> \> ${\cal L}$ $\Leftarrow$ $P_m \cap P_n$; \\
\> \> \> \> $InterList( P_m )$ $\Leftarrow$ $InterList( P_m )$ $\cup$ ${\cal L}$; \\
\> \> \> \> $InterList( P_n )$ $\Leftarrow$ $InterList( P_n )$ $\cup$ ${\cal L}$; \\
\> \> \> end \\
\> \> end \\
\> end
\end{tabbing}
\end{tt}
\end{bf}

At the end of the first, intersection stage, each polygon, $P_m$ in
either ${\cal M}_1$ (or $P_n \in {\cal M}_2$) contains in its
$InterList( P_m )$ the set of line segments of the intersection(s) of
$P_m$ with all polygons in the other model.  In order to figure out
orientation, each intersection line segment also contains an
orientation hint as to the inside direction of the two polygons that
intersect at that line segment.  This list of intersection line
segments can now be processed:

\begin{bf}
\begin{tt}
\begin{tabbing}
\hspace{2em}\=\hspace{2em}\=\hspace{2em}\=\hspace{2em}\=\hspace{2em}\=\kill
\> For each polygon $P_m$ in ${\cal M}_1$ \\
\> begin \\
\> \> $Polys$ $\Leftarrow$ $InterList( P_m )$ sorted into polylines; \\
\> \> $OpenList( P_m )$ $\Leftarrow$ Open polylines of $Polys$; \\
\> \> $ClosedList( P_m )$ $\Leftarrow$ Closed polylines of $Polys$; \\
\> end \\
\> For each polygon $P_n$ in ${\cal M}_2$ \\
\> begin \\
\> \> $Polys$ $\Leftarrow$ $InterList( P_n )$ sorted into polylines; \\
\> \> $OpenList( P_n )$ $\Leftarrow$ Open polylines of $Polys$; \\
\> \> $ClosedList( P_n )$ $\Leftarrow$ Closed polylines of $Polys$; \\
\> end \\
\end{tabbing}
\end{tt}
\end{bf}

Open polylines, $OpenList( P_m )$, are intersection polylines that are
open.  These polylines starts on the boundary of polygon $P_m$ and
also ends on the boundary, at a different boundary location though.
In contrast, the set of closed polylines, $ClosedList( P_n )$, contains
closed loops that are completely contained in $P_n$.

The polygons are now ready for {\em trimming}.  Depending upon the Boolean
operation that is requested, the inside or the outside of ${\cal M}_1$ or
${\cal M}_2$ is selected for the output result.  Denote by
${\cal M}_i \ Out \ {\cal M}_j$ (${\cal M}_i \ In \ {\cal M}_j$) the region of
${\cal M}_i$ that is outside (inside) ${\cal M}_j$. Then,

\bigskip 
\begin{center}
\begin{tabular}{||c|c||} \hline
Boolean Operation & implementation as \\ \hline \hline
${\cal M}_1 \cup {\cal M}_1$ & $\{ {\cal M}_1\ Out\ {\cal M}_2 \} \cup
			     \{ {\cal M}_2\ Out\ {\cal M}_1 \}$ \\
${\cal M}_1 \cap {\cal M}_1$ & $\{ {\cal M}_1\ In\ {\cal M}_2 \} \cup
			     \{ {\cal M}_2\ In\ {\cal M}_1 \}$ \\
${\cal M}_1 - {\cal M}_1$ & $\{ {\cal M}_1\ Out\ {\cal M}_2 \} \cup
			     Inverse\{ {\cal M}_2\ In\ {\cal M}_1 \}$ \\ \hline
\end{tabular}
\end{center}
\bigskip 

Note the $Inverse$ operation that reverses the orientation inside
out.

At this point, all Boolean operations are translated into two complementary
type of operations, $In$ and $Out$. The computation of
${\cal M}_i \ Out \ {\cal M}_j$ is conducted by tracing the outside regions
of the polygons of ${\cal M}_i$ that {\em intersect} ${\cal M}_j$ and
{\em propagating} this inside--outside information to the non intersecting
polygons:
\begin{bf}
\begin{tt}
\begin{tabbing}
\hspace{2em}\=\hspace{2em}\=\hspace{2em}\=\hspace{2em}\=\hspace{2em}\=\kill
\> For each polygon $P_m$ in ${\cal M}_1$ with non empty $OpenList( P_m )$ \\
\> \mbox{\hspace{2.425in}} or non empty $ClosedList( P_m )$ \\
\> begin \\
\> \> $P_m^{in}$ $\Leftarrow$ Inside region of $P_m$; \\
\> \> $P_m^{out}$ $\Leftarrow$ Outside region of $P_m$; \\
\\
\> \> For each {\em original} edge, $e \in P_m^{in}$ \\
\> \> begin \\
\> \> \> Push $e$ onto $InsideStack$; \\
\> \> end \\
\> \> For each {\em original} edge, $e \in P_m^{out}$ \\
\> \> begin \\
\> \> \> Push $e$ onto $OutSideStack$; \\
\> \> end \\
\> end \\
\\
\> While $InsideStack$ not empty\\
\> begin \\
\> \> $e$ $\Leftarrow$ edge poped from top of $InsideStack$; \\
\> \> $P_m$, $P_l$ $\Leftarrow$ two polygons sharing $e$ in ${\cal M}_1$; \\
\\
\> \> If ( $P_m$ has empty $OpenList( P_m )$ and empty $ClosedList( P_m )$, \\
\> \> \mbox{\hspace{3in}} and $P_m$ is not classified yet ) \\
\> \> begin \\
\> \> \> Classify $P_m$ as inside; \\
\> \> \> push all edges of $P_m$, but $e$, onto $InsideStack$; \\
\> \> end \\
\> \> If ( $P_l$ has empty $OpenList( P_l )$ and empty $ClosedList( P_l )$, \\
\> \> \mbox{\hspace{3in}} and $P_l$ is not classified yet ) \\
\> \> begin \\
\> \> \> Classify $P_l$ as inside; \\
\> \> \> push all edges of $P_l$, but $e$, onto $InsideStack$; \\
\> \> end \\
\> end \\
\\
\> While $OutsideStack$ not empty\\
\> begin \\
\> \> $e$ $\Leftarrow$ edge poped from top of $outsideStack$; \\
\> \> $P_m$, $P_l$ $\Leftarrow$ two polygons sharing $e$ in ${\cal M}_1$; \\
\\
\> \> If ( $P_m$ has empty $OpenList( P_m )$ and empty $ClosedList( P_m )$, \\
\> \> \mbox{\hspace{3in}} and $P_m$ is not classified yet ) \\
\> \> begin \\
\> \> \> Classify $P_m$ as outside; \\
\> \> \> push all edges of $P_m$, but $e$, onto $OutsideStack$; \\
\> \> end \\
\> \> If ( $P_l$ has empty $OpenList( P_l )$ and empty $ClosedList( P_l )$, \\
\> \> \mbox{\hspace{3in}} and $P_l$ is not classified yet ) \\
\> \> begin \\
\> \> \> Classify $P_l$ as outside; \\
\> \> \> push all edges of $P_l$, but $e$, onto $OutsideStack$; \\
\> \> end \\
\> end \\
\end{tabbing}
\end{tt}
\end{bf}

At the end of this process, $M_1$ is completely split into two sets of
polygons, classified as either inside $M_2$ or outside $M_2$.  An
identical spliting process can be applied to ${\cal M}_2$.  The proper
inside/outside lists of ${\cal M}_1$ can now be combined with the
proper list of classified polygons of ${\cal M}_2$ to yield the proper
result of the requested Boolean operation.

While this completes the Boolean operation, vast room for improvements
can be found, with some improvements implemented in IRIT.  For
example, the first step of the intersection computation necessitates
an $O(n^2)$ polygon--polygon intersection tests, where $n$ is in the
order of the number of polygons in ${\cal M}_i$.  Clearly, one can
hope for a better time complexity.  Bounding boxes, hierarchical
bounding boxes, or three dimensional space sweep techniques can all be
applied to reduce the overhead invested in these $O(n^2)$ tests.
Efficient sorting of line segments within a polygon, when forming the
open and closed loops is another optimization consideration.

\section{Library Functions}
\input{prog_man/bool_lib.tex}

%%%%%%%%%%%%%%%%%%%%%%%%%%%%%%%%%%%%%%%%%%%%%%%%%%%%%%%%%%%%%%%%%%%%%%%%%%%%%
\chapter{CAGD Library, cagd\_lib}

\section{General Information}

This library provides a rich set of function to create, convert,
display and process freeform Bezier and NURBs curves and surfaces. The
interface of the library is defined in {\em include/cagd\_lib.h}. This
library mainly supports low level freeform curve and surface
operations.  Supported are curves and surfaces from scalars to five
dimensions as E1/P1 to E5/P5 using the {\bf CagdPointType}. Pi is a
rational (projective) version of Ei, with an additional W coefficient.
Polynomial in the power basis have some very limited support as well.
Different data structures to hold UV parameter values, control points,
vectors, planes, bounding boxes, polylines and polygons are defined as
well as the data strcutures to hold the curves and surfaces
themselves,
\begin{verbatim}
typedef struct CagdCrvStruct {
    struct CagdCrvStruct *Pnext;
    struct IPAttributeStruct *Attr;
    CagdGeomType GType;
    CagdPointType PType;
    int Length;            /* Number of control points (== order in Bezier). */
    int Order;	    /* Order of curve (only for Bspline, ignored in Bezier). */
    CagdBType Periodic;			   /* Valid only for Bspline curves. */
    CagdRType *Points[CAGD_MAX_PT_SIZE];     /* Pointer on each axis vector. */
    CagdRType *KnotVector;
} CagdCrvStruct;

typedef struct CagdSrfStruct {
    struct CagdSrfStruct *Pnext;
    struct IPAttributeStruct *Attr;
    CagdGeomType GType;
    CagdPointType PType;
    int ULength, VLength;	 /* Mesh size in the tensor product surface. */
    int UOrder, VOrder;   /* Order in tensor product surface (Bspline only). */
    CagdBType UPeriodic, VPeriodic;      /* Valid only for Bspline surfaces. */
    CagdRType *Points[CAGD_MAX_PT_SIZE];     /* Pointer on each axis vector. */
    CagdRType *UKnotVector, *VKnotVector;
} CagdSrfStruct;
\end{verbatim}

Curves and surfaces have a geometric type {\bf GType} to prescribe the
type of entity (such as CAGD\_SBEZIER\_TYPE for Bezier surface) and a
point type {\bf PType} to prescribe the point type of the entity (such
as CAGD\_PT\_E3\_TYPE for three dimensional Euclidean control points).
{\bf Length} and {\bf Order} slots are used to hold the number of
control points in the mesh and or control polygon and the order(s) of
the basis functions. {\bf Periodic} flag(s) are used to denote
periodic end conditions. In addition, {\bf KnotVector} slot(s) are
used if the entity exploits Bspline basis functions, or NULL
otherwise.

The control polygon and/or mesh itself is organized in the {\bf
Points} slot as a vector of size {\bf CAGD\_MAX\_PT\_SIZE} of vectors
of {\bf CagdRType}s. For surfaces, the mesh is ordered U first and the
macros of {\bf CAGD\_NEXT\_U} {\bf CAGD\_NEXT\_V}, and {\bf
CAGD\_MESH\_UV} can be used to determine indices in the mesh.

All structures in the cagd library can be allocated using New
constrcutures (i.e. {\bf CagdUVNew} or {\bf CagdCrfNew}, freed using
Free destructores (i.e. {\bf CagdSrfFree} or {CagdBBoxFree}, linked
list free using FreeList destructores (i.e. {\bf
CagdPolylineFreeList}), and copied using copy constructores {i.e. {\bf
CagdPtCopy} or {\bf CagdCtlPtCopyList}).

This library has its own error handler, which by default prints an
error message and exit the program called {\bf CagdFatalError}.

Most globals in this library have a prefix of {\bf Cagd} for general
cagd routines. Prefix of {\bf Bzr} is used for Bezier routines, prefix
of {\bf Bsp} for Bspline specific routines, prefix of {\bf Cnvrt} for
conversion routines, and {\bf Afd} for adaptive forward differencing
routines.

\section{Library Functions}
\input{prog_man/cagd_lib.tex}

%%%%%%%%%%%%%%%%%%%%%%%%%%%%%%%%%%%%%%%%%%%%%%%%%%%%%%%%%%%%%%%%%%%%%%%%%%%%%
\chapter{Geometry Library, geom\_lib}

\section{General Information}

This library handles general computational geometry algorithms and
geometric queries such as a distance between two lines, bounding
boxes, convexity and convex hull of polygons, polygonal constructors
of primitives (cylinders, spheres, etc.), basic scan conversion
routines, etc.

\section{Library Functions}
\input{prog_man/geom_lib.tex}

%%%%%%%%%%%%%%%%%%%%%%%%%%%%%%%%%%%%%%%%%%%%%%%%%%%%%%%%%%%%%%%%%%%%%%%%%%%%%
\chapter{Graphics Library, grap\_lib}

\section{General Information}

This library handles general drawing and display algorithms, including
tesselation of all geometric objects, such as curves and surfaces, into
displayable primitives, i.e. polygons and polylines.

\section{Library Functions}
\input{prog_man/grap_lib.tex}

%%%%%%%%%%%%%%%%%%%%%%%%%%%%%%%%%%%%%%%%%%%%%%%%%%%%%%%%%%%%%%%%%%%%%%%%%%%%%
\chapter{Model Library, mdl\_lib}

\section{General Information}

This library provides the necessary tools to represent and process
models.  Models are sets of trimmed surfaces forming a closed
2-manifold shell.  Models are typically the result of Boolean operations
over freeform (trimmed) NURBs geometry but can also be constructed
directly or via other schemes.

\section{Library Functions}
\input{prog_man/mdl_lib.tex}

%%%%%%%%%%%%%%%%%%%%%%%%%%%%%%%%%%%%%%%%%%%%%%%%%%%%%%%%%%%%%%%%%%%%%%%%%%%%%
\chapter{Miscelleneous Library, misc\_lib}

\section{General Information}

This library holds general miscelleneous functions such as reading
configuration files, low level homogeneous matrices computation, low
level attributes and general machine specific low level routines.
Several header files can be found for this library:
\begin{center}
\begin{tabular}{||c|c||} \hline
    Header (include/*.h) & Functionality \\ \hline
    config.h   & Manipulation of configuration files (*.cfg files) \\
    dist\_pts.h & Enegry based distribution of points. \\
    gen\_mat.h  & Homogeneous matrices manipulation \\
    getarg.h   & Command line parsing for application tools \\
    imalloc.h  & Low level dynamic memory functions for IRIT \\
    miscattr.h & Low level attribute related functions \\
    priorque.h & An implementation of a priority queue \\
    xgeneral.h & Low level, machine specific, routines \\ \hline
\end{tabular}
\end{center}

\section{Library Functions}
\input{prog_man/misc_lib.tex}

%%%%%%%%%%%%%%%%%%%%%%%%%%%%%%%%%%%%%%%%%%%%%%%%%%%%%%%%%%%%%%%%%%%%%%%%%%%%%
\chapter{Multi variate functions, mvar\_lib}

\section{General Information}

This library holds functions to handle functions of arbitrary number
of variables. In this context curves (univariate), surfaces
(bivariate) and trivariates are special cases. This library provides a
rich set of functions to manipulate freeform Bezier and/or NURBs
multivariates. This library heavily depends on the cagd an symb
libraries. Functions are provided to create, copy, and destruct
multivariates, to extract isoparametric lower degree multivariates, to
evaluate, refine and subdivide, to read and write multivariates, to
differentiate, degree raise, make compatible and convert back and
forth to/from curves, surfaces, and trivariates.

A multivariate has m orders, m Length prescriptions and,
possibly, m knot vectors (if Bspline).  In addition it contains
an m dimensional volume of control points,
\begin{verbatim}
typedef struct MvarMVStruct {
    struct MvarMVStruct *Pnext;
    struct IPAttributeStruct *Attr;
    MvarGeomType GType;
    CagdPointType PType;
    int Dim;		      /* Number of dimensions in this multi variate. */
    int *Lengths;               /* Dimensions of mesh size in multi-variate. */
    int *SubSpaces;	   /* SubSpaces[i] = Prod(i = 0, i-1) of Lengths[i]. */
    int *Orders;                /* Orders of multi varariate (Bspline only). */
    CagdBType *Periodic;            /* Periodicity - valid only for Bspline. */
    CagdRType *Points[CAGD_MAX_PT_SIZE];     /* Pointer on each axis vector. */
    CagdRType **KnotVectors;
} MvarMVStruct;
\end{verbatim}

The interface of the library is defined in {\em include/mvar\_lib.h}. 

This library has its own error handler, which by default prints an
error message and exit the program called {\bf MvarFatalError}.

All globals in this library have a prefix of {\bf Mvar}.

\section{Library Functions}
\input{prog_man/mvar_lib.tex}

%%%%%%%%%%%%%%%%%%%%%%%%%%%%%%%%%%%%%%%%%%%%%%%%%%%%%%%%%%%%%%%%%%%%%%%%%%%%%
\chapter{Prsr Library, prsr\_lib}

\section{General Information}

This library provides the data file interface for IRIT. Functions are
provided to read and write data files, both compressed (on unix only,
using {\em compress}), and uncompressed, in binary and/or ascii text
modes. This library is also used to exchange data between the IRIT
server and the display devices' clients.  Several header files can be
found for this library:
\begin{center}
\begin{tabular}{||c|c||} \hline
    Header (include/*.h) & Functionality \\ \hline
    allocate.h & High level dynamic allocation of objects \\
    attribut.h & High level attributes for objects \\
    ip\_cnvrt.h & Freeform to polygon and polyline high level conversion \\
    iritprsr.h & Main interface to reading and writing data \\
    irit\_soc.h & Socket communication for data exchange \\ \hline
\end{tabular}
\end{center}

\section{Library Functions}
\input{prog_man/prsr_lib.tex}

%%%%%%%%%%%%%%%%%%%%%%%%%%%%%%%%%%%%%%%%%%%%%%%%%%%%%%%%%%%%%%%%%%%%%%%%%%%%%
\chapter{Rendering Library, rndr\_lib}

\section{General Information}

This library provides a powerful full screen scan conversion Z-buffer
tool to process IRIT geometry and convert it into images.  This
library allows one to scan convert any IRIT geometry including
polylines and curves that are converted to skinny polygons on the fly.
The library offers regular scan conversion with flat, Gouraud, and
Phong shading and several antialiasing filters along with advanced
features such as transparency and animation support, and width depth
cueing on polyline/curves rendering.  The library also provide direct
access to the depth Z-buffer as well as a stencil buffer.

\section{Library Functions}
\input{prog_man/rndr_lib.tex}

%%%%%%%%%%%%%%%%%%%%%%%%%%%%%%%%%%%%%%%%%%%%%%%%%%%%%%%%%%%%%%%%%%%%%%%%%%%%%
\chapter{Symbolic Library, symb\_lib}

\section{General Information}

This library provides a rich set of functions to symbolically manipulate
freeform curves and surfaces. This library heavily depends on the cagd
library. Functions are provided to low level add, subtract, and multiply
freeform curves and surfaces, to compute fields such as curvature,
and to extract singular points such as extremums, zeros, and inflections.
High level tools to metamorph curves and surfaces, to compute layout (prisa)
of freeform surfaces, to compute offset approximations of curves and
surfaces, and to compose curves and surfaces are also provided.

The interface of the library is defined in {\em include/symb\_lib.h}. 

This library has its own error handler, which by default prints an
error message and exit the program called {\bf SymbFatalError}.

Globals in this library have a prefix of {\bf Symb} for general
symbolic routines. Prefix of {\bf Bzr} is used for Bezier routines,
and prefix of {\bf Bsp} for Bspline specific routines.

\section{Library Functions}
\input{prog_man/symb_lib.tex}

%%%%%%%%%%%%%%%%%%%%%%%%%%%%%%%%%%%%%%%%%%%%%%%%%%%%%%%%%%%%%%%%%%%%%%%%%%%%%
\chapter{Trimmed surfaces Library, trim\_lib}

\section{General Information}

This library provides a set of functions to manipulate freeform
trimmed Bezier and/or NURBs surfaces. This library heavily depends on
the cagd library. Functions are provided to create, copy, and destruct
trimmed surfaces to extract isoparametric curves, to evaluate, refine
and subdivide, to read and write trimmed surfaces, degree raise, and
approximate using polygonal representations.  A trimming surface is
defined out of a tensor product surface and a set of trimming loops
that trims out portions of the parametric space of the surface,
\begin{verbatim}
typedef struct TrimSrfStruct {
    struct TrimSrfStruct *Pnext;
    IPAttributeStruct *Attr;
    int Tags;
    CagdSrfStruct *Srf;			  /* Surface trimmed by TrimCrvList. */
    TrimCrvStruct *TrimCrvList;		         /* List of trimming curves. */
} TrimSrfStruct;
\end{verbatim}

Each trimming loop consists of a set of trimming curve segments:
\begin{verbatim}
typedef struct TrimCrvStruct {
    struct TrimCrvStruct *Pnext;
    IPAttributeStruct *Attr;
    TrimCrvSegStruct *TrimCrvSegList;    /* List of trimming curve segments. */
} TrimCrvStruct;
\end{verbatim}

Each trimming curve segment contains a representation for the curve in the
UV space of the surface as well as a representation in the Euclidean space,
\begin{verbatim}
typedef struct TrimCrvSegStruct {
    struct TrimCrvSegStruct *Pnext;
    IPAttributeStruct *Attr;
    CagdCrvStruct *UVCrv;    /* Trimming crv segment in srf's param. domain. */
    CagdCrvStruct *EucCrv;       /* Trimming curve as an E3 Euclidean curve. */
} TrimCrvSegStruct;
\end{verbatim}


The interface of the library is defined in {\em include/trim\_lib.h}. 

This library has its own error handler, which by default prints an
error message and exit the program called {\bf TrimFatalError}.

All globals in this library have a prefix of {\bf Trim}.

\section{Library Functions}
\input{prog_man/trim_lib.tex}

%%%%%%%%%%%%%%%%%%%%%%%%%%%%%%%%%%%%%%%%%%%%%%%%%%%%%%%%%%%%%%%%%%%%%%%%%%%%%
\chapter{Trivariate Library, triv\_lib}

\section{General Information}

This library provides a rich set of functions to manipulate freeform
Bezier and/or NURBs trivariate. This library heavily depends on the cagd
library. Functions are provided to create, copy, and destruct trivariates,
to extract isoparametric surfaces, to evaluate, refine and subdivide, to
read and write trivariates, to differentiate, degree raise, make compatible
and approximate iso-surface at iso values using polygonal representations.

A trivariate has three orders, three Length prescriptions and,
possibly, three knot vectors (if Bspline).  In addition it contains
a three dimensional volume of control points,
\begin{verbatim}
typedef struct TrivTVStruct {
    struct TrivTVStruct *Pnext;
    struct IPAttributeStruct *Attr;
    TrivGeomType GType;
    CagdPointType PType;
    int ULength, VLength, WLength;/* Mesh size in tri-variate tensor product.*/
    int UVPlane;	  /* Should equal ULength * VLength for fast access. */
    int UOrder, VOrder, WOrder;       /* Order in trivariate (Bspline only). */
    CagdBType UPeriodic, VPeriodic, WPeriodic;    /* Valid only for Bspline. */
    CagdRType *Points[CAGD_MAX_PT_SIZE];     /* Pointer on each axis vector. */
    CagdRType *UKnotVector, *VKnotVector, *WKnotVector;
} TrivTVStruct;
\end{verbatim}

The interface of the library is defined in {\em include/triv\_lib.h}. 

This library has its own error handler, which by default prints an
error message and exit the program called {\bf TrivFatalError}.

All globals in this library have a prefix of {\bf Triv}.

\section{Library Functions}
\input{prog_man/triv_lib.tex}

%%%%%%%%%%%%%%%%%%%%%%%%%%%%%%%%%%%%%%%%%%%%%%%%%%%%%%%%%%%%%%%%%%%%%%%%%%%%%
\chapter{Triangular Library, trng\_lib}

\section{General Information}

This library provides a subset of functions to manipulate freeform
triangular Bezier and Bspline patches. This library heavily depends on
the cagd library. Functions are provided to create, copy, and destruct
triangular patches, to extract isoparametric curves, to evaluate,
refine and subdivide, to read and write triangular patches, to
differentiate, and approximate using polygonal representations.

A triangular patch has one prescription of Length and one prescription
of Order, the total Order and the length of an edge of the triangle.
The control mesh mesh has Length * (Length + 1) / 2 control points,
\begin{verbatim}
typedef struct TrngTriangSrfStruct {
    struct TrngTriangSrfStruct *Pnext;
    struct IPAttributeStruct *Attr;
    TrngGeomType GType;
    CagdPointType PType;
    int Length;		    /* Mesh size (length of edge of triangular mesh. */
    int Order;		      /* Order of triangular surface (Bspline only). */
    CagdRType *Points[CAGD_MAX_PT_SIZE];     /* Pointer on each axis vector. */
    CagdRType *KnotVector;
} TrngTriangSrfStruct;
\end{verbatim}

The interface of the library is defined in {\em include/trng\_lib.h}. 

This library has its own error handler, which by default prints an
error message and exit the program called {\bf TrngFatalError}.

All globals in this library have a prefix of {\bf Trng}.

\section{Library Functions}
\input{prog_man/trng_lib.tex}

%%%%%%%%%%%%%%%%%%%%%%%%%%%%%%%%%%%%%%%%%%%%%%%%%%%%%%%%%%%%%%%%%%%%%%%%%%%%%
\chapter{User Library, user\_lib}

\section{General Information}

This library includes user interface related geometrical functions such
as ray surface intersection (for mouse click/select operations), etc.

The interface of the library is defined in {\em include/user\_lib.h}. 

\section{Library Functions}
\input{prog_man/user_lib.tex}

%%%%%%%%%%%%%%%%%%%%%%%%%%%%%%%%%%%%%%%%%%%%%%%%%%%%%%%%%%%%%%%%%%%%%%%%%%%%%
\chapter{Extra Library, xtra\_lib}

\section{General Information}

This library is not an official part of IRIT and contains public
domain code that is used by routines in IRIT.

The interface of the library is defined in {\em include/extra\_fn.h}. 

\section{Library Functions}
\input{prog_man/xtra_lib.tex}

%%%%%%%%%%%%%%%%%%%%%%%%%%%%%%%%%%%%%%%%%%%%%%%%%%%%%%%%%%%%%%%%%%%%%%%%%%%%%
\chapter{Programming Examples}
\label{chap-examples}

This chapter describes several simple examples of C programs that
exploits the libraries of IRIT. All external function are defined in
the {\em include} subdirectory of IRIT and one can 'grep' there for the
exact include file that contains a certain function name. All C programs
in all C files should include 'irit\_sm.h' as their first include file
of IRIT, before any other include file of IRIT. Header files are set so
C++ code can compile and link to it without any special treatment.

\section{Setting up the Compilation Environment}

In order to compile programs that uses the libraries of IRIT, a makefile
has to be constructed. Assuming IRIT is installed in {\em /usr/local/irit},
here is a simple makefile that can be used (for a unix environment):

\begin{verbatim}
IRIT_DIR = /usr/local/irit

include $(IRIT_DIR)/makeflag.unx

OBJS	= program.o

program: $(OBJS)
	$(CC) $(CFLAGS) -o program $(OBJS) $(LIBS) -lm $(MORELIBS)
\end{verbatim}

The simplicity of this makefile is drawn from the complexity of
makeflag.unx. The file makeflag.unx sets the CC, CFLAGS, LIBS, and
MORELIBS for the machined using among other things. Furthermore,
makeflag.unx also sets the default compilation rules from C sources to
object files.  The file makeflag.unx had to be modified once, when
IRIT was installed on this system. If the used system is not a unix
environment, then the file makefile.unx will have to be replaced with
the proper makeflag file.  In an OS2 environment, using the emx gcc
compiler, the makefile is even simpler since the linking rule is also
defined in makeflag.os2:

\begin{verbatim}
IRIT_DIR = \usr\local\irit

include $(IRIT_DIR)\makeflag.os2

OBJS	= program.o

program.exe: $(OBJS)
\end{verbatim}

Finally, here is a proper makefile for Windows NT:

\begin{verbatim}
IRIT_DIR = \usr\local\irit

include $(IRIT_DIR)\makeflag.wnt

OBJS	= program.obj

program.exe: $(OBJS)
        $(IRITCONLINK) -out:program.exe $(OBJS) $(LIBS) $(W32CONLIBS)
\end{verbatim}

\section{Simple C Programs using IRIT}

Now that we have an idea how to compile C code using IRIT, here are several
examples to read, manipulate and write IRIT data files. You will be able to
find all these examples in the doc/cexample directory.

\subsection{Boolean Operations over Polyhedra (boolean.c)}

This example demonstates the use of the Boolean operations package
(bool\_lib), creates few polygonal primitives and apply Boolean
operations between them.  The resulting model is dumped to stdout.

\input{cexample/boolean.tex}

\subsection{Distance Maps to Planar Curves (dist\_map.c)}

This example demonstates the use of symbolic distance maps
computations to freeform curves, in the plane, and the sampling
of the map into an image that is dumped out.

\input{cexample/dist_map.tex}

\subsection{Importance Edges Evaluations in Polygonal Meshes (imprtnc.c)}

This example demonstates the tesselation of arbitrary input model(s)
into polygons and the probabalistic estimation of importance edges in the
polygonal geometry.  Dumped out is the polygonal geometry with vertices
having UV coordinates if available, normals if available, and importance.

\begin{verbatim}
/*****************************************************************************
*   Computes the importance of vertices and dump as vertex list + polygons   *
******************************************************************************
* (C) Gershon Elber, Technion, Israel Institute of Technology                *
******************************************************************************
* Written by:  Gershon Elber                                Ver 1.0, June 1995   *
*****************************************************************************/

#include "irit_sm.h"
#include "allocate.h"
#include "iritprsr.h"
#include "geom_lib.h"
#include "misc_lib.h"
#include "ip_cnvrt.h"

IRIT_STATIC_DATA char
    *CtrlStr = "Imprtnc c%- h%- DFiles!*s";
IRIT_STATIC_DATA int GlblVrtxImportanceCount,
    GlblCrvtrInfoFlag = FALSE;
IRIT_STATIC_DATA IrtRType GlblVrtxImportanceVal;
IRIT_STATIC_DATA IPVertexStruct *GlblVrtxImportance;

static void DumpOneTraversedObject(IPObjectStruct *PObj, IrtHmgnMatType Mat);
static IPObjectStruct *SetCurvatureEstimates(IPObjectStruct *PObj);
static void DumpOneObjData(IPObjectStruct *PObj);
static void ProcessVertexImportance(IPVertexStruct *V1,
                                    IPVertexStruct *V2,
                                    IPPolygonStruct *Pl1,
                                    IPPolygonStruct *Pl2);
static void GenPolyImportance(IPObjectStruct *PObj);

void main(int argc, char **argv)
{
    int NumFiles, Error,
        HelpFlag = FALSE;
    char **FileNames;
    IPObjectStruct *PObjects;
    IrtHmgnMatType CrntViewMat;

    if ((Error = GAGetArgs(argc, argv, CtrlStr, &GlblCrvtrInfoFlag,
                           &HelpFlag, &NumFiles, &FileNames)) != 0) {
        GAPrintErrMsg(Error);
        GAPrintHowTo(CtrlStr);
        exit(1);
    }

    if (HelpFlag) {
        fprintf(stderr, "This is Importance testing...\n");
        GAPrintHowTo(CtrlStr);
        exit(0);
    }

    /* Get the data files: */
    IPSetFlattenObjects(FALSE);
    if ((PObjects = IPGetDataFiles(FileNames, NumFiles, TRUE, FALSE)) == NULL)
        exit(1);
    PObjects = IPResolveInstances(PObjects);

    if (IPWasPrspMat)
        MatMultTwo4by4(CrntViewMat, IPViewMat, IPPrspMat);
    else
        IRIT_GEN_COPY(CrntViewMat, IPViewMat, sizeof(IrtHmgnMatType));

    /* Here some useful parameters to play with in tesselating freeforms: */
    IPFFCState.FineNess = 15; /* Resolution of tesselation, larger is finer. */
    IPFFCState.ComputeUV = TRUE;       /* Wants UV coordinates for textures. */
    IPFFCState.FourPerFlat = TRUE;/* 4 polygons per ~flat patch, 2 otherwise.*/
    IPFFCState.LinearOnePolyFlag = TRUE;   /* Linear srf generates one poly. */

    IPTraverseObjListHierarchy(PObjects, CrntViewMat, DumpOneTraversedObject);

    IPFreeObjectList(PObjects);
}

/*****************************************************************************
* DESCRIPTION:                                                               *
*   Update curvature property attribute to vertices of the given poly model. *
*                                                                            *
* PARAMETERS:                                                                *
*   PObj:   Poly model to udpate curvatrue info into.                             *
*                                                                            *
* RETURN VALUE:                                                              *
*   IPObjectStruct *:   A poly model, similar to PObj, but with curvature    *
*                        information attached as attributes.                  *
*****************************************************************************/
static IPObjectStruct *SetCurvatureEstimates(IPObjectStruct *PObj)
{
    int OldCirc = IPSetPolyListCirc(TRUE);
    IPPolygonStruct *Pl;
    IPObjectStruct *PTmp1, *PTmp2;

    /* Convert to a regular polygonal model with triangles only. */
    for (Pl = PObj -> U.Pl; Pl != NULL; Pl = Pl -> Pnext) {
        if (IPVrtxListLen(Pl -> PVertex) != 3)
            break;
    }
    PTmp2 = IPCopyObject(NULL, PObj, FALSE);
    GMVrtxListToCircOrLin(PTmp2 -> U.Pl, TRUE);
    if (Pl != NULL) {
        PTmp1 = GMConvertPolysToTriangles(PTmp2);
        IPFreeObject(PTmp2);
        PTmp2 = GMRegularizePolyModel(PTmp1);
        IPFreeObject(PTmp1);
    }

    GMVrtxListToCircOrLin(PTmp2 -> U.Pl, FALSE);
    IPSetPolyListCirc(OldCirc);

    GMPlCrvtrSetCurvatureAttr(PTmp2 -> U.Pl, 1, TRUE);

    return PTmp2;
}

/*****************************************************************************
* DESCRIPTION:                                                               *
*   Call back function of IPTraverseObjListHierarchy. Called on every non    *
* list object found in hierarchy.                                            *
*                                                                            *
* PARAMETERS:                                                                *
*   PObj:       Non list object to handle.                                   *
*   Mat:        Transformation matrix to apply to this object.               *
*                                                                            *
* RETURN VALUE:                                                              *
*   void                                                                     *
*****************************************************************************/
static void DumpOneTraversedObject(IPObjectStruct *PObj, IrtHmgnMatType Mat)
{
    IPObjectStruct *PObjs;

    if (IP_IS_FFGEOM_OBJ(PObj))
        PObjs = IPConvertFreeForm(PObj, &IPFFCState);
    else
        PObjs = PObj;

    for (PObj = PObjs; PObj != NULL; PObj = PObj -> Pnext) {
        if (IP_IS_POLY_OBJ(PObj)) {
            if (GlblCrvtrInfoFlag) {
                IPObjectStruct
                    *PTri = SetCurvatureEstimates(PObj);

                DumpOneObjData(PTri);
                IPFreeObject(PTri);
            }
            else
                DumpOneObjData(PObj);
        }
    }
}

/*****************************************************************************
* DESCRIPTION:                                                               *
*   Convert the polygonal mesh into a vertex list with importance and the    *
* polygons as indices into this list.                                        *
*                                                                            *
* PARAMETERS:                                                                *
*   PObj:   A polygonal mesh to dump.                                        *
*                                                                            *
* RETURN VALUE:                                                              *
*   void                                                                     *
*****************************************************************************/
static void DumpOneObjData(IPObjectStruct *PObj)
{
    IPPolyVrtxIdxStruct
        *PVIdx = IPCnvPolyToPolyVrtxIdxStruct(PObj, TRUE, 0);
    int        i,
        **Polygons = PVIdx -> Polygons;
    IPVertexStruct
        **Vertices = PVIdx -> Vertices;

    /* Compute importance for the vertices as "Imprt" attributes.  Note the */
    /* PVIdx data structure points on the original vertices so we are fine. */
    GenPolyImportance(PObj);

    /* Dump the vertices: */
    fprintf(stderr, "OBJECT \"%s\" - VERTICES:\n", PObj -> ObjName);
    for (i = 0; Vertices[i] != NULL; i++) {
        IrtRType R;
        float *uv;

        printf("%3d:  %6.3f %6.3f %6.3f", i,
               Vertices[i] -> Coord[0], 
               Vertices[i] -> Coord[1], 
               Vertices[i] -> Coord[2]);
        if (IP_HAS_NORMAL_VRTX(Vertices[i]))
            printf(" [%6.3f %6.3f %6.3f]",
                   Vertices[i] -> Normal[0], 
                   Vertices[i] -> Normal[1], 
                   Vertices[i] -> Normal[2]);
        if ((uv = AttrGetUVAttrib(Vertices[i] -> Attr, "uvvals")) != NULL)
            printf(" {%6.3f %6.3f}", uv[0], uv[1]);

        R = AttrGetRealAttrib(Vertices[i] -> Attr, "Imprt");
        if (!IP_ATTR_IS_BAD_REAL(R))
            printf(" (%6.3f)", R);

        if (GlblCrvtrInfoFlag) {
            printf("\n\t<K1=%6.3f D1 = %s>\n\t<K2=%6.3f D2 = %s>",
                   AttrGetRealAttrib(Vertices[i] -> Attr, "K1Curv"),
                   AttrGetStrAttrib(Vertices[i] -> Attr, "D1"),
                   AttrGetRealAttrib(Vertices[i] -> Attr, "K2Curv"),
                   AttrGetStrAttrib(Vertices[i] -> Attr, "D2"));
        }

        printf("\n");
    }

    fprintf(stderr, "\nOBJECT \"%s\" - Vertices in polygons:\n",
            PObj -> ObjName);
    for (i = 0; PVIdx -> PPolys[i] != NULL; i++) {
        IPPolyPtrStruct
            *PPoly = PVIdx -> PPolys[i];

        printf("%3d: ", i);
        for ( ; PPoly != NULL; PPoly = PPoly -> Pnext) {
            printf("%3d ", AttrGetIntAttrib(PPoly -> Poly -> Attr, "_PIdx"));
        }
        printf("\n");
    }

    /* Dump the polygons: */
    fprintf(stderr, "\nOBJECT \"%s\" - Polygons from vertices:\n",
            PObj -> ObjName);
    for (i = 0; Polygons[i] != NULL; i++) {
        int *Pl = Polygons[i];

        printf("%3d: ", i);
        while (*Pl >= 0)
            fprintf(stderr, " %5d", *Pl++);

        fprintf(stderr, "    -1\n");
    }

    IPPolyVrtxIdxFree(PVIdx);
}

/*****************************************************************************
* DESCRIPTION:                                                               *
*   Call back function for GMPolyAdjacncyVertex to process every edge of a   *
* given vertex.  The edge is provided as (V, V -> Pnext)                     *
*                                                                            *
* PARAMETERS:                                                                *
*   V1, V2:    Two vertices defining this edge.  Note the vertices are NOT   *
*               necessarily chained together into a list.                     *
*   Pl1, Pl2:  The two polygons that share this edge.  The edge (V1, V2) is  *
*               in both Pl1 and Pl2, with not necessarily the exact pointers  *
*               IPVertexStruct of V1 and V2.                                     *
*                                                                            *
* RETURN VALUE:                                                              *
*   void                                                                     *
*****************************************************************************/
static void ProcessVertexImportance(IPVertexStruct *V1,
                                    IPVertexStruct *V2,
                                    IPPolygonStruct *Pl1,
                                    IPPolygonStruct *Pl2)
{
    if (!IRIT_PT_APX_EQ_EPS(V1 -> Coord, GlblVrtxImportance -> Coord,
                            IRIT_EPS) &&
        !IRIT_PT_APX_EQ_EPS(V2 -> Coord, GlblVrtxImportance -> Coord,
                            IRIT_EPS))
        fprintf(stderr, "Edge does not match the given vertex - adj error!\n");

    if (Pl1 != NULL && Pl2 != NULL) {
        GlblVrtxImportanceCount++;
        GlblVrtxImportanceVal += acos(IRIT_DOT_PROD(Pl1 -> Plane,
                                                    Pl2 -> Plane));
    }
}

/*****************************************************************************
* DESCRIPTION:                                                               *
*   Computes importance for each vertex in the given polygonal mesh.         *
*                                                                            *
* PARAMETERS:                                                                *
*   PObj:         Polygonal model to process for importance.                     *
*                                                                            *
* RETURN VALUE:                                                              *
*   void                                                                     *
*****************************************************************************/
static void GenPolyImportance(IPObjectStruct *PObj)
{
    VoidPtr
        PAdj = GMPolyAdjacncyGen(PObj, IRIT_EPS);
    IPPolygonStruct *Pl;

    for (Pl = PObj -> U.Pl; Pl != NULL; Pl = Pl -> Pnext) {
        int PlImportanceCount = 0;
        IrtRType
            PlImportanceVal = 0.0;
        IPVertexStruct *V;

        for (V = Pl -> PVertex; V != NULL; V = V -> Pnext) {
            GlblVrtxImportanceVal = 0.0;
            GlblVrtxImportanceCount = 0;
            GlblVrtxImportance = V;

            GMPolyAdjacncyVertex(V, PAdj, ProcessVertexImportance);

            if (GlblVrtxImportanceCount > 0) {
                GlblVrtxImportanceVal /= (GlblVrtxImportanceCount * M_PI);

                AttrSetRealAttrib(&V -> Attr, "Imprt", GlblVrtxImportanceVal);
            }
            else
                fprintf(stderr, "Failed to compute Importance for vertex\n");
        }
    }

    GMPolyAdjacncyFree(PAdj);
}
\end{verbatim}


\subsection{Least sqaures curve fitting and process communication (lst\_sqrs.c)}

This example creates a Bspline curve that least squares approximates
a given set of data points. In the default values, a quadratic Bspline
curve of 10 control points least sqaures approximates a set of 100
three dimensional points. Also shown is a polyline of the original
(100) points. Arbitrary number of curves will be displayed every, one
for every <return> keystroke until Q<return> is typed.

\begin{verbatim}
/*****************************************************************************
* Least squares fit a curve to random points.                                     *
* We also see how to fork out a display device and communicate with it.      *
******************************************************************************
* (C) Gershon Elber, Technion, Israel Institute of Technology                *
******************************************************************************
* Written by:  Gershon Elber                                Ver 1.0, June 1995   *
*****************************************************************************/

#include "irit_sm.h"
#include "iritprsr.h"
#include "allocate.h"
#include "attribut.h"
#include "cagd_lib.h"
#include "geom_lib.h"
#include "grap_lib.h"
#include "misc_lib.h"

static char *CtrlStr =
    "Lst_Sqrs n%-#Pts!d d%-Degree!d f%-DOF!d p%-PrgmName!s h%-";

void main(int argc, char **argv)
{
    int i, Error, PrgmIO,
        NumOfPoints = 100,
        NumOfPtsFlag = FALSE,
        Degree = 3,
        DegreeFlag = FALSE,
        NumOfDOF = 10,
        NumOfDOFFlag = FALSE,
        PrgmFlag = FALSE,
        HelpFlag = FALSE;
    char *Err,
        *Program = getenv("IRIT_DISPLAY");

#ifdef __WINNT__
    if (Program == NULL)
        Program = "wntgdrvs -s-";
#endif /* __WINNT__ */
#ifdef __UNIX__
    if (Program == NULL)
        Program = "x11drvs -s-";
#endif /* __UNIX__ */

    if ((Error = GAGetArgs(argc, argv, CtrlStr,
                           &NumOfPtsFlag, &NumOfPoints,
                           &DegreeFlag, &Degree,
                           &NumOfDOFFlag, &NumOfDOF,
                           &PrgmFlag, &Program,
                           &HelpFlag)) != 0) {
        GAPrintErrMsg(Error);
        GAPrintHowTo(CtrlStr);
        exit(1);
    }

    if (HelpFlag) {
        GAPrintHowTo(CtrlStr);
        exit(0);
    }

    IPSocSrvrInit();            /* Initialize the listen socket for clients. */

    if ((PrgmIO = IPSocExecAndConnect(Program,
                                      getenv("IRIT_BIN_IPC") != NULL)) >= 0) {
        char Line[IRIT_LINE_LEN];
        IPObjectStruct
            *PClrObj = IPGenStrObject("command_", "clear", NULL);

        do {
            CagdPtStruct
                *PtList = NULL;
            IPPolygonStruct
                *PPoly = IPAllocPolygon(0, NULL, NULL);
            CagdCrvStruct *Crv;
            IPObjectStruct *PCrvObj, *PPolyObj;

            for (i = 0; i < NumOfPoints; i++) {
                int j;
                IPVertexStruct *V;
                CagdPtStruct
                    *Pt = CagdPtNew();

                if (i == 0) {
                    for (j = 0; j < 3; j++)
                        Pt -> Pt[j] = IritRandom(-1.0, 1.0);
                }
                else {
                    for (j = 0; j < 3; j++)
                        Pt -> Pt[j] = PtList -> Pt[j] + IritRandom(-0.1, 0.1);
                }

                V = IPAllocVertex(0, NULL, PPoly -> PVertex);
                for (j = 0; j < 3; j++)
                    V -> Coord[j] = Pt -> Pt[j];
                PPoly -> PVertex = V;

                IRIT_LIST_PUSH(Pt, PtList);
            }

            Crv = BspCrvInterpPts(PtList, Degree + 1,
                                  NumOfDOF, CAGD_UNIFORM_PARAM, FALSE);
            CagdPtFreeList(PtList);

            CagdCrvWriteToFile3(Crv, stdout, 0, "This is from LstSqrs", &Err);

            /* Generate objects out of the geometry and set proper attrs. */
            PCrvObj = IPGenCRVObject(Crv);
            AttrSetObjectColor(PCrvObj, IG_IRIT_CYAN);

            PPolyObj = IPGenPOLYObject(PPoly);
            IP_SET_POLYLINE_OBJ(PPolyObj);
            AttrSetObjectColor(PPolyObj, IG_IRIT_YELLOW);

            /* Clear old data and display our curve and data. */
            IPSocWriteOneObject(PrgmIO, PClrObj);
            IPSocWriteOneObject(PrgmIO, PCrvObj);
            IPSocWriteOneObject(PrgmIO, PPolyObj);

            IPFreeObject(PCrvObj);
            IPFreeObject(PPolyObj);

            gets(Line);
        }
        while (Line[0] != 'q' && Line[0] != 'Q');

        IPSocDisConnectAndKill(TRUE, PrgmIO);
    }

    exit(0);
}
\end{verbatim}


\subsection{Multivariate Solver (msolve.c)}

This example demonstates the use of the multivariate solver from the
mvar\_lib library.  Input is a set of n polynomial equations in m
variables and dumped out are the solutions to these equations.

\input{cexample/msolve.tex}

\subsection{Compute Area of a Polygonal Model (polyarea.c)}

Here is a simple program to compute the total area of all polygons in
the given data file. The program expects one argument on the command
line which is the name of the file to read, and it prints out one
line with the total computed area.

\input{cexample/polyarea.tex}

\subsection{Converts a Freeform Surface into Polygons (polygons.c)}

This true filter reads a single surface from stdin and dumps out a
polygonal approximation of it to stdout. They are several parameters
that controls the way a surface is approximated into polygons and in
this simple filter they are being held fixed in a set of integer
variables.

\begin{verbatim}
/*****************************************************************************
* Read a surface and dump out a tesselated version of it.                     *
******************************************************************************
* (C) Gershon Elber, Technion, Israel Institute of Technology                *
******************************************************************************
* Written by:  Gershon Elber                                Ver 1.0, June 1995   *
*****************************************************************************/

#include "irit_sm.h"
#include "cagd_lib.h"
#include "iritprsr.h"
#include "ip_cnvrt.h"
#include "allocate.h"

void main(int argc, char **argv)
{
    int Handler,
        FourPerFlat = TRUE, /* Settable parameters of IritSurface2Polygons. */
        FineNess = 20,
        ComputeUV = FALSE,
        ComputeNrml = FALSE,
        Optimal = FALSE;

    if ((Handler = IPOpenDataFile("-", TRUE, TRUE)) >= 0) {
        IPObjectStruct
            *PObj = IPGetObjects(Handler);

        /* Done with file - close it. */
        IPCloseStream(Handler, TRUE);

        /* Process the surface into polygons. */
        if (IP_IS_SRF_OBJ(PObj)) {
            IPPolygonStruct
                *PPoly = IPSurface2Polygons(PObj -> U.Srfs, FourPerFlat,
                                            FineNess, ComputeUV,
                                            ComputeNrml, Optimal);
            IPObjectStruct
                *PObjPoly = IPGenPOLYObject(PPoly);

            IPStdoutObject(PObjPoly, FALSE);

            IPFreeObject(PObjPoly);
        }
        else
            fprintf(stderr, "Read object is not a surface.\n");

        IPFreeObject(PObj);
    }
    else {
        fprintf(stderr, "Failed to read from stdin\n");
        exit(1);
    }

    exit(0);
}
\end{verbatim}


\subsection{Linear Transformations' Filter (transfrm.c)}

This little more complex program transfroms all the geometry in the
read data which can be any number of files according to the specified
transformations on the command line. The command line is parsed via
{\bf GAGetArgs} and its associated functions. The transformation
matrix is then computed with the aid of the matrix package and applied
to the read geometry at once.

\input{cexample/transfrm.tex}

Here is the result of running 'transfrm -h':
\begin{verbatim}
This is Transform...
Usage: Transfrm [-x Degs] [-y Degs] [-z Degs] [-t X Y Z] [-s Scale] [-h] DFiles
\end{verbatim}

When you are considering the usefulness of this tool remember that the
transformations are applied to the geometry in an internal order which is
different from the command line order. That is,
\begin{verbatim}
     transfrm -x 30 -y 30 geometry.dat > tgeometry.dat
\end{verbatim}
will compute the exact same transfrom as,
\begin{verbatim}
     transfrm -y 30 -x 30 geometry.dat > tgeometry.dat
\end{verbatim}
This, while rotation is not a commutative operation.
Nonetheless, you may split the operations. That is:
\begin{verbatim}
     transfrm -y 30 geometry.dat | transfrm -x 30 - > tgeometry.dat
\end{verbatim}
or
\begin{verbatim}
     transfrm -x 30 geometry.dat | transfrm -y 30 - > tgeometry.dat
\end{verbatim}
will do exactly what one expects.

%%%%%%%%%%%%%%%%%%%%%%%%%%%%%%%%%%%%%%%%%%%%%%%%%%%%%%%%%%%%%%%%%%%%%%%%%%%%%
\twocolumn
% 
% prog_man.tex - The IRIT programmer's manual - main module.
% 
% Author:	Gershon Elber
% 		Computer Science Dept.
% 		Technion, IIT
%

%\documentstyle[10pt]{book}
\documentstyle{book}

\setlength{\oddsidemargin}{0.0in}
\setlength{\evensidemargin}{0.0in}
\setlength{\topmargin}{-0.2in}
\setlength{\textwidth}{6.5in}
\setlength{\textheight}{9.0in}

\makeindex

\begin{document}

\small
\setlength{\baselineskip}{0.85\baselineskip}

\tableofcontents

%%%%%%%%%%%%%%%%%%%%%%%%%%%%%%%%%%%%%%%%%%%%%%%%%%%%%%%%%%%%%%%%%%%%%%%%%%%%%
\chapter{Introduction}

This manual describes the different libraries of the IRIT solid
modeling environment. Quite a few libraries can be found to manipulate
geometry in general, freeform curves and surfaces, symbolic
computation, trimmed surfaces, triangular patches, freeform trivariate
functions, Boolean operations, input output data file parsing, and
miscelleneous.

All interface to the libraries should be made via the appropriate header
files that can be found in the include subdirectory. Most libraries
have a single header file that is named the same as the library.
Functions and constants that are visible to the users of the libraries
are prefixed with a unique prefix, usually derived from the library
name itself. External definitions that start with an underscore should
not be used, even if found in header files.

The header file {\bf include/irit\_sm.h} must be sourced by every
source file in the solid modeller. In most cases, this file is sourced
indirectly via local header files.

The following libraries are avaliable in IRIT:
\begin{center}
\begin{tabular}{|l|l|} \hline
    Name of Library & Tasks \\ \hline
    bool & Boolean operations on polygonal models. \\
    cagd & Low level freeform curves and surfaces. \\
    geom & General geometry functions. \\
    grap & General graphics/display functions. \\
    mdl  & Model's processing functions. \\
    misc & Memory allocation, configuration files, attributes, etc. \\
    mvar & Multi variate functions. \\
    prsr & Input and output for file/sockets of objects of IRIT. \\
    rndr & Scan conversion rendering functions. \\
    symb & Symbolic manipulation of curves and surfaces. \\
    trim & Trimmed surfaces support. \\
    triv & Freeform trivariate functions. \\
    trng & Triangular patches support. \\
    user & General high level user inteface functions. \\
    xtra & Public domain code that is not part of IRIT. \\ \hline
\end{tabular}
\end{center}

%%%%%%%%%%%%%%%%%%%%%%%%%%%%%%%%%%%%%%%%%%%%%%%%%%%%%%%%%%%%%%%%%%%%%%%%%%%%%

\subsection{Tools and Programs}

The IRIT package includes several complete programs such as poly3d-h
(hidden line removal), irender (scan conversion tool), and irit2ps (a
filter to Postscript).  Somewhat different than most other programs is
the kernel interpreter, also called irit.  The irit program is nothing
more than an interpreter (written in C) that invokes numerous
functions in the several libraries provided.  In order to add a new
function to the irit interpreter, the following sequence of operations
must be followed:

\begin{enumerate}
\item Write a C function that accepts only one or more of the following
     type of parameters.  All parameters, including the IrtRType, {\em must}
     be transferred by address:
    \begin{itemize}
	\item IrtRType         (see irit\_sm.h).
	\item IrtVecType       (see irit\_sm.h).
	\item IrtPtType        (see irit\_sm.h).
	\item CagdCtlPtStruct  (see cagd\_lib.h).
	\item IrtPlnType       (see irit\_sm.h).
	\item StringType       (char *).
	\item IPObjectStruct   (see iritprsr.h).  This includes all object
		types in IPObjectStruct other than the above.
    \end{itemize}
\item The written C function can return one of:
    \begin{itemize}
	\item IrtRType by value.
	\item IPObjectStruct by address.
	\item Nothing (a procedure).
    \end{itemize}
\item According to the returned type by the new C function, go to file
	{\bf inptprsl.h}, and add a new enum NEW\_FUNC for the new function in
	enum RealValueFuncType (for IrtRType returned value), in
	enum ObjValueFuncType (for IPObjectStruct * returned value), or in
	enum GenValueFuncType (for no returned value).
\item In {\bf inptevl1.c}, add one new line in one of NumFuncTable,
	ObjFuncTable, or GenFuncTable (depends upon the return value).
	The line will have the form of:
	\begin{verbatim}
            { ``FuncName'', NEW_FUNC, CFunctionName,  N, { Param1Type,
                                              Param2Type, ... ParamNType } }
	\end{verbatim}
	for procedures with no return values, and of the form of:
	\begin{verbatim}
            { ``FuncName'', NEW_FUNC, CFunctionName,  N, { Param1Type,
                                    Param2Type, ... ParamNType }, RetValType }
	\end{verbatim}
	otherwise. \verb-N- is the number of parameters,
	\verb-`FuncName''- is the (unique) name that will be used in
	the interpreter, CFunctionName is the name of the C function you
	have written. This C function must be declared in one of the header
	files that inptevl1.c includes. ParamIType, for I between 1 and N,
	and RetValType are of type IritExprType (see inptprsl.h).
\item Thats it!
\end{enumerate}

For example, to add the C function declared as
\begin{verbatim}
    IPObjectStruct *PolyhedraMoments(IPObjectStruct *IPObjectStruct,
                                     IrtRType *m);
\end{verbatim}
one has to add the following line to ObjFuncTable in iritevl1.c
\begin{verbatim}
    { ``PMOMENT'', PMOMENT, { POLY_EXPR, REAL_EXPR }, VECTOR_EXPR },
\end{verbatim}
where PMOMENT needs to be added to ObjValueFuncType in iritprsl.h.

While all objects in the interpreted space are of type IPObjectStruct,
the functions' interface unfolds many types to simplify matter.
IrtRType, IrtVecType, Strings, etc. are all extracted from the given
(IPObjectStruct) parameters and passed directly by address.  For
returned values, only numeric real data is allowed where everything
else must be returned wrapped in an IPObjectStruct.

%%%%%%%%%%%%%%%%%%%%%%%%%%%%%%%%%%%%%%%%%%%%%%%%%%%%%%%%%%%%%%%%%%%%%%%%%%%%%

The following chapters reference the different functions of the different
libraries. Chapter~\ref{chap-examples} provides several examples of
writing C code using the IRIT libraries.

%%%%%%%%%%%%%%%%%%%%%%%%%%%%%%%%%%%%%%%%%%%%%%%%%%%%%%%%%%%%%%%%%%%%%%%%%%%%%
\chapter{Boolean Library, bool\_lib}

\section{General Information}

On of the crucial operation in any solid modeling environment is the
ability to perform Boolean operations among different geometric
objects.  The interface of the library is defined in {\em
include/bool\_lib.h}. This library supports only Boolean operations of
polygonal objects. The Boolean operations of OR, AND, SUBtract,
NEGate, CUT, and MERGE are supported via the {\bf BoolOperType}
typedef:
\begin{center}
\begin{tabular}{||c|c||} \hline 
    BoolOperType & Meaning \\ \hline
    BOOL\_OPER\_OR & Union of two geometrical objects \\
    BOOL\_OPER\_AND & Intersection of two geometrical objects \\
    BOOL\_OPER\_SUB & The difference of two geometrical objects \\
    BOOL\_OPER\_NEG & Unary Inside-out of one geometrical object \\
    BOOL\_OPER\_CUT & Boundary of one object outside the other \\
    BOOL\_OPER\_MERGE & Simple merge without any computation \\ \hline
\end{tabular}
\end{center}

The {\bf BoolOperType} typedef is used in two dimensional Boolean operations
invoked via {\bf Boolean2D}. Three dimensional Boolean operations are
invoked via {\bf BooleanXXX} where XXX is one of OR, AND, SUB, NEG, CUT
or MERGE, denoting the same as in the table above.

In addition several state functions are available to control the way
the Boolean operations are conducted. Functions to enable the dump of
(only) the intersection curves, to handle coplanar polygons, and to
set the axis along which the sweep is performed are provided.

All globals in this library have a profix of {\bf Bool}.

%%%%%%%%%%%%%%%%%%%%%%%%%%%%%%%%%

\section{Algorithmic Hints Behind the Boolean Operations}

Let ${\cal M}_i,~i = 1,2$ be two polygonal models to compute a Boolean
operation between and let $P_m$ denote the $m$'th polygon of ${\cal M}_i$.
Denote by $InterList( P_m ),~P_m \in {\cal M}_i$, the list of line
segments representing the intersections of polygon $P_m$ with polygons
in ${\cal M}_j$.  A naive first, intersection, step in a Boolean
operation between two polyhedra models, ${\cal M}_1$ and ${\cal M}_2$,
will consist of the following stages:
\begin{bf}
\begin{tt}
\begin{tabbing}
\hspace{2em}\=\hspace{2em}\=\hspace{2em}\=\hspace{2em}\=\hspace{2em}\=\hspace{2em}\=\kill
\> For each polygon $P_m$ in ${\cal M}_1$ \\
\> begin \\
\> \> $InterList( P_m )$ $\Leftarrow$ $\emptyset$ \\
\> end \\
\> For each polygon $P_n$ in ${\cal M}_2$ \\
\> begin \\
\> \> $InterList( P_n )$ $\Leftarrow$ $\emptyset$ \\
\> end \\
\\
\> For each polygon $P_m$ in ${\cal M}_1$ \\
\> begin \\
\> \> For each polygon $P_n$ in ${\cal M}_2$ \\
\> \> begin \\
\> \> \> If ( $P_m \cap P_n$ ) \\
\> \> \> begin \\
\> \> \> \> ${\cal L}$ $\Leftarrow$ $P_m \cap P_n$; \\
\> \> \> \> $InterList( P_m )$ $\Leftarrow$ $InterList( P_m )$ $\cup$ ${\cal L}$; \\
\> \> \> \> $InterList( P_n )$ $\Leftarrow$ $InterList( P_n )$ $\cup$ ${\cal L}$; \\
\> \> \> end \\
\> \> end \\
\> end
\end{tabbing}
\end{tt}
\end{bf}

At the end of the first, intersection stage, each polygon, $P_m$ in
either ${\cal M}_1$ (or $P_n \in {\cal M}_2$) contains in its
$InterList( P_m )$ the set of line segments of the intersection(s) of
$P_m$ with all polygons in the other model.  In order to figure out
orientation, each intersection line segment also contains an
orientation hint as to the inside direction of the two polygons that
intersect at that line segment.  This list of intersection line
segments can now be processed:

\begin{bf}
\begin{tt}
\begin{tabbing}
\hspace{2em}\=\hspace{2em}\=\hspace{2em}\=\hspace{2em}\=\hspace{2em}\=\kill
\> For each polygon $P_m$ in ${\cal M}_1$ \\
\> begin \\
\> \> $Polys$ $\Leftarrow$ $InterList( P_m )$ sorted into polylines; \\
\> \> $OpenList( P_m )$ $\Leftarrow$ Open polylines of $Polys$; \\
\> \> $ClosedList( P_m )$ $\Leftarrow$ Closed polylines of $Polys$; \\
\> end \\
\> For each polygon $P_n$ in ${\cal M}_2$ \\
\> begin \\
\> \> $Polys$ $\Leftarrow$ $InterList( P_n )$ sorted into polylines; \\
\> \> $OpenList( P_n )$ $\Leftarrow$ Open polylines of $Polys$; \\
\> \> $ClosedList( P_n )$ $\Leftarrow$ Closed polylines of $Polys$; \\
\> end \\
\end{tabbing}
\end{tt}
\end{bf}

Open polylines, $OpenList( P_m )$, are intersection polylines that are
open.  These polylines starts on the boundary of polygon $P_m$ and
also ends on the boundary, at a different boundary location though.
In contrast, the set of closed polylines, $ClosedList( P_n )$, contains
closed loops that are completely contained in $P_n$.

The polygons are now ready for {\em trimming}.  Depending upon the Boolean
operation that is requested, the inside or the outside of ${\cal M}_1$ or
${\cal M}_2$ is selected for the output result.  Denote by
${\cal M}_i \ Out \ {\cal M}_j$ (${\cal M}_i \ In \ {\cal M}_j$) the region of
${\cal M}_i$ that is outside (inside) ${\cal M}_j$. Then,

\bigskip 
\begin{center}
\begin{tabular}{||c|c||} \hline
Boolean Operation & implementation as \\ \hline \hline
${\cal M}_1 \cup {\cal M}_1$ & $\{ {\cal M}_1\ Out\ {\cal M}_2 \} \cup
			     \{ {\cal M}_2\ Out\ {\cal M}_1 \}$ \\
${\cal M}_1 \cap {\cal M}_1$ & $\{ {\cal M}_1\ In\ {\cal M}_2 \} \cup
			     \{ {\cal M}_2\ In\ {\cal M}_1 \}$ \\
${\cal M}_1 - {\cal M}_1$ & $\{ {\cal M}_1\ Out\ {\cal M}_2 \} \cup
			     Inverse\{ {\cal M}_2\ In\ {\cal M}_1 \}$ \\ \hline
\end{tabular}
\end{center}
\bigskip 

Note the $Inverse$ operation that reverses the orientation inside
out.

At this point, all Boolean operations are translated into two complementary
type of operations, $In$ and $Out$. The computation of
${\cal M}_i \ Out \ {\cal M}_j$ is conducted by tracing the outside regions
of the polygons of ${\cal M}_i$ that {\em intersect} ${\cal M}_j$ and
{\em propagating} this inside--outside information to the non intersecting
polygons:
\begin{bf}
\begin{tt}
\begin{tabbing}
\hspace{2em}\=\hspace{2em}\=\hspace{2em}\=\hspace{2em}\=\hspace{2em}\=\kill
\> For each polygon $P_m$ in ${\cal M}_1$ with non empty $OpenList( P_m )$ \\
\> \mbox{\hspace{2.425in}} or non empty $ClosedList( P_m )$ \\
\> begin \\
\> \> $P_m^{in}$ $\Leftarrow$ Inside region of $P_m$; \\
\> \> $P_m^{out}$ $\Leftarrow$ Outside region of $P_m$; \\
\\
\> \> For each {\em original} edge, $e \in P_m^{in}$ \\
\> \> begin \\
\> \> \> Push $e$ onto $InsideStack$; \\
\> \> end \\
\> \> For each {\em original} edge, $e \in P_m^{out}$ \\
\> \> begin \\
\> \> \> Push $e$ onto $OutSideStack$; \\
\> \> end \\
\> end \\
\\
\> While $InsideStack$ not empty\\
\> begin \\
\> \> $e$ $\Leftarrow$ edge poped from top of $InsideStack$; \\
\> \> $P_m$, $P_l$ $\Leftarrow$ two polygons sharing $e$ in ${\cal M}_1$; \\
\\
\> \> If ( $P_m$ has empty $OpenList( P_m )$ and empty $ClosedList( P_m )$, \\
\> \> \mbox{\hspace{3in}} and $P_m$ is not classified yet ) \\
\> \> begin \\
\> \> \> Classify $P_m$ as inside; \\
\> \> \> push all edges of $P_m$, but $e$, onto $InsideStack$; \\
\> \> end \\
\> \> If ( $P_l$ has empty $OpenList( P_l )$ and empty $ClosedList( P_l )$, \\
\> \> \mbox{\hspace{3in}} and $P_l$ is not classified yet ) \\
\> \> begin \\
\> \> \> Classify $P_l$ as inside; \\
\> \> \> push all edges of $P_l$, but $e$, onto $InsideStack$; \\
\> \> end \\
\> end \\
\\
\> While $OutsideStack$ not empty\\
\> begin \\
\> \> $e$ $\Leftarrow$ edge poped from top of $outsideStack$; \\
\> \> $P_m$, $P_l$ $\Leftarrow$ two polygons sharing $e$ in ${\cal M}_1$; \\
\\
\> \> If ( $P_m$ has empty $OpenList( P_m )$ and empty $ClosedList( P_m )$, \\
\> \> \mbox{\hspace{3in}} and $P_m$ is not classified yet ) \\
\> \> begin \\
\> \> \> Classify $P_m$ as outside; \\
\> \> \> push all edges of $P_m$, but $e$, onto $OutsideStack$; \\
\> \> end \\
\> \> If ( $P_l$ has empty $OpenList( P_l )$ and empty $ClosedList( P_l )$, \\
\> \> \mbox{\hspace{3in}} and $P_l$ is not classified yet ) \\
\> \> begin \\
\> \> \> Classify $P_l$ as outside; \\
\> \> \> push all edges of $P_l$, but $e$, onto $OutsideStack$; \\
\> \> end \\
\> end \\
\end{tabbing}
\end{tt}
\end{bf}

At the end of this process, $M_1$ is completely split into two sets of
polygons, classified as either inside $M_2$ or outside $M_2$.  An
identical spliting process can be applied to ${\cal M}_2$.  The proper
inside/outside lists of ${\cal M}_1$ can now be combined with the
proper list of classified polygons of ${\cal M}_2$ to yield the proper
result of the requested Boolean operation.

While this completes the Boolean operation, vast room for improvements
can be found, with some improvements implemented in IRIT.  For
example, the first step of the intersection computation necessitates
an $O(n^2)$ polygon--polygon intersection tests, where $n$ is in the
order of the number of polygons in ${\cal M}_i$.  Clearly, one can
hope for a better time complexity.  Bounding boxes, hierarchical
bounding boxes, or three dimensional space sweep techniques can all be
applied to reduce the overhead invested in these $O(n^2)$ tests.
Efficient sorting of line segments within a polygon, when forming the
open and closed loops is another optimization consideration.

\section{Library Functions}
\input{prog_man/bool_lib.tex}

%%%%%%%%%%%%%%%%%%%%%%%%%%%%%%%%%%%%%%%%%%%%%%%%%%%%%%%%%%%%%%%%%%%%%%%%%%%%%
\chapter{CAGD Library, cagd\_lib}

\section{General Information}

This library provides a rich set of function to create, convert,
display and process freeform Bezier and NURBs curves and surfaces. The
interface of the library is defined in {\em include/cagd\_lib.h}. This
library mainly supports low level freeform curve and surface
operations.  Supported are curves and surfaces from scalars to five
dimensions as E1/P1 to E5/P5 using the {\bf CagdPointType}. Pi is a
rational (projective) version of Ei, with an additional W coefficient.
Polynomial in the power basis have some very limited support as well.
Different data structures to hold UV parameter values, control points,
vectors, planes, bounding boxes, polylines and polygons are defined as
well as the data strcutures to hold the curves and surfaces
themselves,
\begin{verbatim}
typedef struct CagdCrvStruct {
    struct CagdCrvStruct *Pnext;
    struct IPAttributeStruct *Attr;
    CagdGeomType GType;
    CagdPointType PType;
    int Length;            /* Number of control points (== order in Bezier). */
    int Order;	    /* Order of curve (only for Bspline, ignored in Bezier). */
    CagdBType Periodic;			   /* Valid only for Bspline curves. */
    CagdRType *Points[CAGD_MAX_PT_SIZE];     /* Pointer on each axis vector. */
    CagdRType *KnotVector;
} CagdCrvStruct;

typedef struct CagdSrfStruct {
    struct CagdSrfStruct *Pnext;
    struct IPAttributeStruct *Attr;
    CagdGeomType GType;
    CagdPointType PType;
    int ULength, VLength;	 /* Mesh size in the tensor product surface. */
    int UOrder, VOrder;   /* Order in tensor product surface (Bspline only). */
    CagdBType UPeriodic, VPeriodic;      /* Valid only for Bspline surfaces. */
    CagdRType *Points[CAGD_MAX_PT_SIZE];     /* Pointer on each axis vector. */
    CagdRType *UKnotVector, *VKnotVector;
} CagdSrfStruct;
\end{verbatim}

Curves and surfaces have a geometric type {\bf GType} to prescribe the
type of entity (such as CAGD\_SBEZIER\_TYPE for Bezier surface) and a
point type {\bf PType} to prescribe the point type of the entity (such
as CAGD\_PT\_E3\_TYPE for three dimensional Euclidean control points).
{\bf Length} and {\bf Order} slots are used to hold the number of
control points in the mesh and or control polygon and the order(s) of
the basis functions. {\bf Periodic} flag(s) are used to denote
periodic end conditions. In addition, {\bf KnotVector} slot(s) are
used if the entity exploits Bspline basis functions, or NULL
otherwise.

The control polygon and/or mesh itself is organized in the {\bf
Points} slot as a vector of size {\bf CAGD\_MAX\_PT\_SIZE} of vectors
of {\bf CagdRType}s. For surfaces, the mesh is ordered U first and the
macros of {\bf CAGD\_NEXT\_U} {\bf CAGD\_NEXT\_V}, and {\bf
CAGD\_MESH\_UV} can be used to determine indices in the mesh.

All structures in the cagd library can be allocated using New
constrcutures (i.e. {\bf CagdUVNew} or {\bf CagdCrfNew}, freed using
Free destructores (i.e. {\bf CagdSrfFree} or {CagdBBoxFree}, linked
list free using FreeList destructores (i.e. {\bf
CagdPolylineFreeList}), and copied using copy constructores {i.e. {\bf
CagdPtCopy} or {\bf CagdCtlPtCopyList}).

This library has its own error handler, which by default prints an
error message and exit the program called {\bf CagdFatalError}.

Most globals in this library have a prefix of {\bf Cagd} for general
cagd routines. Prefix of {\bf Bzr} is used for Bezier routines, prefix
of {\bf Bsp} for Bspline specific routines, prefix of {\bf Cnvrt} for
conversion routines, and {\bf Afd} for adaptive forward differencing
routines.

\section{Library Functions}
\input{prog_man/cagd_lib.tex}

%%%%%%%%%%%%%%%%%%%%%%%%%%%%%%%%%%%%%%%%%%%%%%%%%%%%%%%%%%%%%%%%%%%%%%%%%%%%%
\chapter{Geometry Library, geom\_lib}

\section{General Information}

This library handles general computational geometry algorithms and
geometric queries such as a distance between two lines, bounding
boxes, convexity and convex hull of polygons, polygonal constructors
of primitives (cylinders, spheres, etc.), basic scan conversion
routines, etc.

\section{Library Functions}
\input{prog_man/geom_lib.tex}

%%%%%%%%%%%%%%%%%%%%%%%%%%%%%%%%%%%%%%%%%%%%%%%%%%%%%%%%%%%%%%%%%%%%%%%%%%%%%
\chapter{Graphics Library, grap\_lib}

\section{General Information}

This library handles general drawing and display algorithms, including
tesselation of all geometric objects, such as curves and surfaces, into
displayable primitives, i.e. polygons and polylines.

\section{Library Functions}
\input{prog_man/grap_lib.tex}

%%%%%%%%%%%%%%%%%%%%%%%%%%%%%%%%%%%%%%%%%%%%%%%%%%%%%%%%%%%%%%%%%%%%%%%%%%%%%
\chapter{Model Library, mdl\_lib}

\section{General Information}

This library provides the necessary tools to represent and process
models.  Models are sets of trimmed surfaces forming a closed
2-manifold shell.  Models are typically the result of Boolean operations
over freeform (trimmed) NURBs geometry but can also be constructed
directly or via other schemes.

\section{Library Functions}
\input{prog_man/mdl_lib.tex}

%%%%%%%%%%%%%%%%%%%%%%%%%%%%%%%%%%%%%%%%%%%%%%%%%%%%%%%%%%%%%%%%%%%%%%%%%%%%%
\chapter{Miscelleneous Library, misc\_lib}

\section{General Information}

This library holds general miscelleneous functions such as reading
configuration files, low level homogeneous matrices computation, low
level attributes and general machine specific low level routines.
Several header files can be found for this library:
\begin{center}
\begin{tabular}{||c|c||} \hline
    Header (include/*.h) & Functionality \\ \hline
    config.h   & Manipulation of configuration files (*.cfg files) \\
    dist\_pts.h & Enegry based distribution of points. \\
    gen\_mat.h  & Homogeneous matrices manipulation \\
    getarg.h   & Command line parsing for application tools \\
    imalloc.h  & Low level dynamic memory functions for IRIT \\
    miscattr.h & Low level attribute related functions \\
    priorque.h & An implementation of a priority queue \\
    xgeneral.h & Low level, machine specific, routines \\ \hline
\end{tabular}
\end{center}

\section{Library Functions}
\input{prog_man/misc_lib.tex}

%%%%%%%%%%%%%%%%%%%%%%%%%%%%%%%%%%%%%%%%%%%%%%%%%%%%%%%%%%%%%%%%%%%%%%%%%%%%%
\chapter{Multi variate functions, mvar\_lib}

\section{General Information}

This library holds functions to handle functions of arbitrary number
of variables. In this context curves (univariate), surfaces
(bivariate) and trivariates are special cases. This library provides a
rich set of functions to manipulate freeform Bezier and/or NURBs
multivariates. This library heavily depends on the cagd an symb
libraries. Functions are provided to create, copy, and destruct
multivariates, to extract isoparametric lower degree multivariates, to
evaluate, refine and subdivide, to read and write multivariates, to
differentiate, degree raise, make compatible and convert back and
forth to/from curves, surfaces, and trivariates.

A multivariate has m orders, m Length prescriptions and,
possibly, m knot vectors (if Bspline).  In addition it contains
an m dimensional volume of control points,
\begin{verbatim}
typedef struct MvarMVStruct {
    struct MvarMVStruct *Pnext;
    struct IPAttributeStruct *Attr;
    MvarGeomType GType;
    CagdPointType PType;
    int Dim;		      /* Number of dimensions in this multi variate. */
    int *Lengths;               /* Dimensions of mesh size in multi-variate. */
    int *SubSpaces;	   /* SubSpaces[i] = Prod(i = 0, i-1) of Lengths[i]. */
    int *Orders;                /* Orders of multi varariate (Bspline only). */
    CagdBType *Periodic;            /* Periodicity - valid only for Bspline. */
    CagdRType *Points[CAGD_MAX_PT_SIZE];     /* Pointer on each axis vector. */
    CagdRType **KnotVectors;
} MvarMVStruct;
\end{verbatim}

The interface of the library is defined in {\em include/mvar\_lib.h}. 

This library has its own error handler, which by default prints an
error message and exit the program called {\bf MvarFatalError}.

All globals in this library have a prefix of {\bf Mvar}.

\section{Library Functions}
\input{prog_man/mvar_lib.tex}

%%%%%%%%%%%%%%%%%%%%%%%%%%%%%%%%%%%%%%%%%%%%%%%%%%%%%%%%%%%%%%%%%%%%%%%%%%%%%
\chapter{Prsr Library, prsr\_lib}

\section{General Information}

This library provides the data file interface for IRIT. Functions are
provided to read and write data files, both compressed (on unix only,
using {\em compress}), and uncompressed, in binary and/or ascii text
modes. This library is also used to exchange data between the IRIT
server and the display devices' clients.  Several header files can be
found for this library:
\begin{center}
\begin{tabular}{||c|c||} \hline
    Header (include/*.h) & Functionality \\ \hline
    allocate.h & High level dynamic allocation of objects \\
    attribut.h & High level attributes for objects \\
    ip\_cnvrt.h & Freeform to polygon and polyline high level conversion \\
    iritprsr.h & Main interface to reading and writing data \\
    irit\_soc.h & Socket communication for data exchange \\ \hline
\end{tabular}
\end{center}

\section{Library Functions}
\input{prog_man/prsr_lib.tex}

%%%%%%%%%%%%%%%%%%%%%%%%%%%%%%%%%%%%%%%%%%%%%%%%%%%%%%%%%%%%%%%%%%%%%%%%%%%%%
\chapter{Rendering Library, rndr\_lib}

\section{General Information}

This library provides a powerful full screen scan conversion Z-buffer
tool to process IRIT geometry and convert it into images.  This
library allows one to scan convert any IRIT geometry including
polylines and curves that are converted to skinny polygons on the fly.
The library offers regular scan conversion with flat, Gouraud, and
Phong shading and several antialiasing filters along with advanced
features such as transparency and animation support, and width depth
cueing on polyline/curves rendering.  The library also provide direct
access to the depth Z-buffer as well as a stencil buffer.

\section{Library Functions}
\input{prog_man/rndr_lib.tex}

%%%%%%%%%%%%%%%%%%%%%%%%%%%%%%%%%%%%%%%%%%%%%%%%%%%%%%%%%%%%%%%%%%%%%%%%%%%%%
\chapter{Symbolic Library, symb\_lib}

\section{General Information}

This library provides a rich set of functions to symbolically manipulate
freeform curves and surfaces. This library heavily depends on the cagd
library. Functions are provided to low level add, subtract, and multiply
freeform curves and surfaces, to compute fields such as curvature,
and to extract singular points such as extremums, zeros, and inflections.
High level tools to metamorph curves and surfaces, to compute layout (prisa)
of freeform surfaces, to compute offset approximations of curves and
surfaces, and to compose curves and surfaces are also provided.

The interface of the library is defined in {\em include/symb\_lib.h}. 

This library has its own error handler, which by default prints an
error message and exit the program called {\bf SymbFatalError}.

Globals in this library have a prefix of {\bf Symb} for general
symbolic routines. Prefix of {\bf Bzr} is used for Bezier routines,
and prefix of {\bf Bsp} for Bspline specific routines.

\section{Library Functions}
\input{prog_man/symb_lib.tex}

%%%%%%%%%%%%%%%%%%%%%%%%%%%%%%%%%%%%%%%%%%%%%%%%%%%%%%%%%%%%%%%%%%%%%%%%%%%%%
\chapter{Trimmed surfaces Library, trim\_lib}

\section{General Information}

This library provides a set of functions to manipulate freeform
trimmed Bezier and/or NURBs surfaces. This library heavily depends on
the cagd library. Functions are provided to create, copy, and destruct
trimmed surfaces to extract isoparametric curves, to evaluate, refine
and subdivide, to read and write trimmed surfaces, degree raise, and
approximate using polygonal representations.  A trimming surface is
defined out of a tensor product surface and a set of trimming loops
that trims out portions of the parametric space of the surface,
\begin{verbatim}
typedef struct TrimSrfStruct {
    struct TrimSrfStruct *Pnext;
    IPAttributeStruct *Attr;
    int Tags;
    CagdSrfStruct *Srf;			  /* Surface trimmed by TrimCrvList. */
    TrimCrvStruct *TrimCrvList;		         /* List of trimming curves. */
} TrimSrfStruct;
\end{verbatim}

Each trimming loop consists of a set of trimming curve segments:
\begin{verbatim}
typedef struct TrimCrvStruct {
    struct TrimCrvStruct *Pnext;
    IPAttributeStruct *Attr;
    TrimCrvSegStruct *TrimCrvSegList;    /* List of trimming curve segments. */
} TrimCrvStruct;
\end{verbatim}

Each trimming curve segment contains a representation for the curve in the
UV space of the surface as well as a representation in the Euclidean space,
\begin{verbatim}
typedef struct TrimCrvSegStruct {
    struct TrimCrvSegStruct *Pnext;
    IPAttributeStruct *Attr;
    CagdCrvStruct *UVCrv;    /* Trimming crv segment in srf's param. domain. */
    CagdCrvStruct *EucCrv;       /* Trimming curve as an E3 Euclidean curve. */
} TrimCrvSegStruct;
\end{verbatim}


The interface of the library is defined in {\em include/trim\_lib.h}. 

This library has its own error handler, which by default prints an
error message and exit the program called {\bf TrimFatalError}.

All globals in this library have a prefix of {\bf Trim}.

\section{Library Functions}
\input{prog_man/trim_lib.tex}

%%%%%%%%%%%%%%%%%%%%%%%%%%%%%%%%%%%%%%%%%%%%%%%%%%%%%%%%%%%%%%%%%%%%%%%%%%%%%
\chapter{Trivariate Library, triv\_lib}

\section{General Information}

This library provides a rich set of functions to manipulate freeform
Bezier and/or NURBs trivariate. This library heavily depends on the cagd
library. Functions are provided to create, copy, and destruct trivariates,
to extract isoparametric surfaces, to evaluate, refine and subdivide, to
read and write trivariates, to differentiate, degree raise, make compatible
and approximate iso-surface at iso values using polygonal representations.

A trivariate has three orders, three Length prescriptions and,
possibly, three knot vectors (if Bspline).  In addition it contains
a three dimensional volume of control points,
\begin{verbatim}
typedef struct TrivTVStruct {
    struct TrivTVStruct *Pnext;
    struct IPAttributeStruct *Attr;
    TrivGeomType GType;
    CagdPointType PType;
    int ULength, VLength, WLength;/* Mesh size in tri-variate tensor product.*/
    int UVPlane;	  /* Should equal ULength * VLength for fast access. */
    int UOrder, VOrder, WOrder;       /* Order in trivariate (Bspline only). */
    CagdBType UPeriodic, VPeriodic, WPeriodic;    /* Valid only for Bspline. */
    CagdRType *Points[CAGD_MAX_PT_SIZE];     /* Pointer on each axis vector. */
    CagdRType *UKnotVector, *VKnotVector, *WKnotVector;
} TrivTVStruct;
\end{verbatim}

The interface of the library is defined in {\em include/triv\_lib.h}. 

This library has its own error handler, which by default prints an
error message and exit the program called {\bf TrivFatalError}.

All globals in this library have a prefix of {\bf Triv}.

\section{Library Functions}
\input{prog_man/triv_lib.tex}

%%%%%%%%%%%%%%%%%%%%%%%%%%%%%%%%%%%%%%%%%%%%%%%%%%%%%%%%%%%%%%%%%%%%%%%%%%%%%
\chapter{Triangular Library, trng\_lib}

\section{General Information}

This library provides a subset of functions to manipulate freeform
triangular Bezier and Bspline patches. This library heavily depends on
the cagd library. Functions are provided to create, copy, and destruct
triangular patches, to extract isoparametric curves, to evaluate,
refine and subdivide, to read and write triangular patches, to
differentiate, and approximate using polygonal representations.

A triangular patch has one prescription of Length and one prescription
of Order, the total Order and the length of an edge of the triangle.
The control mesh mesh has Length * (Length + 1) / 2 control points,
\begin{verbatim}
typedef struct TrngTriangSrfStruct {
    struct TrngTriangSrfStruct *Pnext;
    struct IPAttributeStruct *Attr;
    TrngGeomType GType;
    CagdPointType PType;
    int Length;		    /* Mesh size (length of edge of triangular mesh. */
    int Order;		      /* Order of triangular surface (Bspline only). */
    CagdRType *Points[CAGD_MAX_PT_SIZE];     /* Pointer on each axis vector. */
    CagdRType *KnotVector;
} TrngTriangSrfStruct;
\end{verbatim}

The interface of the library is defined in {\em include/trng\_lib.h}. 

This library has its own error handler, which by default prints an
error message and exit the program called {\bf TrngFatalError}.

All globals in this library have a prefix of {\bf Trng}.

\section{Library Functions}
\input{prog_man/trng_lib.tex}

%%%%%%%%%%%%%%%%%%%%%%%%%%%%%%%%%%%%%%%%%%%%%%%%%%%%%%%%%%%%%%%%%%%%%%%%%%%%%
\chapter{User Library, user\_lib}

\section{General Information}

This library includes user interface related geometrical functions such
as ray surface intersection (for mouse click/select operations), etc.

The interface of the library is defined in {\em include/user\_lib.h}. 

\section{Library Functions}
\input{prog_man/user_lib.tex}

%%%%%%%%%%%%%%%%%%%%%%%%%%%%%%%%%%%%%%%%%%%%%%%%%%%%%%%%%%%%%%%%%%%%%%%%%%%%%
\chapter{Extra Library, xtra\_lib}

\section{General Information}

This library is not an official part of IRIT and contains public
domain code that is used by routines in IRIT.

The interface of the library is defined in {\em include/extra\_fn.h}. 

\section{Library Functions}
\input{prog_man/xtra_lib.tex}

%%%%%%%%%%%%%%%%%%%%%%%%%%%%%%%%%%%%%%%%%%%%%%%%%%%%%%%%%%%%%%%%%%%%%%%%%%%%%
\chapter{Programming Examples}
\label{chap-examples}

This chapter describes several simple examples of C programs that
exploits the libraries of IRIT. All external function are defined in
the {\em include} subdirectory of IRIT and one can 'grep' there for the
exact include file that contains a certain function name. All C programs
in all C files should include 'irit\_sm.h' as their first include file
of IRIT, before any other include file of IRIT. Header files are set so
C++ code can compile and link to it without any special treatment.

\section{Setting up the Compilation Environment}

In order to compile programs that uses the libraries of IRIT, a makefile
has to be constructed. Assuming IRIT is installed in {\em /usr/local/irit},
here is a simple makefile that can be used (for a unix environment):

\begin{verbatim}
IRIT_DIR = /usr/local/irit

include $(IRIT_DIR)/makeflag.unx

OBJS	= program.o

program: $(OBJS)
	$(CC) $(CFLAGS) -o program $(OBJS) $(LIBS) -lm $(MORELIBS)
\end{verbatim}

The simplicity of this makefile is drawn from the complexity of
makeflag.unx. The file makeflag.unx sets the CC, CFLAGS, LIBS, and
MORELIBS for the machined using among other things. Furthermore,
makeflag.unx also sets the default compilation rules from C sources to
object files.  The file makeflag.unx had to be modified once, when
IRIT was installed on this system. If the used system is not a unix
environment, then the file makefile.unx will have to be replaced with
the proper makeflag file.  In an OS2 environment, using the emx gcc
compiler, the makefile is even simpler since the linking rule is also
defined in makeflag.os2:

\begin{verbatim}
IRIT_DIR = \usr\local\irit

include $(IRIT_DIR)\makeflag.os2

OBJS	= program.o

program.exe: $(OBJS)
\end{verbatim}

Finally, here is a proper makefile for Windows NT:

\begin{verbatim}
IRIT_DIR = \usr\local\irit

include $(IRIT_DIR)\makeflag.wnt

OBJS	= program.obj

program.exe: $(OBJS)
        $(IRITCONLINK) -out:program.exe $(OBJS) $(LIBS) $(W32CONLIBS)
\end{verbatim}

\section{Simple C Programs using IRIT}

Now that we have an idea how to compile C code using IRIT, here are several
examples to read, manipulate and write IRIT data files. You will be able to
find all these examples in the doc/cexample directory.

\subsection{Boolean Operations over Polyhedra (boolean.c)}

This example demonstates the use of the Boolean operations package
(bool\_lib), creates few polygonal primitives and apply Boolean
operations between them.  The resulting model is dumped to stdout.

\input{cexample/boolean.tex}

\subsection{Distance Maps to Planar Curves (dist\_map.c)}

This example demonstates the use of symbolic distance maps
computations to freeform curves, in the plane, and the sampling
of the map into an image that is dumped out.

\input{cexample/dist_map.tex}

\subsection{Importance Edges Evaluations in Polygonal Meshes (imprtnc.c)}

This example demonstates the tesselation of arbitrary input model(s)
into polygons and the probabalistic estimation of importance edges in the
polygonal geometry.  Dumped out is the polygonal geometry with vertices
having UV coordinates if available, normals if available, and importance.

\begin{verbatim}
/*****************************************************************************
*   Computes the importance of vertices and dump as vertex list + polygons   *
******************************************************************************
* (C) Gershon Elber, Technion, Israel Institute of Technology                *
******************************************************************************
* Written by:  Gershon Elber                                Ver 1.0, June 1995   *
*****************************************************************************/

#include "irit_sm.h"
#include "allocate.h"
#include "iritprsr.h"
#include "geom_lib.h"
#include "misc_lib.h"
#include "ip_cnvrt.h"

IRIT_STATIC_DATA char
    *CtrlStr = "Imprtnc c%- h%- DFiles!*s";
IRIT_STATIC_DATA int GlblVrtxImportanceCount,
    GlblCrvtrInfoFlag = FALSE;
IRIT_STATIC_DATA IrtRType GlblVrtxImportanceVal;
IRIT_STATIC_DATA IPVertexStruct *GlblVrtxImportance;

static void DumpOneTraversedObject(IPObjectStruct *PObj, IrtHmgnMatType Mat);
static IPObjectStruct *SetCurvatureEstimates(IPObjectStruct *PObj);
static void DumpOneObjData(IPObjectStruct *PObj);
static void ProcessVertexImportance(IPVertexStruct *V1,
                                    IPVertexStruct *V2,
                                    IPPolygonStruct *Pl1,
                                    IPPolygonStruct *Pl2);
static void GenPolyImportance(IPObjectStruct *PObj);

void main(int argc, char **argv)
{
    int NumFiles, Error,
        HelpFlag = FALSE;
    char **FileNames;
    IPObjectStruct *PObjects;
    IrtHmgnMatType CrntViewMat;

    if ((Error = GAGetArgs(argc, argv, CtrlStr, &GlblCrvtrInfoFlag,
                           &HelpFlag, &NumFiles, &FileNames)) != 0) {
        GAPrintErrMsg(Error);
        GAPrintHowTo(CtrlStr);
        exit(1);
    }

    if (HelpFlag) {
        fprintf(stderr, "This is Importance testing...\n");
        GAPrintHowTo(CtrlStr);
        exit(0);
    }

    /* Get the data files: */
    IPSetFlattenObjects(FALSE);
    if ((PObjects = IPGetDataFiles(FileNames, NumFiles, TRUE, FALSE)) == NULL)
        exit(1);
    PObjects = IPResolveInstances(PObjects);

    if (IPWasPrspMat)
        MatMultTwo4by4(CrntViewMat, IPViewMat, IPPrspMat);
    else
        IRIT_GEN_COPY(CrntViewMat, IPViewMat, sizeof(IrtHmgnMatType));

    /* Here some useful parameters to play with in tesselating freeforms: */
    IPFFCState.FineNess = 15; /* Resolution of tesselation, larger is finer. */
    IPFFCState.ComputeUV = TRUE;       /* Wants UV coordinates for textures. */
    IPFFCState.FourPerFlat = TRUE;/* 4 polygons per ~flat patch, 2 otherwise.*/
    IPFFCState.LinearOnePolyFlag = TRUE;   /* Linear srf generates one poly. */

    IPTraverseObjListHierarchy(PObjects, CrntViewMat, DumpOneTraversedObject);

    IPFreeObjectList(PObjects);
}

/*****************************************************************************
* DESCRIPTION:                                                               *
*   Update curvature property attribute to vertices of the given poly model. *
*                                                                            *
* PARAMETERS:                                                                *
*   PObj:   Poly model to udpate curvatrue info into.                             *
*                                                                            *
* RETURN VALUE:                                                              *
*   IPObjectStruct *:   A poly model, similar to PObj, but with curvature    *
*                        information attached as attributes.                  *
*****************************************************************************/
static IPObjectStruct *SetCurvatureEstimates(IPObjectStruct *PObj)
{
    int OldCirc = IPSetPolyListCirc(TRUE);
    IPPolygonStruct *Pl;
    IPObjectStruct *PTmp1, *PTmp2;

    /* Convert to a regular polygonal model with triangles only. */
    for (Pl = PObj -> U.Pl; Pl != NULL; Pl = Pl -> Pnext) {
        if (IPVrtxListLen(Pl -> PVertex) != 3)
            break;
    }
    PTmp2 = IPCopyObject(NULL, PObj, FALSE);
    GMVrtxListToCircOrLin(PTmp2 -> U.Pl, TRUE);
    if (Pl != NULL) {
        PTmp1 = GMConvertPolysToTriangles(PTmp2);
        IPFreeObject(PTmp2);
        PTmp2 = GMRegularizePolyModel(PTmp1);
        IPFreeObject(PTmp1);
    }

    GMVrtxListToCircOrLin(PTmp2 -> U.Pl, FALSE);
    IPSetPolyListCirc(OldCirc);

    GMPlCrvtrSetCurvatureAttr(PTmp2 -> U.Pl, 1, TRUE);

    return PTmp2;
}

/*****************************************************************************
* DESCRIPTION:                                                               *
*   Call back function of IPTraverseObjListHierarchy. Called on every non    *
* list object found in hierarchy.                                            *
*                                                                            *
* PARAMETERS:                                                                *
*   PObj:       Non list object to handle.                                   *
*   Mat:        Transformation matrix to apply to this object.               *
*                                                                            *
* RETURN VALUE:                                                              *
*   void                                                                     *
*****************************************************************************/
static void DumpOneTraversedObject(IPObjectStruct *PObj, IrtHmgnMatType Mat)
{
    IPObjectStruct *PObjs;

    if (IP_IS_FFGEOM_OBJ(PObj))
        PObjs = IPConvertFreeForm(PObj, &IPFFCState);
    else
        PObjs = PObj;

    for (PObj = PObjs; PObj != NULL; PObj = PObj -> Pnext) {
        if (IP_IS_POLY_OBJ(PObj)) {
            if (GlblCrvtrInfoFlag) {
                IPObjectStruct
                    *PTri = SetCurvatureEstimates(PObj);

                DumpOneObjData(PTri);
                IPFreeObject(PTri);
            }
            else
                DumpOneObjData(PObj);
        }
    }
}

/*****************************************************************************
* DESCRIPTION:                                                               *
*   Convert the polygonal mesh into a vertex list with importance and the    *
* polygons as indices into this list.                                        *
*                                                                            *
* PARAMETERS:                                                                *
*   PObj:   A polygonal mesh to dump.                                        *
*                                                                            *
* RETURN VALUE:                                                              *
*   void                                                                     *
*****************************************************************************/
static void DumpOneObjData(IPObjectStruct *PObj)
{
    IPPolyVrtxIdxStruct
        *PVIdx = IPCnvPolyToPolyVrtxIdxStruct(PObj, TRUE, 0);
    int        i,
        **Polygons = PVIdx -> Polygons;
    IPVertexStruct
        **Vertices = PVIdx -> Vertices;

    /* Compute importance for the vertices as "Imprt" attributes.  Note the */
    /* PVIdx data structure points on the original vertices so we are fine. */
    GenPolyImportance(PObj);

    /* Dump the vertices: */
    fprintf(stderr, "OBJECT \"%s\" - VERTICES:\n", PObj -> ObjName);
    for (i = 0; Vertices[i] != NULL; i++) {
        IrtRType R;
        float *uv;

        printf("%3d:  %6.3f %6.3f %6.3f", i,
               Vertices[i] -> Coord[0], 
               Vertices[i] -> Coord[1], 
               Vertices[i] -> Coord[2]);
        if (IP_HAS_NORMAL_VRTX(Vertices[i]))
            printf(" [%6.3f %6.3f %6.3f]",
                   Vertices[i] -> Normal[0], 
                   Vertices[i] -> Normal[1], 
                   Vertices[i] -> Normal[2]);
        if ((uv = AttrGetUVAttrib(Vertices[i] -> Attr, "uvvals")) != NULL)
            printf(" {%6.3f %6.3f}", uv[0], uv[1]);

        R = AttrGetRealAttrib(Vertices[i] -> Attr, "Imprt");
        if (!IP_ATTR_IS_BAD_REAL(R))
            printf(" (%6.3f)", R);

        if (GlblCrvtrInfoFlag) {
            printf("\n\t<K1=%6.3f D1 = %s>\n\t<K2=%6.3f D2 = %s>",
                   AttrGetRealAttrib(Vertices[i] -> Attr, "K1Curv"),
                   AttrGetStrAttrib(Vertices[i] -> Attr, "D1"),
                   AttrGetRealAttrib(Vertices[i] -> Attr, "K2Curv"),
                   AttrGetStrAttrib(Vertices[i] -> Attr, "D2"));
        }

        printf("\n");
    }

    fprintf(stderr, "\nOBJECT \"%s\" - Vertices in polygons:\n",
            PObj -> ObjName);
    for (i = 0; PVIdx -> PPolys[i] != NULL; i++) {
        IPPolyPtrStruct
            *PPoly = PVIdx -> PPolys[i];

        printf("%3d: ", i);
        for ( ; PPoly != NULL; PPoly = PPoly -> Pnext) {
            printf("%3d ", AttrGetIntAttrib(PPoly -> Poly -> Attr, "_PIdx"));
        }
        printf("\n");
    }

    /* Dump the polygons: */
    fprintf(stderr, "\nOBJECT \"%s\" - Polygons from vertices:\n",
            PObj -> ObjName);
    for (i = 0; Polygons[i] != NULL; i++) {
        int *Pl = Polygons[i];

        printf("%3d: ", i);
        while (*Pl >= 0)
            fprintf(stderr, " %5d", *Pl++);

        fprintf(stderr, "    -1\n");
    }

    IPPolyVrtxIdxFree(PVIdx);
}

/*****************************************************************************
* DESCRIPTION:                                                               *
*   Call back function for GMPolyAdjacncyVertex to process every edge of a   *
* given vertex.  The edge is provided as (V, V -> Pnext)                     *
*                                                                            *
* PARAMETERS:                                                                *
*   V1, V2:    Two vertices defining this edge.  Note the vertices are NOT   *
*               necessarily chained together into a list.                     *
*   Pl1, Pl2:  The two polygons that share this edge.  The edge (V1, V2) is  *
*               in both Pl1 and Pl2, with not necessarily the exact pointers  *
*               IPVertexStruct of V1 and V2.                                     *
*                                                                            *
* RETURN VALUE:                                                              *
*   void                                                                     *
*****************************************************************************/
static void ProcessVertexImportance(IPVertexStruct *V1,
                                    IPVertexStruct *V2,
                                    IPPolygonStruct *Pl1,
                                    IPPolygonStruct *Pl2)
{
    if (!IRIT_PT_APX_EQ_EPS(V1 -> Coord, GlblVrtxImportance -> Coord,
                            IRIT_EPS) &&
        !IRIT_PT_APX_EQ_EPS(V2 -> Coord, GlblVrtxImportance -> Coord,
                            IRIT_EPS))
        fprintf(stderr, "Edge does not match the given vertex - adj error!\n");

    if (Pl1 != NULL && Pl2 != NULL) {
        GlblVrtxImportanceCount++;
        GlblVrtxImportanceVal += acos(IRIT_DOT_PROD(Pl1 -> Plane,
                                                    Pl2 -> Plane));
    }
}

/*****************************************************************************
* DESCRIPTION:                                                               *
*   Computes importance for each vertex in the given polygonal mesh.         *
*                                                                            *
* PARAMETERS:                                                                *
*   PObj:         Polygonal model to process for importance.                     *
*                                                                            *
* RETURN VALUE:                                                              *
*   void                                                                     *
*****************************************************************************/
static void GenPolyImportance(IPObjectStruct *PObj)
{
    VoidPtr
        PAdj = GMPolyAdjacncyGen(PObj, IRIT_EPS);
    IPPolygonStruct *Pl;

    for (Pl = PObj -> U.Pl; Pl != NULL; Pl = Pl -> Pnext) {
        int PlImportanceCount = 0;
        IrtRType
            PlImportanceVal = 0.0;
        IPVertexStruct *V;

        for (V = Pl -> PVertex; V != NULL; V = V -> Pnext) {
            GlblVrtxImportanceVal = 0.0;
            GlblVrtxImportanceCount = 0;
            GlblVrtxImportance = V;

            GMPolyAdjacncyVertex(V, PAdj, ProcessVertexImportance);

            if (GlblVrtxImportanceCount > 0) {
                GlblVrtxImportanceVal /= (GlblVrtxImportanceCount * M_PI);

                AttrSetRealAttrib(&V -> Attr, "Imprt", GlblVrtxImportanceVal);
            }
            else
                fprintf(stderr, "Failed to compute Importance for vertex\n");
        }
    }

    GMPolyAdjacncyFree(PAdj);
}
\end{verbatim}


\subsection{Least sqaures curve fitting and process communication (lst\_sqrs.c)}

This example creates a Bspline curve that least squares approximates
a given set of data points. In the default values, a quadratic Bspline
curve of 10 control points least sqaures approximates a set of 100
three dimensional points. Also shown is a polyline of the original
(100) points. Arbitrary number of curves will be displayed every, one
for every <return> keystroke until Q<return> is typed.

\begin{verbatim}
/*****************************************************************************
* Least squares fit a curve to random points.                                     *
* We also see how to fork out a display device and communicate with it.      *
******************************************************************************
* (C) Gershon Elber, Technion, Israel Institute of Technology                *
******************************************************************************
* Written by:  Gershon Elber                                Ver 1.0, June 1995   *
*****************************************************************************/

#include "irit_sm.h"
#include "iritprsr.h"
#include "allocate.h"
#include "attribut.h"
#include "cagd_lib.h"
#include "geom_lib.h"
#include "grap_lib.h"
#include "misc_lib.h"

static char *CtrlStr =
    "Lst_Sqrs n%-#Pts!d d%-Degree!d f%-DOF!d p%-PrgmName!s h%-";

void main(int argc, char **argv)
{
    int i, Error, PrgmIO,
        NumOfPoints = 100,
        NumOfPtsFlag = FALSE,
        Degree = 3,
        DegreeFlag = FALSE,
        NumOfDOF = 10,
        NumOfDOFFlag = FALSE,
        PrgmFlag = FALSE,
        HelpFlag = FALSE;
    char *Err,
        *Program = getenv("IRIT_DISPLAY");

#ifdef __WINNT__
    if (Program == NULL)
        Program = "wntgdrvs -s-";
#endif /* __WINNT__ */
#ifdef __UNIX__
    if (Program == NULL)
        Program = "x11drvs -s-";
#endif /* __UNIX__ */

    if ((Error = GAGetArgs(argc, argv, CtrlStr,
                           &NumOfPtsFlag, &NumOfPoints,
                           &DegreeFlag, &Degree,
                           &NumOfDOFFlag, &NumOfDOF,
                           &PrgmFlag, &Program,
                           &HelpFlag)) != 0) {
        GAPrintErrMsg(Error);
        GAPrintHowTo(CtrlStr);
        exit(1);
    }

    if (HelpFlag) {
        GAPrintHowTo(CtrlStr);
        exit(0);
    }

    IPSocSrvrInit();            /* Initialize the listen socket for clients. */

    if ((PrgmIO = IPSocExecAndConnect(Program,
                                      getenv("IRIT_BIN_IPC") != NULL)) >= 0) {
        char Line[IRIT_LINE_LEN];
        IPObjectStruct
            *PClrObj = IPGenStrObject("command_", "clear", NULL);

        do {
            CagdPtStruct
                *PtList = NULL;
            IPPolygonStruct
                *PPoly = IPAllocPolygon(0, NULL, NULL);
            CagdCrvStruct *Crv;
            IPObjectStruct *PCrvObj, *PPolyObj;

            for (i = 0; i < NumOfPoints; i++) {
                int j;
                IPVertexStruct *V;
                CagdPtStruct
                    *Pt = CagdPtNew();

                if (i == 0) {
                    for (j = 0; j < 3; j++)
                        Pt -> Pt[j] = IritRandom(-1.0, 1.0);
                }
                else {
                    for (j = 0; j < 3; j++)
                        Pt -> Pt[j] = PtList -> Pt[j] + IritRandom(-0.1, 0.1);
                }

                V = IPAllocVertex(0, NULL, PPoly -> PVertex);
                for (j = 0; j < 3; j++)
                    V -> Coord[j] = Pt -> Pt[j];
                PPoly -> PVertex = V;

                IRIT_LIST_PUSH(Pt, PtList);
            }

            Crv = BspCrvInterpPts(PtList, Degree + 1,
                                  NumOfDOF, CAGD_UNIFORM_PARAM, FALSE);
            CagdPtFreeList(PtList);

            CagdCrvWriteToFile3(Crv, stdout, 0, "This is from LstSqrs", &Err);

            /* Generate objects out of the geometry and set proper attrs. */
            PCrvObj = IPGenCRVObject(Crv);
            AttrSetObjectColor(PCrvObj, IG_IRIT_CYAN);

            PPolyObj = IPGenPOLYObject(PPoly);
            IP_SET_POLYLINE_OBJ(PPolyObj);
            AttrSetObjectColor(PPolyObj, IG_IRIT_YELLOW);

            /* Clear old data and display our curve and data. */
            IPSocWriteOneObject(PrgmIO, PClrObj);
            IPSocWriteOneObject(PrgmIO, PCrvObj);
            IPSocWriteOneObject(PrgmIO, PPolyObj);

            IPFreeObject(PCrvObj);
            IPFreeObject(PPolyObj);

            gets(Line);
        }
        while (Line[0] != 'q' && Line[0] != 'Q');

        IPSocDisConnectAndKill(TRUE, PrgmIO);
    }

    exit(0);
}
\end{verbatim}


\subsection{Multivariate Solver (msolve.c)}

This example demonstates the use of the multivariate solver from the
mvar\_lib library.  Input is a set of n polynomial equations in m
variables and dumped out are the solutions to these equations.

\input{cexample/msolve.tex}

\subsection{Compute Area of a Polygonal Model (polyarea.c)}

Here is a simple program to compute the total area of all polygons in
the given data file. The program expects one argument on the command
line which is the name of the file to read, and it prints out one
line with the total computed area.

\input{cexample/polyarea.tex}

\subsection{Converts a Freeform Surface into Polygons (polygons.c)}

This true filter reads a single surface from stdin and dumps out a
polygonal approximation of it to stdout. They are several parameters
that controls the way a surface is approximated into polygons and in
this simple filter they are being held fixed in a set of integer
variables.

\begin{verbatim}
/*****************************************************************************
* Read a surface and dump out a tesselated version of it.                     *
******************************************************************************
* (C) Gershon Elber, Technion, Israel Institute of Technology                *
******************************************************************************
* Written by:  Gershon Elber                                Ver 1.0, June 1995   *
*****************************************************************************/

#include "irit_sm.h"
#include "cagd_lib.h"
#include "iritprsr.h"
#include "ip_cnvrt.h"
#include "allocate.h"

void main(int argc, char **argv)
{
    int Handler,
        FourPerFlat = TRUE, /* Settable parameters of IritSurface2Polygons. */
        FineNess = 20,
        ComputeUV = FALSE,
        ComputeNrml = FALSE,
        Optimal = FALSE;

    if ((Handler = IPOpenDataFile("-", TRUE, TRUE)) >= 0) {
        IPObjectStruct
            *PObj = IPGetObjects(Handler);

        /* Done with file - close it. */
        IPCloseStream(Handler, TRUE);

        /* Process the surface into polygons. */
        if (IP_IS_SRF_OBJ(PObj)) {
            IPPolygonStruct
                *PPoly = IPSurface2Polygons(PObj -> U.Srfs, FourPerFlat,
                                            FineNess, ComputeUV,
                                            ComputeNrml, Optimal);
            IPObjectStruct
                *PObjPoly = IPGenPOLYObject(PPoly);

            IPStdoutObject(PObjPoly, FALSE);

            IPFreeObject(PObjPoly);
        }
        else
            fprintf(stderr, "Read object is not a surface.\n");

        IPFreeObject(PObj);
    }
    else {
        fprintf(stderr, "Failed to read from stdin\n");
        exit(1);
    }

    exit(0);
}
\end{verbatim}


\subsection{Linear Transformations' Filter (transfrm.c)}

This little more complex program transfroms all the geometry in the
read data which can be any number of files according to the specified
transformations on the command line. The command line is parsed via
{\bf GAGetArgs} and its associated functions. The transformation
matrix is then computed with the aid of the matrix package and applied
to the read geometry at once.

\input{cexample/transfrm.tex}

Here is the result of running 'transfrm -h':
\begin{verbatim}
This is Transform...
Usage: Transfrm [-x Degs] [-y Degs] [-z Degs] [-t X Y Z] [-s Scale] [-h] DFiles
\end{verbatim}

When you are considering the usefulness of this tool remember that the
transformations are applied to the geometry in an internal order which is
different from the command line order. That is,
\begin{verbatim}
     transfrm -x 30 -y 30 geometry.dat > tgeometry.dat
\end{verbatim}
will compute the exact same transfrom as,
\begin{verbatim}
     transfrm -y 30 -x 30 geometry.dat > tgeometry.dat
\end{verbatim}
This, while rotation is not a commutative operation.
Nonetheless, you may split the operations. That is:
\begin{verbatim}
     transfrm -y 30 geometry.dat | transfrm -x 30 - > tgeometry.dat
\end{verbatim}
or
\begin{verbatim}
     transfrm -x 30 geometry.dat | transfrm -y 30 - > tgeometry.dat
\end{verbatim}
will do exactly what one expects.

%%%%%%%%%%%%%%%%%%%%%%%%%%%%%%%%%%%%%%%%%%%%%%%%%%%%%%%%%%%%%%%%%%%%%%%%%%%%%
\twocolumn
% 
% prog_man.tex - The IRIT programmer's manual - main module.
% 
% Author:	Gershon Elber
% 		Computer Science Dept.
% 		Technion, IIT
%

%\documentstyle[10pt]{book}
\documentstyle{book}

\setlength{\oddsidemargin}{0.0in}
\setlength{\evensidemargin}{0.0in}
\setlength{\topmargin}{-0.2in}
\setlength{\textwidth}{6.5in}
\setlength{\textheight}{9.0in}

\makeindex

\begin{document}

\small
\setlength{\baselineskip}{0.85\baselineskip}

\tableofcontents

%%%%%%%%%%%%%%%%%%%%%%%%%%%%%%%%%%%%%%%%%%%%%%%%%%%%%%%%%%%%%%%%%%%%%%%%%%%%%
\chapter{Introduction}

This manual describes the different libraries of the IRIT solid
modeling environment. Quite a few libraries can be found to manipulate
geometry in general, freeform curves and surfaces, symbolic
computation, trimmed surfaces, triangular patches, freeform trivariate
functions, Boolean operations, input output data file parsing, and
miscelleneous.

All interface to the libraries should be made via the appropriate header
files that can be found in the include subdirectory. Most libraries
have a single header file that is named the same as the library.
Functions and constants that are visible to the users of the libraries
are prefixed with a unique prefix, usually derived from the library
name itself. External definitions that start with an underscore should
not be used, even if found in header files.

The header file {\bf include/irit\_sm.h} must be sourced by every
source file in the solid modeller. In most cases, this file is sourced
indirectly via local header files.

The following libraries are avaliable in IRIT:
\begin{center}
\begin{tabular}{|l|l|} \hline
    Name of Library & Tasks \\ \hline
    bool & Boolean operations on polygonal models. \\
    cagd & Low level freeform curves and surfaces. \\
    geom & General geometry functions. \\
    grap & General graphics/display functions. \\
    mdl  & Model's processing functions. \\
    misc & Memory allocation, configuration files, attributes, etc. \\
    mvar & Multi variate functions. \\
    prsr & Input and output for file/sockets of objects of IRIT. \\
    rndr & Scan conversion rendering functions. \\
    symb & Symbolic manipulation of curves and surfaces. \\
    trim & Trimmed surfaces support. \\
    triv & Freeform trivariate functions. \\
    trng & Triangular patches support. \\
    user & General high level user inteface functions. \\
    xtra & Public domain code that is not part of IRIT. \\ \hline
\end{tabular}
\end{center}

%%%%%%%%%%%%%%%%%%%%%%%%%%%%%%%%%%%%%%%%%%%%%%%%%%%%%%%%%%%%%%%%%%%%%%%%%%%%%

\subsection{Tools and Programs}

The IRIT package includes several complete programs such as poly3d-h
(hidden line removal), irender (scan conversion tool), and irit2ps (a
filter to Postscript).  Somewhat different than most other programs is
the kernel interpreter, also called irit.  The irit program is nothing
more than an interpreter (written in C) that invokes numerous
functions in the several libraries provided.  In order to add a new
function to the irit interpreter, the following sequence of operations
must be followed:

\begin{enumerate}
\item Write a C function that accepts only one or more of the following
     type of parameters.  All parameters, including the IrtRType, {\em must}
     be transferred by address:
    \begin{itemize}
	\item IrtRType         (see irit\_sm.h).
	\item IrtVecType       (see irit\_sm.h).
	\item IrtPtType        (see irit\_sm.h).
	\item CagdCtlPtStruct  (see cagd\_lib.h).
	\item IrtPlnType       (see irit\_sm.h).
	\item StringType       (char *).
	\item IPObjectStruct   (see iritprsr.h).  This includes all object
		types in IPObjectStruct other than the above.
    \end{itemize}
\item The written C function can return one of:
    \begin{itemize}
	\item IrtRType by value.
	\item IPObjectStruct by address.
	\item Nothing (a procedure).
    \end{itemize}
\item According to the returned type by the new C function, go to file
	{\bf inptprsl.h}, and add a new enum NEW\_FUNC for the new function in
	enum RealValueFuncType (for IrtRType returned value), in
	enum ObjValueFuncType (for IPObjectStruct * returned value), or in
	enum GenValueFuncType (for no returned value).
\item In {\bf inptevl1.c}, add one new line in one of NumFuncTable,
	ObjFuncTable, or GenFuncTable (depends upon the return value).
	The line will have the form of:
	\begin{verbatim}
            { ``FuncName'', NEW_FUNC, CFunctionName,  N, { Param1Type,
                                              Param2Type, ... ParamNType } }
	\end{verbatim}
	for procedures with no return values, and of the form of:
	\begin{verbatim}
            { ``FuncName'', NEW_FUNC, CFunctionName,  N, { Param1Type,
                                    Param2Type, ... ParamNType }, RetValType }
	\end{verbatim}
	otherwise. \verb-N- is the number of parameters,
	\verb-`FuncName''- is the (unique) name that will be used in
	the interpreter, CFunctionName is the name of the C function you
	have written. This C function must be declared in one of the header
	files that inptevl1.c includes. ParamIType, for I between 1 and N,
	and RetValType are of type IritExprType (see inptprsl.h).
\item Thats it!
\end{enumerate}

For example, to add the C function declared as
\begin{verbatim}
    IPObjectStruct *PolyhedraMoments(IPObjectStruct *IPObjectStruct,
                                     IrtRType *m);
\end{verbatim}
one has to add the following line to ObjFuncTable in iritevl1.c
\begin{verbatim}
    { ``PMOMENT'', PMOMENT, { POLY_EXPR, REAL_EXPR }, VECTOR_EXPR },
\end{verbatim}
where PMOMENT needs to be added to ObjValueFuncType in iritprsl.h.

While all objects in the interpreted space are of type IPObjectStruct,
the functions' interface unfolds many types to simplify matter.
IrtRType, IrtVecType, Strings, etc. are all extracted from the given
(IPObjectStruct) parameters and passed directly by address.  For
returned values, only numeric real data is allowed where everything
else must be returned wrapped in an IPObjectStruct.

%%%%%%%%%%%%%%%%%%%%%%%%%%%%%%%%%%%%%%%%%%%%%%%%%%%%%%%%%%%%%%%%%%%%%%%%%%%%%

The following chapters reference the different functions of the different
libraries. Chapter~\ref{chap-examples} provides several examples of
writing C code using the IRIT libraries.

%%%%%%%%%%%%%%%%%%%%%%%%%%%%%%%%%%%%%%%%%%%%%%%%%%%%%%%%%%%%%%%%%%%%%%%%%%%%%
\chapter{Boolean Library, bool\_lib}

\section{General Information}

On of the crucial operation in any solid modeling environment is the
ability to perform Boolean operations among different geometric
objects.  The interface of the library is defined in {\em
include/bool\_lib.h}. This library supports only Boolean operations of
polygonal objects. The Boolean operations of OR, AND, SUBtract,
NEGate, CUT, and MERGE are supported via the {\bf BoolOperType}
typedef:
\begin{center}
\begin{tabular}{||c|c||} \hline 
    BoolOperType & Meaning \\ \hline
    BOOL\_OPER\_OR & Union of two geometrical objects \\
    BOOL\_OPER\_AND & Intersection of two geometrical objects \\
    BOOL\_OPER\_SUB & The difference of two geometrical objects \\
    BOOL\_OPER\_NEG & Unary Inside-out of one geometrical object \\
    BOOL\_OPER\_CUT & Boundary of one object outside the other \\
    BOOL\_OPER\_MERGE & Simple merge without any computation \\ \hline
\end{tabular}
\end{center}

The {\bf BoolOperType} typedef is used in two dimensional Boolean operations
invoked via {\bf Boolean2D}. Three dimensional Boolean operations are
invoked via {\bf BooleanXXX} where XXX is one of OR, AND, SUB, NEG, CUT
or MERGE, denoting the same as in the table above.

In addition several state functions are available to control the way
the Boolean operations are conducted. Functions to enable the dump of
(only) the intersection curves, to handle coplanar polygons, and to
set the axis along which the sweep is performed are provided.

All globals in this library have a profix of {\bf Bool}.

%%%%%%%%%%%%%%%%%%%%%%%%%%%%%%%%%

\section{Algorithmic Hints Behind the Boolean Operations}

Let ${\cal M}_i,~i = 1,2$ be two polygonal models to compute a Boolean
operation between and let $P_m$ denote the $m$'th polygon of ${\cal M}_i$.
Denote by $InterList( P_m ),~P_m \in {\cal M}_i$, the list of line
segments representing the intersections of polygon $P_m$ with polygons
in ${\cal M}_j$.  A naive first, intersection, step in a Boolean
operation between two polyhedra models, ${\cal M}_1$ and ${\cal M}_2$,
will consist of the following stages:
\begin{bf}
\begin{tt}
\begin{tabbing}
\hspace{2em}\=\hspace{2em}\=\hspace{2em}\=\hspace{2em}\=\hspace{2em}\=\hspace{2em}\=\kill
\> For each polygon $P_m$ in ${\cal M}_1$ \\
\> begin \\
\> \> $InterList( P_m )$ $\Leftarrow$ $\emptyset$ \\
\> end \\
\> For each polygon $P_n$ in ${\cal M}_2$ \\
\> begin \\
\> \> $InterList( P_n )$ $\Leftarrow$ $\emptyset$ \\
\> end \\
\\
\> For each polygon $P_m$ in ${\cal M}_1$ \\
\> begin \\
\> \> For each polygon $P_n$ in ${\cal M}_2$ \\
\> \> begin \\
\> \> \> If ( $P_m \cap P_n$ ) \\
\> \> \> begin \\
\> \> \> \> ${\cal L}$ $\Leftarrow$ $P_m \cap P_n$; \\
\> \> \> \> $InterList( P_m )$ $\Leftarrow$ $InterList( P_m )$ $\cup$ ${\cal L}$; \\
\> \> \> \> $InterList( P_n )$ $\Leftarrow$ $InterList( P_n )$ $\cup$ ${\cal L}$; \\
\> \> \> end \\
\> \> end \\
\> end
\end{tabbing}
\end{tt}
\end{bf}

At the end of the first, intersection stage, each polygon, $P_m$ in
either ${\cal M}_1$ (or $P_n \in {\cal M}_2$) contains in its
$InterList( P_m )$ the set of line segments of the intersection(s) of
$P_m$ with all polygons in the other model.  In order to figure out
orientation, each intersection line segment also contains an
orientation hint as to the inside direction of the two polygons that
intersect at that line segment.  This list of intersection line
segments can now be processed:

\begin{bf}
\begin{tt}
\begin{tabbing}
\hspace{2em}\=\hspace{2em}\=\hspace{2em}\=\hspace{2em}\=\hspace{2em}\=\kill
\> For each polygon $P_m$ in ${\cal M}_1$ \\
\> begin \\
\> \> $Polys$ $\Leftarrow$ $InterList( P_m )$ sorted into polylines; \\
\> \> $OpenList( P_m )$ $\Leftarrow$ Open polylines of $Polys$; \\
\> \> $ClosedList( P_m )$ $\Leftarrow$ Closed polylines of $Polys$; \\
\> end \\
\> For each polygon $P_n$ in ${\cal M}_2$ \\
\> begin \\
\> \> $Polys$ $\Leftarrow$ $InterList( P_n )$ sorted into polylines; \\
\> \> $OpenList( P_n )$ $\Leftarrow$ Open polylines of $Polys$; \\
\> \> $ClosedList( P_n )$ $\Leftarrow$ Closed polylines of $Polys$; \\
\> end \\
\end{tabbing}
\end{tt}
\end{bf}

Open polylines, $OpenList( P_m )$, are intersection polylines that are
open.  These polylines starts on the boundary of polygon $P_m$ and
also ends on the boundary, at a different boundary location though.
In contrast, the set of closed polylines, $ClosedList( P_n )$, contains
closed loops that are completely contained in $P_n$.

The polygons are now ready for {\em trimming}.  Depending upon the Boolean
operation that is requested, the inside or the outside of ${\cal M}_1$ or
${\cal M}_2$ is selected for the output result.  Denote by
${\cal M}_i \ Out \ {\cal M}_j$ (${\cal M}_i \ In \ {\cal M}_j$) the region of
${\cal M}_i$ that is outside (inside) ${\cal M}_j$. Then,

\bigskip 
\begin{center}
\begin{tabular}{||c|c||} \hline
Boolean Operation & implementation as \\ \hline \hline
${\cal M}_1 \cup {\cal M}_1$ & $\{ {\cal M}_1\ Out\ {\cal M}_2 \} \cup
			     \{ {\cal M}_2\ Out\ {\cal M}_1 \}$ \\
${\cal M}_1 \cap {\cal M}_1$ & $\{ {\cal M}_1\ In\ {\cal M}_2 \} \cup
			     \{ {\cal M}_2\ In\ {\cal M}_1 \}$ \\
${\cal M}_1 - {\cal M}_1$ & $\{ {\cal M}_1\ Out\ {\cal M}_2 \} \cup
			     Inverse\{ {\cal M}_2\ In\ {\cal M}_1 \}$ \\ \hline
\end{tabular}
\end{center}
\bigskip 

Note the $Inverse$ operation that reverses the orientation inside
out.

At this point, all Boolean operations are translated into two complementary
type of operations, $In$ and $Out$. The computation of
${\cal M}_i \ Out \ {\cal M}_j$ is conducted by tracing the outside regions
of the polygons of ${\cal M}_i$ that {\em intersect} ${\cal M}_j$ and
{\em propagating} this inside--outside information to the non intersecting
polygons:
\begin{bf}
\begin{tt}
\begin{tabbing}
\hspace{2em}\=\hspace{2em}\=\hspace{2em}\=\hspace{2em}\=\hspace{2em}\=\kill
\> For each polygon $P_m$ in ${\cal M}_1$ with non empty $OpenList( P_m )$ \\
\> \mbox{\hspace{2.425in}} or non empty $ClosedList( P_m )$ \\
\> begin \\
\> \> $P_m^{in}$ $\Leftarrow$ Inside region of $P_m$; \\
\> \> $P_m^{out}$ $\Leftarrow$ Outside region of $P_m$; \\
\\
\> \> For each {\em original} edge, $e \in P_m^{in}$ \\
\> \> begin \\
\> \> \> Push $e$ onto $InsideStack$; \\
\> \> end \\
\> \> For each {\em original} edge, $e \in P_m^{out}$ \\
\> \> begin \\
\> \> \> Push $e$ onto $OutSideStack$; \\
\> \> end \\
\> end \\
\\
\> While $InsideStack$ not empty\\
\> begin \\
\> \> $e$ $\Leftarrow$ edge poped from top of $InsideStack$; \\
\> \> $P_m$, $P_l$ $\Leftarrow$ two polygons sharing $e$ in ${\cal M}_1$; \\
\\
\> \> If ( $P_m$ has empty $OpenList( P_m )$ and empty $ClosedList( P_m )$, \\
\> \> \mbox{\hspace{3in}} and $P_m$ is not classified yet ) \\
\> \> begin \\
\> \> \> Classify $P_m$ as inside; \\
\> \> \> push all edges of $P_m$, but $e$, onto $InsideStack$; \\
\> \> end \\
\> \> If ( $P_l$ has empty $OpenList( P_l )$ and empty $ClosedList( P_l )$, \\
\> \> \mbox{\hspace{3in}} and $P_l$ is not classified yet ) \\
\> \> begin \\
\> \> \> Classify $P_l$ as inside; \\
\> \> \> push all edges of $P_l$, but $e$, onto $InsideStack$; \\
\> \> end \\
\> end \\
\\
\> While $OutsideStack$ not empty\\
\> begin \\
\> \> $e$ $\Leftarrow$ edge poped from top of $outsideStack$; \\
\> \> $P_m$, $P_l$ $\Leftarrow$ two polygons sharing $e$ in ${\cal M}_1$; \\
\\
\> \> If ( $P_m$ has empty $OpenList( P_m )$ and empty $ClosedList( P_m )$, \\
\> \> \mbox{\hspace{3in}} and $P_m$ is not classified yet ) \\
\> \> begin \\
\> \> \> Classify $P_m$ as outside; \\
\> \> \> push all edges of $P_m$, but $e$, onto $OutsideStack$; \\
\> \> end \\
\> \> If ( $P_l$ has empty $OpenList( P_l )$ and empty $ClosedList( P_l )$, \\
\> \> \mbox{\hspace{3in}} and $P_l$ is not classified yet ) \\
\> \> begin \\
\> \> \> Classify $P_l$ as outside; \\
\> \> \> push all edges of $P_l$, but $e$, onto $OutsideStack$; \\
\> \> end \\
\> end \\
\end{tabbing}
\end{tt}
\end{bf}

At the end of this process, $M_1$ is completely split into two sets of
polygons, classified as either inside $M_2$ or outside $M_2$.  An
identical spliting process can be applied to ${\cal M}_2$.  The proper
inside/outside lists of ${\cal M}_1$ can now be combined with the
proper list of classified polygons of ${\cal M}_2$ to yield the proper
result of the requested Boolean operation.

While this completes the Boolean operation, vast room for improvements
can be found, with some improvements implemented in IRIT.  For
example, the first step of the intersection computation necessitates
an $O(n^2)$ polygon--polygon intersection tests, where $n$ is in the
order of the number of polygons in ${\cal M}_i$.  Clearly, one can
hope for a better time complexity.  Bounding boxes, hierarchical
bounding boxes, or three dimensional space sweep techniques can all be
applied to reduce the overhead invested in these $O(n^2)$ tests.
Efficient sorting of line segments within a polygon, when forming the
open and closed loops is another optimization consideration.

\section{Library Functions}
\input{prog_man/bool_lib.tex}

%%%%%%%%%%%%%%%%%%%%%%%%%%%%%%%%%%%%%%%%%%%%%%%%%%%%%%%%%%%%%%%%%%%%%%%%%%%%%
\chapter{CAGD Library, cagd\_lib}

\section{General Information}

This library provides a rich set of function to create, convert,
display and process freeform Bezier and NURBs curves and surfaces. The
interface of the library is defined in {\em include/cagd\_lib.h}. This
library mainly supports low level freeform curve and surface
operations.  Supported are curves and surfaces from scalars to five
dimensions as E1/P1 to E5/P5 using the {\bf CagdPointType}. Pi is a
rational (projective) version of Ei, with an additional W coefficient.
Polynomial in the power basis have some very limited support as well.
Different data structures to hold UV parameter values, control points,
vectors, planes, bounding boxes, polylines and polygons are defined as
well as the data strcutures to hold the curves and surfaces
themselves,
\begin{verbatim}
typedef struct CagdCrvStruct {
    struct CagdCrvStruct *Pnext;
    struct IPAttributeStruct *Attr;
    CagdGeomType GType;
    CagdPointType PType;
    int Length;            /* Number of control points (== order in Bezier). */
    int Order;	    /* Order of curve (only for Bspline, ignored in Bezier). */
    CagdBType Periodic;			   /* Valid only for Bspline curves. */
    CagdRType *Points[CAGD_MAX_PT_SIZE];     /* Pointer on each axis vector. */
    CagdRType *KnotVector;
} CagdCrvStruct;

typedef struct CagdSrfStruct {
    struct CagdSrfStruct *Pnext;
    struct IPAttributeStruct *Attr;
    CagdGeomType GType;
    CagdPointType PType;
    int ULength, VLength;	 /* Mesh size in the tensor product surface. */
    int UOrder, VOrder;   /* Order in tensor product surface (Bspline only). */
    CagdBType UPeriodic, VPeriodic;      /* Valid only for Bspline surfaces. */
    CagdRType *Points[CAGD_MAX_PT_SIZE];     /* Pointer on each axis vector. */
    CagdRType *UKnotVector, *VKnotVector;
} CagdSrfStruct;
\end{verbatim}

Curves and surfaces have a geometric type {\bf GType} to prescribe the
type of entity (such as CAGD\_SBEZIER\_TYPE for Bezier surface) and a
point type {\bf PType} to prescribe the point type of the entity (such
as CAGD\_PT\_E3\_TYPE for three dimensional Euclidean control points).
{\bf Length} and {\bf Order} slots are used to hold the number of
control points in the mesh and or control polygon and the order(s) of
the basis functions. {\bf Periodic} flag(s) are used to denote
periodic end conditions. In addition, {\bf KnotVector} slot(s) are
used if the entity exploits Bspline basis functions, or NULL
otherwise.

The control polygon and/or mesh itself is organized in the {\bf
Points} slot as a vector of size {\bf CAGD\_MAX\_PT\_SIZE} of vectors
of {\bf CagdRType}s. For surfaces, the mesh is ordered U first and the
macros of {\bf CAGD\_NEXT\_U} {\bf CAGD\_NEXT\_V}, and {\bf
CAGD\_MESH\_UV} can be used to determine indices in the mesh.

All structures in the cagd library can be allocated using New
constrcutures (i.e. {\bf CagdUVNew} or {\bf CagdCrfNew}, freed using
Free destructores (i.e. {\bf CagdSrfFree} or {CagdBBoxFree}, linked
list free using FreeList destructores (i.e. {\bf
CagdPolylineFreeList}), and copied using copy constructores {i.e. {\bf
CagdPtCopy} or {\bf CagdCtlPtCopyList}).

This library has its own error handler, which by default prints an
error message and exit the program called {\bf CagdFatalError}.

Most globals in this library have a prefix of {\bf Cagd} for general
cagd routines. Prefix of {\bf Bzr} is used for Bezier routines, prefix
of {\bf Bsp} for Bspline specific routines, prefix of {\bf Cnvrt} for
conversion routines, and {\bf Afd} for adaptive forward differencing
routines.

\section{Library Functions}
\input{prog_man/cagd_lib.tex}

%%%%%%%%%%%%%%%%%%%%%%%%%%%%%%%%%%%%%%%%%%%%%%%%%%%%%%%%%%%%%%%%%%%%%%%%%%%%%
\chapter{Geometry Library, geom\_lib}

\section{General Information}

This library handles general computational geometry algorithms and
geometric queries such as a distance between two lines, bounding
boxes, convexity and convex hull of polygons, polygonal constructors
of primitives (cylinders, spheres, etc.), basic scan conversion
routines, etc.

\section{Library Functions}
\input{prog_man/geom_lib.tex}

%%%%%%%%%%%%%%%%%%%%%%%%%%%%%%%%%%%%%%%%%%%%%%%%%%%%%%%%%%%%%%%%%%%%%%%%%%%%%
\chapter{Graphics Library, grap\_lib}

\section{General Information}

This library handles general drawing and display algorithms, including
tesselation of all geometric objects, such as curves and surfaces, into
displayable primitives, i.e. polygons and polylines.

\section{Library Functions}
\input{prog_man/grap_lib.tex}

%%%%%%%%%%%%%%%%%%%%%%%%%%%%%%%%%%%%%%%%%%%%%%%%%%%%%%%%%%%%%%%%%%%%%%%%%%%%%
\chapter{Model Library, mdl\_lib}

\section{General Information}

This library provides the necessary tools to represent and process
models.  Models are sets of trimmed surfaces forming a closed
2-manifold shell.  Models are typically the result of Boolean operations
over freeform (trimmed) NURBs geometry but can also be constructed
directly or via other schemes.

\section{Library Functions}
\input{prog_man/mdl_lib.tex}

%%%%%%%%%%%%%%%%%%%%%%%%%%%%%%%%%%%%%%%%%%%%%%%%%%%%%%%%%%%%%%%%%%%%%%%%%%%%%
\chapter{Miscelleneous Library, misc\_lib}

\section{General Information}

This library holds general miscelleneous functions such as reading
configuration files, low level homogeneous matrices computation, low
level attributes and general machine specific low level routines.
Several header files can be found for this library:
\begin{center}
\begin{tabular}{||c|c||} \hline
    Header (include/*.h) & Functionality \\ \hline
    config.h   & Manipulation of configuration files (*.cfg files) \\
    dist\_pts.h & Enegry based distribution of points. \\
    gen\_mat.h  & Homogeneous matrices manipulation \\
    getarg.h   & Command line parsing for application tools \\
    imalloc.h  & Low level dynamic memory functions for IRIT \\
    miscattr.h & Low level attribute related functions \\
    priorque.h & An implementation of a priority queue \\
    xgeneral.h & Low level, machine specific, routines \\ \hline
\end{tabular}
\end{center}

\section{Library Functions}
\input{prog_man/misc_lib.tex}

%%%%%%%%%%%%%%%%%%%%%%%%%%%%%%%%%%%%%%%%%%%%%%%%%%%%%%%%%%%%%%%%%%%%%%%%%%%%%
\chapter{Multi variate functions, mvar\_lib}

\section{General Information}

This library holds functions to handle functions of arbitrary number
of variables. In this context curves (univariate), surfaces
(bivariate) and trivariates are special cases. This library provides a
rich set of functions to manipulate freeform Bezier and/or NURBs
multivariates. This library heavily depends on the cagd an symb
libraries. Functions are provided to create, copy, and destruct
multivariates, to extract isoparametric lower degree multivariates, to
evaluate, refine and subdivide, to read and write multivariates, to
differentiate, degree raise, make compatible and convert back and
forth to/from curves, surfaces, and trivariates.

A multivariate has m orders, m Length prescriptions and,
possibly, m knot vectors (if Bspline).  In addition it contains
an m dimensional volume of control points,
\begin{verbatim}
typedef struct MvarMVStruct {
    struct MvarMVStruct *Pnext;
    struct IPAttributeStruct *Attr;
    MvarGeomType GType;
    CagdPointType PType;
    int Dim;		      /* Number of dimensions in this multi variate. */
    int *Lengths;               /* Dimensions of mesh size in multi-variate. */
    int *SubSpaces;	   /* SubSpaces[i] = Prod(i = 0, i-1) of Lengths[i]. */
    int *Orders;                /* Orders of multi varariate (Bspline only). */
    CagdBType *Periodic;            /* Periodicity - valid only for Bspline. */
    CagdRType *Points[CAGD_MAX_PT_SIZE];     /* Pointer on each axis vector. */
    CagdRType **KnotVectors;
} MvarMVStruct;
\end{verbatim}

The interface of the library is defined in {\em include/mvar\_lib.h}. 

This library has its own error handler, which by default prints an
error message and exit the program called {\bf MvarFatalError}.

All globals in this library have a prefix of {\bf Mvar}.

\section{Library Functions}
\input{prog_man/mvar_lib.tex}

%%%%%%%%%%%%%%%%%%%%%%%%%%%%%%%%%%%%%%%%%%%%%%%%%%%%%%%%%%%%%%%%%%%%%%%%%%%%%
\chapter{Prsr Library, prsr\_lib}

\section{General Information}

This library provides the data file interface for IRIT. Functions are
provided to read and write data files, both compressed (on unix only,
using {\em compress}), and uncompressed, in binary and/or ascii text
modes. This library is also used to exchange data between the IRIT
server and the display devices' clients.  Several header files can be
found for this library:
\begin{center}
\begin{tabular}{||c|c||} \hline
    Header (include/*.h) & Functionality \\ \hline
    allocate.h & High level dynamic allocation of objects \\
    attribut.h & High level attributes for objects \\
    ip\_cnvrt.h & Freeform to polygon and polyline high level conversion \\
    iritprsr.h & Main interface to reading and writing data \\
    irit\_soc.h & Socket communication for data exchange \\ \hline
\end{tabular}
\end{center}

\section{Library Functions}
\input{prog_man/prsr_lib.tex}

%%%%%%%%%%%%%%%%%%%%%%%%%%%%%%%%%%%%%%%%%%%%%%%%%%%%%%%%%%%%%%%%%%%%%%%%%%%%%
\chapter{Rendering Library, rndr\_lib}

\section{General Information}

This library provides a powerful full screen scan conversion Z-buffer
tool to process IRIT geometry and convert it into images.  This
library allows one to scan convert any IRIT geometry including
polylines and curves that are converted to skinny polygons on the fly.
The library offers regular scan conversion with flat, Gouraud, and
Phong shading and several antialiasing filters along with advanced
features such as transparency and animation support, and width depth
cueing on polyline/curves rendering.  The library also provide direct
access to the depth Z-buffer as well as a stencil buffer.

\section{Library Functions}
\input{prog_man/rndr_lib.tex}

%%%%%%%%%%%%%%%%%%%%%%%%%%%%%%%%%%%%%%%%%%%%%%%%%%%%%%%%%%%%%%%%%%%%%%%%%%%%%
\chapter{Symbolic Library, symb\_lib}

\section{General Information}

This library provides a rich set of functions to symbolically manipulate
freeform curves and surfaces. This library heavily depends on the cagd
library. Functions are provided to low level add, subtract, and multiply
freeform curves and surfaces, to compute fields such as curvature,
and to extract singular points such as extremums, zeros, and inflections.
High level tools to metamorph curves and surfaces, to compute layout (prisa)
of freeform surfaces, to compute offset approximations of curves and
surfaces, and to compose curves and surfaces are also provided.

The interface of the library is defined in {\em include/symb\_lib.h}. 

This library has its own error handler, which by default prints an
error message and exit the program called {\bf SymbFatalError}.

Globals in this library have a prefix of {\bf Symb} for general
symbolic routines. Prefix of {\bf Bzr} is used for Bezier routines,
and prefix of {\bf Bsp} for Bspline specific routines.

\section{Library Functions}
\input{prog_man/symb_lib.tex}

%%%%%%%%%%%%%%%%%%%%%%%%%%%%%%%%%%%%%%%%%%%%%%%%%%%%%%%%%%%%%%%%%%%%%%%%%%%%%
\chapter{Trimmed surfaces Library, trim\_lib}

\section{General Information}

This library provides a set of functions to manipulate freeform
trimmed Bezier and/or NURBs surfaces. This library heavily depends on
the cagd library. Functions are provided to create, copy, and destruct
trimmed surfaces to extract isoparametric curves, to evaluate, refine
and subdivide, to read and write trimmed surfaces, degree raise, and
approximate using polygonal representations.  A trimming surface is
defined out of a tensor product surface and a set of trimming loops
that trims out portions of the parametric space of the surface,
\begin{verbatim}
typedef struct TrimSrfStruct {
    struct TrimSrfStruct *Pnext;
    IPAttributeStruct *Attr;
    int Tags;
    CagdSrfStruct *Srf;			  /* Surface trimmed by TrimCrvList. */
    TrimCrvStruct *TrimCrvList;		         /* List of trimming curves. */
} TrimSrfStruct;
\end{verbatim}

Each trimming loop consists of a set of trimming curve segments:
\begin{verbatim}
typedef struct TrimCrvStruct {
    struct TrimCrvStruct *Pnext;
    IPAttributeStruct *Attr;
    TrimCrvSegStruct *TrimCrvSegList;    /* List of trimming curve segments. */
} TrimCrvStruct;
\end{verbatim}

Each trimming curve segment contains a representation for the curve in the
UV space of the surface as well as a representation in the Euclidean space,
\begin{verbatim}
typedef struct TrimCrvSegStruct {
    struct TrimCrvSegStruct *Pnext;
    IPAttributeStruct *Attr;
    CagdCrvStruct *UVCrv;    /* Trimming crv segment in srf's param. domain. */
    CagdCrvStruct *EucCrv;       /* Trimming curve as an E3 Euclidean curve. */
} TrimCrvSegStruct;
\end{verbatim}


The interface of the library is defined in {\em include/trim\_lib.h}. 

This library has its own error handler, which by default prints an
error message and exit the program called {\bf TrimFatalError}.

All globals in this library have a prefix of {\bf Trim}.

\section{Library Functions}
\input{prog_man/trim_lib.tex}

%%%%%%%%%%%%%%%%%%%%%%%%%%%%%%%%%%%%%%%%%%%%%%%%%%%%%%%%%%%%%%%%%%%%%%%%%%%%%
\chapter{Trivariate Library, triv\_lib}

\section{General Information}

This library provides a rich set of functions to manipulate freeform
Bezier and/or NURBs trivariate. This library heavily depends on the cagd
library. Functions are provided to create, copy, and destruct trivariates,
to extract isoparametric surfaces, to evaluate, refine and subdivide, to
read and write trivariates, to differentiate, degree raise, make compatible
and approximate iso-surface at iso values using polygonal representations.

A trivariate has three orders, three Length prescriptions and,
possibly, three knot vectors (if Bspline).  In addition it contains
a three dimensional volume of control points,
\begin{verbatim}
typedef struct TrivTVStruct {
    struct TrivTVStruct *Pnext;
    struct IPAttributeStruct *Attr;
    TrivGeomType GType;
    CagdPointType PType;
    int ULength, VLength, WLength;/* Mesh size in tri-variate tensor product.*/
    int UVPlane;	  /* Should equal ULength * VLength for fast access. */
    int UOrder, VOrder, WOrder;       /* Order in trivariate (Bspline only). */
    CagdBType UPeriodic, VPeriodic, WPeriodic;    /* Valid only for Bspline. */
    CagdRType *Points[CAGD_MAX_PT_SIZE];     /* Pointer on each axis vector. */
    CagdRType *UKnotVector, *VKnotVector, *WKnotVector;
} TrivTVStruct;
\end{verbatim}

The interface of the library is defined in {\em include/triv\_lib.h}. 

This library has its own error handler, which by default prints an
error message and exit the program called {\bf TrivFatalError}.

All globals in this library have a prefix of {\bf Triv}.

\section{Library Functions}
\input{prog_man/triv_lib.tex}

%%%%%%%%%%%%%%%%%%%%%%%%%%%%%%%%%%%%%%%%%%%%%%%%%%%%%%%%%%%%%%%%%%%%%%%%%%%%%
\chapter{Triangular Library, trng\_lib}

\section{General Information}

This library provides a subset of functions to manipulate freeform
triangular Bezier and Bspline patches. This library heavily depends on
the cagd library. Functions are provided to create, copy, and destruct
triangular patches, to extract isoparametric curves, to evaluate,
refine and subdivide, to read and write triangular patches, to
differentiate, and approximate using polygonal representations.

A triangular patch has one prescription of Length and one prescription
of Order, the total Order and the length of an edge of the triangle.
The control mesh mesh has Length * (Length + 1) / 2 control points,
\begin{verbatim}
typedef struct TrngTriangSrfStruct {
    struct TrngTriangSrfStruct *Pnext;
    struct IPAttributeStruct *Attr;
    TrngGeomType GType;
    CagdPointType PType;
    int Length;		    /* Mesh size (length of edge of triangular mesh. */
    int Order;		      /* Order of triangular surface (Bspline only). */
    CagdRType *Points[CAGD_MAX_PT_SIZE];     /* Pointer on each axis vector. */
    CagdRType *KnotVector;
} TrngTriangSrfStruct;
\end{verbatim}

The interface of the library is defined in {\em include/trng\_lib.h}. 

This library has its own error handler, which by default prints an
error message and exit the program called {\bf TrngFatalError}.

All globals in this library have a prefix of {\bf Trng}.

\section{Library Functions}
\input{prog_man/trng_lib.tex}

%%%%%%%%%%%%%%%%%%%%%%%%%%%%%%%%%%%%%%%%%%%%%%%%%%%%%%%%%%%%%%%%%%%%%%%%%%%%%
\chapter{User Library, user\_lib}

\section{General Information}

This library includes user interface related geometrical functions such
as ray surface intersection (for mouse click/select operations), etc.

The interface of the library is defined in {\em include/user\_lib.h}. 

\section{Library Functions}
\input{prog_man/user_lib.tex}

%%%%%%%%%%%%%%%%%%%%%%%%%%%%%%%%%%%%%%%%%%%%%%%%%%%%%%%%%%%%%%%%%%%%%%%%%%%%%
\chapter{Extra Library, xtra\_lib}

\section{General Information}

This library is not an official part of IRIT and contains public
domain code that is used by routines in IRIT.

The interface of the library is defined in {\em include/extra\_fn.h}. 

\section{Library Functions}
\input{prog_man/xtra_lib.tex}

%%%%%%%%%%%%%%%%%%%%%%%%%%%%%%%%%%%%%%%%%%%%%%%%%%%%%%%%%%%%%%%%%%%%%%%%%%%%%
\chapter{Programming Examples}
\label{chap-examples}

This chapter describes several simple examples of C programs that
exploits the libraries of IRIT. All external function are defined in
the {\em include} subdirectory of IRIT and one can 'grep' there for the
exact include file that contains a certain function name. All C programs
in all C files should include 'irit\_sm.h' as their first include file
of IRIT, before any other include file of IRIT. Header files are set so
C++ code can compile and link to it without any special treatment.

\section{Setting up the Compilation Environment}

In order to compile programs that uses the libraries of IRIT, a makefile
has to be constructed. Assuming IRIT is installed in {\em /usr/local/irit},
here is a simple makefile that can be used (for a unix environment):

\begin{verbatim}
IRIT_DIR = /usr/local/irit

include $(IRIT_DIR)/makeflag.unx

OBJS	= program.o

program: $(OBJS)
	$(CC) $(CFLAGS) -o program $(OBJS) $(LIBS) -lm $(MORELIBS)
\end{verbatim}

The simplicity of this makefile is drawn from the complexity of
makeflag.unx. The file makeflag.unx sets the CC, CFLAGS, LIBS, and
MORELIBS for the machined using among other things. Furthermore,
makeflag.unx also sets the default compilation rules from C sources to
object files.  The file makeflag.unx had to be modified once, when
IRIT was installed on this system. If the used system is not a unix
environment, then the file makefile.unx will have to be replaced with
the proper makeflag file.  In an OS2 environment, using the emx gcc
compiler, the makefile is even simpler since the linking rule is also
defined in makeflag.os2:

\begin{verbatim}
IRIT_DIR = \usr\local\irit

include $(IRIT_DIR)\makeflag.os2

OBJS	= program.o

program.exe: $(OBJS)
\end{verbatim}

Finally, here is a proper makefile for Windows NT:

\begin{verbatim}
IRIT_DIR = \usr\local\irit

include $(IRIT_DIR)\makeflag.wnt

OBJS	= program.obj

program.exe: $(OBJS)
        $(IRITCONLINK) -out:program.exe $(OBJS) $(LIBS) $(W32CONLIBS)
\end{verbatim}

\section{Simple C Programs using IRIT}

Now that we have an idea how to compile C code using IRIT, here are several
examples to read, manipulate and write IRIT data files. You will be able to
find all these examples in the doc/cexample directory.

\subsection{Boolean Operations over Polyhedra (boolean.c)}

This example demonstates the use of the Boolean operations package
(bool\_lib), creates few polygonal primitives and apply Boolean
operations between them.  The resulting model is dumped to stdout.

\input{cexample/boolean.tex}

\subsection{Distance Maps to Planar Curves (dist\_map.c)}

This example demonstates the use of symbolic distance maps
computations to freeform curves, in the plane, and the sampling
of the map into an image that is dumped out.

\input{cexample/dist_map.tex}

\subsection{Importance Edges Evaluations in Polygonal Meshes (imprtnc.c)}

This example demonstates the tesselation of arbitrary input model(s)
into polygons and the probabalistic estimation of importance edges in the
polygonal geometry.  Dumped out is the polygonal geometry with vertices
having UV coordinates if available, normals if available, and importance.

\input{cexample/imprtnc.tex}

\subsection{Least sqaures curve fitting and process communication (lst\_sqrs.c)}

This example creates a Bspline curve that least squares approximates
a given set of data points. In the default values, a quadratic Bspline
curve of 10 control points least sqaures approximates a set of 100
three dimensional points. Also shown is a polyline of the original
(100) points. Arbitrary number of curves will be displayed every, one
for every <return> keystroke until Q<return> is typed.

\input{cexample/lst_sqrs.tex}

\subsection{Multivariate Solver (msolve.c)}

This example demonstates the use of the multivariate solver from the
mvar\_lib library.  Input is a set of n polynomial equations in m
variables and dumped out are the solutions to these equations.

\input{cexample/msolve.tex}

\subsection{Compute Area of a Polygonal Model (polyarea.c)}

Here is a simple program to compute the total area of all polygons in
the given data file. The program expects one argument on the command
line which is the name of the file to read, and it prints out one
line with the total computed area.

\input{cexample/polyarea.tex}

\subsection{Converts a Freeform Surface into Polygons (polygons.c)}

This true filter reads a single surface from stdin and dumps out a
polygonal approximation of it to stdout. They are several parameters
that controls the way a surface is approximated into polygons and in
this simple filter they are being held fixed in a set of integer
variables.

\input{cexample/polygons.tex}

\subsection{Linear Transformations' Filter (transfrm.c)}

This little more complex program transfroms all the geometry in the
read data which can be any number of files according to the specified
transformations on the command line. The command line is parsed via
{\bf GAGetArgs} and its associated functions. The transformation
matrix is then computed with the aid of the matrix package and applied
to the read geometry at once.

\input{cexample/transfrm.tex}

Here is the result of running 'transfrm -h':
\begin{verbatim}
This is Transform...
Usage: Transfrm [-x Degs] [-y Degs] [-z Degs] [-t X Y Z] [-s Scale] [-h] DFiles
\end{verbatim}

When you are considering the usefulness of this tool remember that the
transformations are applied to the geometry in an internal order which is
different from the command line order. That is,
\begin{verbatim}
     transfrm -x 30 -y 30 geometry.dat > tgeometry.dat
\end{verbatim}
will compute the exact same transfrom as,
\begin{verbatim}
     transfrm -y 30 -x 30 geometry.dat > tgeometry.dat
\end{verbatim}
This, while rotation is not a commutative operation.
Nonetheless, you may split the operations. That is:
\begin{verbatim}
     transfrm -y 30 geometry.dat | transfrm -x 30 - > tgeometry.dat
\end{verbatim}
or
\begin{verbatim}
     transfrm -x 30 geometry.dat | transfrm -y 30 - > tgeometry.dat
\end{verbatim}
will do exactly what one expects.

%%%%%%%%%%%%%%%%%%%%%%%%%%%%%%%%%%%%%%%%%%%%%%%%%%%%%%%%%%%%%%%%%%%%%%%%%%%%%
\twocolumn
\input{prog_man.ind}

%%%%%%%%%%%%%%%%%%%%%%%%%%%%%%%%%%%%%%%%%%%%%%%%%%%%%%%%%%%%%%%%%%%%%%%%%%%%%

\end{document}


%%%%%%%%%%%%%%%%%%%%%%%%%%%%%%%%%%%%%%%%%%%%%%%%%%%%%%%%%%%%%%%%%%%%%%%%%%%%%

\end{document}


%%%%%%%%%%%%%%%%%%%%%%%%%%%%%%%%%%%%%%%%%%%%%%%%%%%%%%%%%%%%%%%%%%%%%%%%%%%%%

\end{document}


%%%%%%%%%%%%%%%%%%%%%%%%%%%%%%%%%%%%%%%%%%%%%%%%%%%%%%%%%%%%%%%%%%%%%%%%%%%%%

\end{document}
