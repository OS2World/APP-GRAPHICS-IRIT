\begin{verbatim}
/*****************************************************************************
* Least squares fit a curve to random points.                                     *
* We also see how to fork out a display device and communicate with it.      *
******************************************************************************
* (C) Gershon Elber, Technion, Israel Institute of Technology                *
******************************************************************************
* Written by:  Gershon Elber                                Ver 1.0, June 1995   *
*****************************************************************************/

#include "irit_sm.h"
#include "iritprsr.h"
#include "allocate.h"
#include "attribut.h"
#include "cagd_lib.h"
#include "geom_lib.h"
#include "grap_lib.h"
#include "misc_lib.h"

static char *CtrlStr =
    "Lst_Sqrs n%-#Pts!d d%-Degree!d f%-DOF!d p%-PrgmName!s h%-";

void main(int argc, char **argv)
{
    int i, Error, PrgmIO,
        NumOfPoints = 100,
        NumOfPtsFlag = FALSE,
        Degree = 3,
        DegreeFlag = FALSE,
        NumOfDOF = 10,
        NumOfDOFFlag = FALSE,
        PrgmFlag = FALSE,
        HelpFlag = FALSE;
    char *Err,
        *Program = getenv("IRIT_DISPLAY");

#ifdef __WINNT__
    if (Program == NULL)
        Program = "wntgdrvs -s-";
#endif /* __WINNT__ */
#ifdef __UNIX__
    if (Program == NULL)
        Program = "x11drvs -s-";
#endif /* __UNIX__ */

    if ((Error = GAGetArgs(argc, argv, CtrlStr,
                           &NumOfPtsFlag, &NumOfPoints,
                           &DegreeFlag, &Degree,
                           &NumOfDOFFlag, &NumOfDOF,
                           &PrgmFlag, &Program,
                           &HelpFlag)) != 0) {
        GAPrintErrMsg(Error);
        GAPrintHowTo(CtrlStr);
        exit(1);
    }

    if (HelpFlag) {
        GAPrintHowTo(CtrlStr);
        exit(0);
    }

    IPSocSrvrInit();            /* Initialize the listen socket for clients. */

    if ((PrgmIO = IPSocExecAndConnect(Program,
                                      getenv("IRIT_BIN_IPC") != NULL)) >= 0) {
        char Line[IRIT_LINE_LEN];
        IPObjectStruct
            *PClrObj = IPGenStrObject("command_", "clear", NULL);

        do {
            CagdPtStruct
                *PtList = NULL;
            IPPolygonStruct
                *PPoly = IPAllocPolygon(0, NULL, NULL);
            CagdCrvStruct *Crv;
            IPObjectStruct *PCrvObj, *PPolyObj;

            for (i = 0; i < NumOfPoints; i++) {
                int j;
                IPVertexStruct *V;
                CagdPtStruct
                    *Pt = CagdPtNew();

                if (i == 0) {
                    for (j = 0; j < 3; j++)
                        Pt -> Pt[j] = IritRandom(-1.0, 1.0);
                }
                else {
                    for (j = 0; j < 3; j++)
                        Pt -> Pt[j] = PtList -> Pt[j] + IritRandom(-0.1, 0.1);
                }

                V = IPAllocVertex(0, NULL, PPoly -> PVertex);
                for (j = 0; j < 3; j++)
                    V -> Coord[j] = Pt -> Pt[j];
                PPoly -> PVertex = V;

                IRIT_LIST_PUSH(Pt, PtList);
            }

            Crv = BspCrvInterpPts(PtList, Degree + 1,
                                  NumOfDOF, CAGD_UNIFORM_PARAM, FALSE);
            CagdPtFreeList(PtList);

            CagdCrvWriteToFile3(Crv, stdout, 0, "This is from LstSqrs", &Err);

            /* Generate objects out of the geometry and set proper attrs. */
            PCrvObj = IPGenCRVObject(Crv);
            AttrSetObjectColor(PCrvObj, IG_IRIT_CYAN);

            PPolyObj = IPGenPOLYObject(PPoly);
            IP_SET_POLYLINE_OBJ(PPolyObj);
            AttrSetObjectColor(PPolyObj, IG_IRIT_YELLOW);

            /* Clear old data and display our curve and data. */
            IPSocWriteOneObject(PrgmIO, PClrObj);
            IPSocWriteOneObject(PrgmIO, PCrvObj);
            IPSocWriteOneObject(PrgmIO, PPolyObj);

            IPFreeObject(PCrvObj);
            IPFreeObject(PPolyObj);

            gets(Line);
        }
        while (Line[0] != 'q' && Line[0] != 'Q');

        IPSocDisConnectAndKill(TRUE, PrgmIO);
    }

    exit(0);
}
\end{verbatim}
